\documentclass[xcolor=dvipsnames]{beamer}

% CUSTOM COMMANDS FOR ALGEBRAIC TOPOLOGY
% ----------------------------------------------------------

% Restriction of a function
\newcommand{\rest}[1]{\raisebox{-.5ex}{$|$}_{#1}}

% Real numbers
\newcommand{\R}{\mathbb{R}}
\newcommand{\PR}{\mathbb{PR}}

% Rational numbers
\newcommand{\Q}{\mathbb{Q}}

% Complex numbers
\newcommand{\C}{\mathbb{C}}

% Natural numbers
\newcommand{\N}{\mathbb{N}}

% Integers
\newcommand{\Z}{\mathbb{Z}}

% Vector bold
\renewcommand{\vec}[1]{\mathbf{#1}}

% Span
\newcommand{\gen}[1]{\langle #1 \rangle}

% Set
\newcommand{\set}[1]{\{ #1 \}}

% Script A, B, M, P
\newcommand{\A}{\mathcal{A}}
\newcommand{\B}{\mathcal{B}}
\newcommand{\M}{\mathcal{M}}
\renewcommand{\P}{\mathcal{P}}
\renewcommand{\S}{\mathfrak{S}}

% Identity
\newcommand{\id}{\mathrm{id}}

% Kernel and image
\DeclareMathOperator{\im}{im}
\DeclareMathOperator{\coker}{coker}

% Absolute value
\newcommand{\abs}[1]{\left\lvert #1 \right\rvert}

% Norm
\newcommand{\norm}[1]{\left\lVert #1 \right\rVert}

\newcommand{\inn}[2]{\left\langle #1, #2 \right\rangle}

% Category of Vector Spaces
\newcommand{\Vect}{\mathsf{Vect}}
\newcommand{\VectK}{\Vect_{K}}
\newcommand{\VectR}{\Vect_{\R}}
\DeclareMathOperator{\Hom}{Hom}
\DeclareMathOperator{\Bil}{Bil}
\newcommand{\dual}{^{\vee}}
\DeclareMathOperator{\tr}{tr}

% Category of Manifolds
\DeclareMathOperator{\Diff}{Diff}

% Epi and monomorphisms
\newcommand{\onto}{\twoheadrightarrow}
\newcommand{\into}{\tailrightarrow}

\newcommand{\parbreak}{
	\begin{center}
		--- $\ast$ ---
	\end{center} 
}

% Defined as
\makeatletter
\newcommand*{\defeq}{\mathrel{\rlap{%
    \raisebox{0.3ex}{$\m@th\cdot$}}%
  \raisebox{-0.3ex}{$\m@th\cdot$}}%
	=
}
\makeatother

% Support
\DeclareMathOperator{\supp}{supp}

% Categories
\newcommand{\Top}{\mathsf{Top}}

\newcommand{\ud}{\, \mathrm{d}}
\renewcommand{\d}{\mathrm{d}}

\newcommand{\basisAt}[3]{\frac{\partial}{\partial #1^{#2}}\Big |_{#3}}
\newcommand{\basis}[2]{\frac{\partial}{\partial #1^{#2}}}
\newcommand{\at}[1]{\Big |_{#1}}

\newcommand{\phant}{\phantom{\alpha}}
\renewcommand{\L}{\mathcal{L}}

\newcommand{\deriv}[1]{\left.\tfrac{\ud}{\ud #1}\right\vert_{#1=0}}


\newcommand{\highlight}[2]{\colorbox{#1}{\( \displaystyle #2 \)}}

\usepackage[english]{babel}
\usepackage[utf8]{inputenc}
\usepackage[T1]{fontenc}
\usepackage{lmodern}
\usepackage{cancel}

\usepackage[style=alphabetic, citestyle=alphabetic]{biblatex}
\usepackage[english]{babel}
\addbibresource{refs.bib}

\usepackage{graphicx}
\graphicspath{{figs/}{../figs/}}

\usepackage{amsmath, amssymb, mathtools}
\usepackage{booktabs}
\renewcommand{\arraystretch}{2}
\usefonttheme[onlymath]{serif}

\hypersetup{
	urlcolor = {red!50!blue},
	linktoc = page
}

\title[Action-dependent gravitation]{\bfseries{A variational derivation of the field
equations of an action-dependent Einstein-Hilbert Lagrangian}}
\author[Arnau Mas]{Arnau Mas}
\institute[]{\footnotesize{\itshape{Supervised by}} \\ Dr Jordi Gaset}
\date[UAB, Jun 27th 2021]{\small Universitat Autònoma de Barcelona \\ June 27th 2021}
\titlegraphic{\includegraphics[scale=0.1]{logo-uab}}

% Theme settings
\usetheme{Madrid}
\usecolortheme[named=MidnightBlue]{structure}
\useoutertheme[subsection=false]{miniframes}
\useinnertheme{circles}

% Move miniframes to footline
\setbeamertemplate{navigation symbols}{}
\setbeamertemplate{headline}{}
\setbeamertemplate{mini frame in other subsection}{}
\makeatletter
\setbeamertemplate{footline}
  {%
    \begin{beamercolorbox}[colsep=1.5pt]{lower separation line foot}
    \end{beamercolorbox}
      \begin{beamercolorbox}[colsep=1.5pt]{upper separation line head}
  \end{beamercolorbox}
  \begin{beamercolorbox}{section in head/foot}
    \vskip2pt\insertnavigation{\paperwidth}\vskip4pt
  \end{beamercolorbox}%
  \begin{beamercolorbox}[colsep=1.5pt]{lower separation line head}
  \end{beamercolorbox}
  }
\makeatother

% Add logo to title
\setbeamerfont{frametitle}{series=\bfseries}
\setbeamertemplate{frametitle}{%
  \nointerlineskip
    \begin{beamercolorbox}[sep=0.5ex,wd=\paperwidth,leftskip=.2cm,rightskip=0cm]{frametitle}%
      \usebeamerfont{frametitle}\usebeamercolor[fg]{frametitle}\insertframetitle
      \hfill
      \raisebox{-0.3\height}{\includegraphics[scale = 0.04]{logo-uab-blanc}}
    \end{beamercolorbox}%
}

\begin{document}
\begin{frame}[plain]
	\titlepage
\end{frame}

\AtBeginSection[]{
\begin{frame}
	\frametitle{Outline}
	\tableofcontents[currentsection]
\end{frame}
}


\AtEndDocument{
	\begin{frame}[plain]
		\titlepage
	\end{frame}
}


\begin{frame}
	\frametitle{Outline}
	\tableofcontents
\end{frame}

\section{Introduction}
\begin{frame}
	\frametitle{Background}
	\begin{itemize}
		\pause
		\item Modified theories of gravity: \( f(R) \), Brans-Dicke, etc. to answer questions
			from cosmology
			\pause
		\item Action-dependent Lagrangian mechanics: variational formulation of dissipative
			systems, applied in many other fields (thermodynamics, control theory).
	\end{itemize}
	\pause
	\textbf{Idea:} what does an action-dependent version of Einstein gravity look like?
\end{frame}

\begin{frame}
	\frametitle{Work done}
	\begin{itemize}
		\pause
		\item	Writing down the equations of action-dependent GR has two main challenges: it
			is a field theory and it is a \emph{second order} field theory. 
			\pause
		\item	An attempt was made in \cite{Lazo2018}, but we claim the equations found there
			are not the correct ones. We will use an alternative formulation of the
			action-dependent theory which allows us to treat second order theories. 
	\end{itemize}

\end{frame}

\section{The Herglotz variational problem}
\begin{frame}
	\frametitle{Herglotz problem: implicit version}
	If we allow the Lagrangian to depend on the action, then we have
	\begin{equation*}
		S[q^\mu](t) = \int_a^t L(q^\mu(\tau), \dot{q}^\mu(\tau), S[q^\mu](\tau)) \ud \tau
	\end{equation*}
	\pause and differentiating
	\begin{equation*}
		\dot{S}[q^\mu](t) = L(q^\mu(t), \dot{q}^\mu(t), S[q^\mu](t)).
	\end{equation*}
	\pause So, solve this ODE for any given trajectory and then find which paths are extrema of \(
	S[q^\mu](a) - S[q^\mu](b) \). \pause Not very elegant! 
\end{frame}

\begin{frame}
	\frametitle{Herglotz problem: constrained version}
	Consider expanded trajectories \( q^\mu, z \), so \( z \) tracks \emph{some quantity}
	along the path. \pause If we want \( z \) to be the change in action then we require
	\begin{equation*}
		\dot{z}(t) = L(q^\mu(t), \dot{q}^\mu(t), z(t)).
	\end{equation*}
	\pause So the action functional becomes
	\begin{equation*}
		S[q^\mu, z] = z(b) - z(a).
	\end{equation*}
	\pause We want to find the extrema of this functional \pause \emph{subject to the constraint}
	that \( z \) actually be the action. 
\end{frame}

\begin{frame}
	\frametitle{Lagrange multipliers}
	Turns out, finding the extrema subject to the constraint is equivalent to finding the
	extrema of 
	\begin{align*}
		\action<+->{\tilde{S}[q^\mu, z, \lambda] 
		& = S[q^\mu, z] - \int_a^b \lambda(t)\big[\dot{z}(t) -
	L(q^\mu(t), \dot{q}^\mu(t), z(t))\big] \ud t \\}
		\action<+->{& = \int_a^b \dot{z}(t) \ud t - \int_a^b \lambda(t)\big[\dot{z}(t) -
		L(q^\mu(t), \dot{q}^\mu(t), z(t))\big] \ud t \\}
		\action<+->{& = \int_a^b (1 - \lambda(t)) \dot{z}(t) + \lambda(t)\big(L(q^\mu(t),
		\dot{q}^\mu(t), z(t))\big) \ud t.}
	\end{align*}	
	\pause The Euler-Lagrange equations of this functional are
	\begin{equation*}
		\frac{\partial L}{\partial q^\mu} - \frac{d}{d t}\frac{\partial L}{\partial \dot{q}^\mu}
		+\frac{\partial L}{\partial \dot{q}^\mu} \frac{\partial L}{\partial z} = 0.
	\end{equation*}
	These are the Herglotz equations. 
\end{frame}

\begin{frame}
	\frametitle{Example: damped harmonic oscillator}
	The harmonic oscillator with linear action coupling is \pause
	\begin{equation*}
		L(q, \dot{q}, z) = \tfrac{1}{2}m\dot{q}^2 - \tfrac{1}{2} m\omega^2 q^2 - \gamma z. 
	\end{equation*}
	\pause The equation of motion is \pause
	\begin{equation*}
		\ddot{q} + \gamma\dot{q} + \omega^2 q = 0,
	\end{equation*}
	which is the equation of a damped harmonic oscillator. 
\end{frame}

\begin{frame}
	\frametitle{Generalisation to field theory}
	There is a subtlety when thinking of action-dependent Lagrangians for field theory. We
	must replace \( z \) by a 3-form \( z = z^\mu \ud x_\mu \) which we call the action
	flux. \pause The action flux does \emph{not} transform like a vector. 

	\pause The Herglotz equations for field theory are
	\begin{equation*}
		\frac{\partial L}{\partial \phi^a} - \partial_\mu \frac{\partial
			L}{\partial(\partial_\mu \phi^a)} + \frac{\partial L}{\partial z^\mu} \frac{\partial
		L}{\partial(\partial_\mu \phi^a)} = 0. 
	\end{equation*}
\end{frame}

\begin{frame}
	\frametitle{Generalisation to field theory}
	\begin{tabular}{rcc}
		\toprule
		& Mechanics & Field theory \\
		\midrule
		\pause
		Action flux & \( z(t) \) & \( z^\mu(x^\nu) \ud x_\mu \) \\
		\pause
		Action functional & \( S[q^\mu, z] = z(b) - z(a) \) & \( S[\phi^a,
		z^\mu] = \int_{\partial D} z \) \\
		\pause
		Constraint & \( \dot{z} = L(q^\mu, \dot{q}^\mu, z) \) & \( \ud z = \mathcal{L}(\phi^a,
		\partial_\mu \phi^a, z^\mu) \) \\
		\bottomrule 
	\end{tabular}
\end{frame}

\section{Action-dependent Einstein gravity}
\begin{frame}
	\frametitle{Einstein-Hilbert Lagrangian}
	Recall the standard Einstein-Hilbert Lagrangian
	\begin{equation*}
		\mathcal{L}_{\text{E-H}} = R\omega_g = R \sqrt{g} \ud^4 x
	\end{equation*}
	where
	\begin{equation*}
		R = g^{ab}R_{ab} = g^{ab}(\partial_m{\Gamma^m}_{ab} - \partial_a {\Gamma^m}_{mb} +
		{\Gamma^m}_{mn} {\Gamma^n}_{ab} - {\Gamma^m}_{an}{\Gamma^n}_{mb}).
	\end{equation*}
	\pause It's second order in \( g \)!

	\pause The Euler-Lagrange equations of this Lagrangian are the Einstein field equations:
	\begin{equation*}
		R_{ab} - \tfrac{1}{2}g_{ab}R = 0
	\end{equation*}
	10 second order PDEs. 
\end{frame}

\begin{frame}
	\frametitle{Action flux coupling}
	How do we couple the action-flux to the Lagrangian? \pause Simplest is linear coupling,
	\begin{equation*}
		\mathcal{L}_\text{E-H}(g_{ab}, g_{ab,\mu}, g_{ab,\mu\nu}, z^\mu) = R\omega_g - \theta
		\wedge z.
	\end{equation*}
	\pause \( \theta \) must be a covector, \( \theta = \theta_\mu \ud x^\mu \), the
	\emph{dissipation form}. 
	
	\pause In coordinates
	\begin{equation*} 
		\theta \wedge z = \theta_\mu z^\mu \ud^4 x.
	\end{equation*}

	\pause This means the constraint is
	\begin{equation*}
		\partial_\mu z^\mu = R\sqrt{g} - \theta_\mu z^\mu.
	\end{equation*}
\end{frame}

\begin{frame}
	\frametitle{Derivation of the modified field equations}
	Let's sketch the derivation of the equations. \pause Write out the expanded action
	\begin{equation*}
		\tilde{S}[g_{ab}, z^\nu] = \int_D \big[(1-\lambda) \partial_\mu z^\mu +
		\lambda(R\sqrt{g} - \theta_\mu z^\mu)\big] \ud^4 x.
	\end{equation*}
	\pause Let's compute the variation of this:
	\begin{align*}
		\delta \tilde{S}[g_{ab}, z^\nu]
		& = \int_D \big[(1-\lambda) \delta \partial_\mu z^\mu + \lambda(\delta(R\sqrt{g}) -
		\theta_\mu \delta z^\mu)\big] \ud^4 x  \\
		\action<+->{& = \alert<+>{\int_D (1 - \lambda) \partial_\mu
			\delta z^\mu - \lambda \theta_\mu \delta z^\mu \ud^4x}
		+ \alert<+>{\int_D \lambda \delta(R \sqrt{g}) \ud^4 x}.}
	\end{align*}

	\pause The variations of the action flux and the metric decouple. 
\end{frame}

\begin{frame}
	\frametitle{Variation of the action flux}
	Integrating by parts in the variation of the action flux, 
	\only<+-+(2)>{
		\begin{align*}
	& \action<+->{\int_D (1 - \lambda) \partial_\mu \delta z^\mu - \lambda
	\theta_\mu \delta z^\mu \ud^4x \\ }
	& \action<+>{\quad = \int_{D} \partial_\mu \big((1 - \lambda)\delta z^\mu\big)
	\ud^4 x + \int_D (\partial_\mu \lambda - \lambda\theta_\mu) \delta z^\mu \ud^4 x}
		\end{align*}
	}
	\only<+->{
		\begin{align*}
	& \int_D (1 - \lambda) \partial_\mu \delta z^\mu - \lambda
	\theta_\mu \delta z^\mu \ud^4x \\ 
	& \quad = \cancel{\int_{D} \partial_\mu \big((1 - \lambda)\delta z^\mu\big)
	\ud^4 x} + \int_D (\partial_\mu \lambda - \lambda\theta_\mu) \delta z^\mu \ud^4 x
		\end{align*}
	}

	\pause
	\pause This must vanish for any variation of \( z^\mu \), so it must be
	\begin{equation*}
		\partial_\mu \lambda = \lambda \theta_\mu. 
	\end{equation*}
\end{frame}

\begin{frame}
	\frametitle{Variation of the metric}
	The variation of the second integral term gives three terms \pause
	\begin{equation*}
		\delta(R\sqrt{g}) = \delta(g^{ab} R_{ab} \sqrt{g}) = \delta g^{ab} R_{ab} \sqrt{g} +
		g_{ab} \delta R^{ab} \sqrt{g} + g^{ab}R_{ab} \delta\sqrt{g}
	\end{equation*}
	\pause So
	\begin{align*}
		\action<+->{& \int_D \lambda \delta(R \sqrt{g}) \ud^4 x =\\}
		\action<+->{& \quad = \int_D \lambda \delta g^{ab} R_{ab} \sqrt{g}
		\ud^4 x + \int_D \lambda g^{ab} \delta R_{ab} \sqrt{g} \ud^4 x + \int_{D} \lambda R
	\delta\sqrt{g} \ud^4 x \\}
		\action<+->{& \quad = \int_D \lambda R_{ab} \delta g^{ab}\sqrt{g}\ud^4x
		+ \int_D \lambda g^{ab} \delta R_{ab} \sqrt{g} \ud^4 x
	- \int_D \tfrac{1}{2} \lambda g_{ab}R \delta g^{ab}\sqrt{g}\ud^4x}
	\end{align*}
\end{frame}

\begin{frame}
	\frametitle{Variation of the metric}
	\begin{align*}
		& \int_D \lambda \delta(R \sqrt{g}) \ud^4 x =\\
		& \quad = \highlight{red!50}{\int_D \lambda R_{ab} \delta g^{ab}\sqrt{g}\ud^4x}
		+ \int_D \lambda g^{ab} \delta R_{ab} \sqrt{g} \ud^4 x
		- \highlight{red!50}{\int_D \tfrac{1}{2} \lambda g_{ab}R \delta g^{ab}\sqrt{g}\ud^4x}
	\end{align*}
	\pause
	If \( \lambda \) weren't there, the second term would vanish and we would get the Einstein
	field equations. To deal with \( \lambda \) we have to integrate by parts twice. 
\end{frame}

\begin{frame}
	\frametitle{Variation of the metric}
	We use 
	\begin{equation*}
		g^{ab} \delta R_{ab} = \nabla_n(g^{ab} {\delta\Gamma^n}_{ab} - g^{nb} \delta
		{\Gamma^m}_{mb})
	\end{equation*}
	\pause so integrating by parts
	\begin{align*}
		\action<+->{& \int_D \lambda g^{ab}\delta R_{ab} \sqrt{g} \ud^4 x = \\}
		\action<+->{& \quad = \int_D \lambda \nabla_n(g^{ab} {\delta\Gamma^n}_{ab} - g^{nb} \delta
			{\Gamma^m}_{mb}) \sqrt{g} \ud^4x \\}
		\action<+->{& \quad = \int_D \nabla_n \left(\lambda (g^{ab} {\delta\Gamma^n}_{ab} - g^{nb}
			\delta {\Gamma^m}_{mb})\right) \sqrt{g}\ud^4 x \\}
		\action<+->{& \qquad - \int_D (\nabla_n \lambda) (g^{ab}
			{\delta\Gamma^n}_{ab} - g^{nb} \delta {\Gamma^m}_{mb}) \sqrt{g} \ud^4 x \\}
		\action<+->{& \quad = - \int_D \lambda \theta_n (g^{ab}
			{\delta\Gamma^n}_{ab} - g^{nb} \delta {\Gamma^m}_{mb}) \sqrt{g} \ud^4 x}
	\end{align*}
\end{frame}

\begin{frame}
	\frametitle{Variation of the metric}
	We use 
	\begin{equation*}
		g^{ab} \delta R_{ab} = \nabla_n(g^{ab} {\delta\Gamma^n}_{ab} - g^{nb} \delta
		{\Gamma^m}_{mb})
	\end{equation*}
	 so integrating by parts
	 \begin{align*}
	& \int_D \lambda g^{ab}\delta R_{ab} \sqrt{g} \ud^4 x = \\
	& \quad = \int_D \lambda \nabla_n(g^{ab} {\delta\Gamma^n}_{ab} - g^{nb} \delta
	{\Gamma^m}_{mb}) \sqrt{g} \ud^4x \\
	& \quad = \int_D \nabla_n \cancel{\left(\lambda (g^{ab} {\delta\Gamma^n}_{ab} - g^{nb}
	\delta {\Gamma^m}_{mb})\right)} \sqrt{g}\ud^4 x \\
	& \qquad - \int_D (\nabla_n \lambda) (g^{ab}
	{\delta\Gamma^n}_{ab} - g^{nb} \delta {\Gamma^m}_{mb}) \sqrt{g} \ud^4 x \\
	& \quad = - \int_D \lambda \theta_n (g^{ab}
	{\delta\Gamma^n}_{ab} - g^{nb} \delta {\Gamma^m}_{mb}) \sqrt{g} \ud^4 x
	 \end{align*}
\end{frame}

\begin{frame}
	\frametitle{Variation of the metric}
	We use 
	\begin{equation*}
		g^{ab} \delta R_{ab} = \nabla_n(g^{ab} {\delta\Gamma^n}_{ab} - g^{nb} \delta
		{\Gamma^m}_{mb})
	\end{equation*}
	 so integrating by parts
	 \begin{align*}
	& \int_D \lambda g^{ab}\delta R_{ab} \sqrt{g} \ud^4 x = \\
	& \quad = \int_D \lambda \nabla_n(g^{ab} {\delta\Gamma^n}_{ab} - g^{nb} \delta
	{\Gamma^m}_{mb}) \sqrt{g} \ud^4x \\
	& \quad = \int_D \nabla_n \cancel{\left(\lambda (g^{ab} {\delta\Gamma^n}_{ab} - g^{nb}
	\delta {\Gamma^m}_{mb})\right)} \sqrt{g}\ud^4 x \\
	& \qquad - \int_D \highlight{red!50}{(\nabla_n \lambda)} (g^{ab}
	{\delta\Gamma^n}_{ab} - g^{nb} \delta {\Gamma^m}_{mb}) \sqrt{g} \ud^4 x \\
	& \quad = - \int_D \highlight{red!50}{\lambda \theta_n} (g^{ab}
	{\delta\Gamma^n}_{ab} - g^{nb} \delta {\Gamma^m}_{mb}) \sqrt{g} \ud^4 x
	 \end{align*}
\end{frame}
	
\begin{frame}
	\frametitle{Variation of the metric}
	We use 
	\begin{equation*}
		\delta {\Gamma^a}_{bc} = \tfrac{1}{2} g^{am}(\nabla_c \delta g_{bm} + \nabla_b \delta
		g_{mc} - \nabla_m \delta g_{bc}).
	\end{equation*}
	\pause so \( g^{ab}\delta R_{ab} \) splits into 6 terms. \pause We will go through one
	in detail
	\begin{align*}
		\action<+->{& - \tfrac{1}{2}\int_D \lambda \theta_n g^{ab} g^{nk} \nabla_b \delta
		g_{ak} \sqrt{g} \ud^4 x \\}
		\action<+->{& \quad = - \tfrac{1}{2}\int_D \lambda \theta_n \nabla_b(g^{ab} g^{nk}
		\delta g_{ak}) \sqrt{g} \ud^4 x \\}
		\action<+->{& \quad = -\tfrac{1}{2}\int_D \nabla_b (\lambda \theta_n g^{ab}
			g^{nk} \delta g_{ak}) \sqrt{g} \ud^4 x + \tfrac{1}{2} \int_D \nabla_b(\lambda
		\theta_n)(g^{ab}g^{nk}\delta g_{ak}) \sqrt{g} \ud^4 x \\}
		\action<+->{& \quad = -\int_D \lambda(\theta_b \theta_n + \nabla_b \theta_n) \delta
		g^{bn} \sqrt{g} \ud^4 x.} 
	\end{align*}
\end{frame}

\begin{frame}
	\frametitle{Variation of the metric}
	We use 
	\begin{equation*}
		\delta {\Gamma^a}_{bc} = \tfrac{1}{2} g^{am}(\nabla_c \delta g_{bm} + \nabla_b \delta
		g_{mc} - \nabla_m \delta g_{bc}).
	\end{equation*}
	so \( g^{ab}\delta R_{ab} \) splits into 6 terms. We will go through one in detail
	\begin{align*}
		& - \tfrac{1}{2}\int_D \lambda \theta_n g^{ab} g^{nk} \nabla_b \delta g_{ak} \sqrt{g}
		\ud^4 x \\ 
		& \quad = - \tfrac{1}{2}\int_D \lambda \theta_n \nabla_b(g^{ab} g^{nk} \delta g_{ak})
		\sqrt{g} \ud^4 x \\
		& \quad = \cancel{-\tfrac{1}{2}\int_D \nabla_b (\lambda \theta_n g^{ab} g^{nk} \delta g_{ak})
		\sqrt{g} \ud^4 x} + \tfrac{1}{2} \int_D \nabla_b(\lambda \theta_n)(g^{ab}g^{nk}\delta
		g_{ak}) \sqrt{g} \ud^4 x \\
		& \quad = -\int_D \lambda(\theta_b \theta_n + \nabla_b \theta_n) \delta g^{bn} \sqrt{g}
		\ud^4 x. 
	\end{align*}
\end{frame}

\begin{frame}
	\frametitle{Final equations}
	After some calculation, one finds \pause
	\begin{align*}
		\delta\tilde{S}[g_{ab}, z^\mu] & = \int_D(\partial_\mu \lambda - \lambda \theta_\mu)
		\delta z^\mu \ud^4 x \\
																	 & \quad + \int_D \lambda (R_{ab} - \tfrac{1}{2} Rg_{ab}
																	 - K_{ab} + Kg_{ab})\delta g^{ab} \sqrt{g} \ud^4 x
	\end{align*}
	with \( K_{ab} = \theta_a \theta_b + \nabla_{(a}\theta_{b)} \). 

	\pause The stationary configurations will be those for which this vanishes for any variation,
	so we arrive at \pause
	\begin{equation*}
		R_{ab} - \tfrac{1}{2} Rg_{ab} - K_{ab} + Kg_{ab} = 0.
	\end{equation*}
\end{frame}

\begin{frame}
	\frametitle{Final equations}
	After some calculation, one finds 
	\begin{align*}
		\delta\tilde{S}[g_{ab}, z^\mu] & = \int_D(\partial_\mu \lambda - \lambda \theta_\mu)
		\delta z^\mu \ud^4 x \\
																	 & \quad + \int_D \lambda (R_{ab} - \tfrac{1}{2} Rg_{ab}
																	 - K_{ab} + Kg_{ab})\delta g^{ab} \sqrt{g} \ud^4 x
	\end{align*}
	with \( K_{ab} = \theta_a \theta_b + \nabla_{(a}\theta_{b)} \). 

	The stationary configurations will be those for which this vanishes for any variation,
	so we arrive at 
	\begin{equation*}
		\highlight{red!50}{R_{ab} - \tfrac{1}{2} Rg_{ab}} - \highlight{blue!50}{K_{ab} + Kg_{ab}
		= 0}. 
	\end{equation*}
\end{frame}
	
\section{Significance of the equations}
\begin{frame}
	\frametitle{The dissipation tensor}
	We call the object \( K_{ab} \) the \emph{dissipation tensor}. \pause Coordinate free we can
	write
	\begin{equation*}
		K(X,Y) = (\theta \otimes \theta)(X,Y) + \tfrac{1}{2}\left[(\nabla_X \theta)Y + (\nabla_Y
		\theta)X\right]
	\end{equation*}
	which is a \( (0,2) \) tensor. \pause This means the equations we have derived are covariant.
\end{frame}

\begin{frame}
	\frametitle{Problems with existing equations}
	In \cite{Lazo2017}, the derived equations are
	\begin{equation*}
		R_{ab} + \tilde{K}_{ab} - \tfrac{1}{2}g_{ab}(R + \tilde{K}) = 0
	\end{equation*}
	where
	\begin{equation*}
		\tilde{K}_{ab} = \theta_m {\Gamma^m}_{ab} - \tfrac{1}{2}\left(\theta_a {\Gamma^m}_{mb} +
		\theta_b {\Gamma^m}_{am}\right). 
	\end{equation*}
	\pause \( \tilde{K}_{ab} \) is not a tensor so they are \emph{not} covariant!

	\pause From mechanics, linear action couplings in second order theories lead to terms
	\emph{quadratic} in the dissipation. \( K_{ab} \) is quadratic in	\( \theta_\mu \), but
	not \( \tilde{K}_{ab} \). 
\end{frame}

\section{Conclusions}
\begin{frame}
	\frametitle{Conclusions}
	As a recap \pause
	\begin{itemize}
		\item The Herglotz problem can be turned into an unconstrained optimisation problem.
			\pause
		\item Because we can write down an action, we can take its variation and deal with
			secon order theories. \pause
		\item We have derived the field equations for the Einstein-Hilbert Lagrangian with a
			linear action flux coupling. \pause 
		\item The result are a set of equations which are \emph{covariant} and with the
			expected structure. \pause
	\end{itemize}
\end{frame}

\begin{frame}
	\frametitle{Further work}
	\pause
	\begin{itemize}
		\item What do the solutions look like? \pause
		\item What happens when we add matter terms? \pause
		\item What kind of cosmology do we get? \pause
		\item What happens with different action flux couplings? \pause
	\end{itemize}
\end{frame}

\end{document}
