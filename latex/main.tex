\documentclass[12pt, oneside]{book}
% -------------------
% PACKAGES
% Basic font setup
\usepackage[utf8]{inputenc}
\usepackage[T1]{fontenc}
\usepackage{lmodern}

% Figures
\usepackage{graphicx}
\usepackage[font={footnotesize, sf}, labelfont=bf]{caption} 

% Maths tools
\usepackage{amsmath, amssymb, mathtools}
\usepackage{amsthm, thmtools}
\usepackage{tikz-cd}
\usepackage{siunitx}

\sisetup{detect-family = true, group-minimum-digits = 4}

% TOC Format
% \usepackage[titles]{tocloft}
% \renewcommand{\cftchappagefont}{\sffamily}   
% \renewcommand{\cftsecpagefont}{\sffamily}   
% \renewcommand{\cftsubsecpagefont}{\sffamily}   

% References
\usepackage{csquotes}

% Utilites
\usepackage{enumerate}

% -------------------
% CUSTOMISATION
% Geometry setup
\usepackage{geometry}
\usepackage{setspace}
\geometry{
	a4paper,
	right = 2.5cm,
	left = 2.5cm,
	bottom = 3cm,
	top = 3cm
}
\renewcommand{\baselinestretch}{1.3}

% -------------------
% Reference setup
\usepackage{hyperref}
\usepackage[english, capitalise, noabbrev]{cleveref}
\hypersetup{
	colorlinks,
	linkcolor = {red!50!blue},
	citecolor = {red!50!blue},
	urlcolor = {red!50!blue},
	linktoc = page
}

% -------------------
% BIBLIOGRAPHY
\usepackage[style=alphabetic, citestyle=alphabetic]{biblatex}
\usepackage[english]{babel}
\addbibresource{refs.bib}

% -------------------
% Theorem environments
\newcommand{\qedtriangle}{\ensuremath{\triangle}}
\newcommand{\qedtriangledown}{\ensuremath{\bigtriangledown}}
\allowdisplaybreaks

\declaretheoremstyle[spaceabove=6pt, spacebelow=6pt, headfont=\bfseries,
notefont=\normalfont, notebraces={(}{)}, qed=\qedtriangle]{definition}
\declaretheoremstyle[spaceabove=6pt, spacebelow=6pt, headfont=\bfseries,
notefont=\normalfont, notebraces={(}{)}, qed=\qedtriangledown]{example}



% -------------------
% LAYOUT
\usepackage[bf,sf,small,pagestyles]{titlesec}
\usepackage{titling}

% Pagestyle defintions

% Format of chapter titles
\titleformat{\chapter}[block]{\sffamily \bfseries \Huge}{\filleft \large Chapter \Huge \thechapter\\}{0pt}{\Huge \titlerule[1pt] \vspace{1ex} \filleft}

% -------------------
% SUBFILES
\usepackage{subfiles}

% CUSTOM COMMANDS FOR ALGEBRAIC TOPOLOGY
% ----------------------------------------------------------

% Restriction of a function
\newcommand{\rest}[1]{\raisebox{-.5ex}{$|$}_{#1}}

% Real numbers
\newcommand{\R}{\mathbb{R}}
\newcommand{\PR}{\mathbb{PR}}

% Rational numbers
\newcommand{\Q}{\mathbb{Q}}

% Complex numbers
\newcommand{\C}{\mathbb{C}}

% Natural numbers
\newcommand{\N}{\mathbb{N}}

% Integers
\newcommand{\Z}{\mathbb{Z}}

% Vector bold
\renewcommand{\vec}[1]{\mathbf{#1}}

% Span
\newcommand{\gen}[1]{\langle #1 \rangle}

% Set
\newcommand{\set}[1]{\{ #1 \}}

% Script A, B, M, P
\newcommand{\A}{\mathcal{A}}
\newcommand{\B}{\mathcal{B}}
\newcommand{\M}{\mathcal{M}}
\renewcommand{\P}{\mathcal{P}}
\renewcommand{\S}{\mathfrak{S}}

% Identity
\newcommand{\id}{\mathrm{id}}

% Kernel and image
\DeclareMathOperator{\im}{im}
\DeclareMathOperator{\coker}{coker}

% Absolute value
\newcommand{\abs}[1]{\left\lvert #1 \right\rvert}

% Norm
\newcommand{\norm}[1]{\left\lVert #1 \right\rVert}

\newcommand{\inn}[2]{\left\langle #1, #2 \right\rangle}

% Category of Vector Spaces
\newcommand{\Vect}{\mathsf{Vect}}
\newcommand{\VectK}{\Vect_{K}}
\newcommand{\VectR}{\Vect_{\R}}
\DeclareMathOperator{\Hom}{Hom}
\DeclareMathOperator{\Bil}{Bil}
\newcommand{\dual}{^{\vee}}
\DeclareMathOperator{\tr}{tr}

% Category of Manifolds
\DeclareMathOperator{\Diff}{Diff}

% Epi and monomorphisms
\newcommand{\onto}{\twoheadrightarrow}
\newcommand{\into}{\tailrightarrow}

\newcommand{\parbreak}{
	\begin{center}
		--- $\ast$ ---
	\end{center} 
}

% Defined as
\makeatletter
\newcommand*{\defeq}{\mathrel{\rlap{%
    \raisebox{0.3ex}{$\m@th\cdot$}}%
  \raisebox{-0.3ex}{$\m@th\cdot$}}%
	=
}
\makeatother

% Support
\DeclareMathOperator{\supp}{supp}

% Categories
\newcommand{\Top}{\mathsf{Top}}

\newcommand{\ud}{\, \mathrm{d}}
\renewcommand{\d}{\mathrm{d}}

\newcommand{\basisAt}[3]{\frac{\partial}{\partial #1^{#2}}\Big |_{#3}}
\newcommand{\basis}[2]{\frac{\partial}{\partial #1^{#2}}}
\newcommand{\at}[1]{\Big |_{#1}}

\newcommand{\phant}{\phantom{\alpha}}
\renewcommand{\L}{\mathcal{L}}

\newcommand{\deriv}[1]{\left.\tfrac{\ud}{\ud #1}\right\vert_{#1=0}}


\title{A variational derivation of the field equations of an action-dependent
	Einstein-Hilbert Lagrangian}
\author{Arnau Mas}
\date{June 2021}

\begin{document}
\begin{titlepage}
	\setstretch{2.0}
	\centering \sffamily

	\vspace*{2cm}

	\includegraphics[width = 8cm]{logo-uab}

	\vspace{2cm}

	{\Large \itshape Physics BSc. Undergraduate Thesis} \\
	{\large \itshape June 2021}

	\vspace{10pt}
	\hrule
	\vspace{10pt}
	{\bfseries \Large A variational derivation of the field equations of an action-dependent
	Einstein-Hilbert Lagrangian}

	\vspace{10pt}
	\hrule		
	\vspace{2cm}

	{\Large Arnau Mas Dorca}

	\vspace{1cm}
	{\large \itshape Supervised by}

	{\Large Dr Jordi Gaset}
\end{titlepage}

% \thispagestyle{empty}
\begin{otherlanguage}{catalan}
\subsubsection{Declaració d'autoria}
{ \sffamily Jo, Arnau Mas Dorca amb DNI 77633181Q, i estudiant del grau en Física i
	Matemàtiques de la Universitat Autònoma de Barcelona, en relació amb la memòria del
	treball de final de grau presentada per a la seva defensa i avaluació durant la
	convocatòria de juny del curs 2020-2021, declaro que
\begin{enumerate}
	\item El document presentat és original i ha estat realitzat per mi.
	\item El treball s'ha dut a terme principalment amb l'objectiu d'avaluar l'assignatura
		de treball de final del grau en Física de la Universitat Autònoma de Barcelona, i no
		s'ha presentat prèviament per a ser qualificat en l'avaluació de cap altre assignatura
		ni en aquesta ni cap altra universitat.
	\item En el cas de continguts de treballs publicats per terceres persones, l'autoria
		està clarament atribuïda, citant les fonts degudament.
	\item En els casos en els que el meu treball s'ha realitzat en co\l.laboració amb altres
		investigadors i/o estudiants es declara amb exactitud quines contribucions es deriven
		del treball de tercers i quines es deriven de la meva contribució. 
	\item A excepció dels punts esmentats anteriorment el treball és de la meva autoria. 
\end{enumerate}

\rightline{\small \itshape Barcelona, 15 de juny de 2021}
}

\clearpage
\thispagestyle{empty}
\subsubsection{Declaració d'extensió}
{ \sffamily Jo, Arnau Mas Dorca amb DNI 77633181Q, i estudiant del grau en Física i
	Matemàtiques de la Universitat Autònoma de Barcelona, en relació amb la memòria del
	treball de final de grau presentada per a la seva defensa i avaluació durant la
	convocatòria de Juny del curs 2020-2021, declaro que
\begin{enumerate}
	\item El nombre total de paraules incloses en les seccions des de la introducció fins a
		les conclusions és de \num{5819} paraules. 
	\item El nombre total de figures és 0.
\end{enumerate}
En total el document consta de
\begin{equation*}
	\num{5819}\text{ paraules} + \num{0} \times \num{200}\text{ paraules/figura} =
	\num{5819}\text{ paraules}
\end{equation*}
que és inferior al límit de \num{10000} paraules. 

\rightline{\small \itshape Barcelona, 15 de juny de 2021}
}
\end{otherlanguage}


\selectlanguage{english}
\pagestyle{plain}
\frontmatter
\hspace{0pt}
\vfill

\begin{itshape}
	I want to express deep gratitude to my supervisor Dr Jordi Gaset. This thesis would not
	have been possible without his generosity with his time and willingness to accomodate
	my interests. Having weekly discussion meetings has been a very rewarding experience and
	has made the process of research that much more engaging. 

	I also wish to mention Adrià Marín, who has also been working under Dr Gaset on a
	related topic and we have had many a productive discussion on our theses. 

	Of course I must also manifest my gratitude to my family for their support and
	understanding. This work is the capstone to five years of study, the last of which
	having taken place in the midst of a global pandemic, and I am sure they will be very
	proud of it. 
\end{itshape}

\vfill
\hspace{0pt}

\pagebreak
\section*{Abstract}
For the past century Einstein's theory of general relativity has constituted our most
advanced understanding of gravitation and all of its associated phenomena. Nevertheless,
there are still many puzzling and unanswered questions in the field of cosmology related
to the nature of the universe over large scales and its history. In recent years a number
of modified versions of general relativity have been developed and explored to attempt to
give satisfying answers to these questions. Among them is what is known as
action-dependent gravity. The mathematical framework for this theory is actually very deep
and not just related to relativity. Indeed, this is an example of what is known in
mathematics as contact geometry, the theory which describes the dynamics dissipative
systems, just as symplectic geometry underpins conservative Lagrangian and Hamiltonian
mechanics.

In this wor we apply recent developments in the field of contact geometry to a
successful derivation of the field equations of an action-dependent theory of gravity by
direct variation of a modified Einstein-Hilbert action. 

	{\small \sffamily \tableofcontents}

\mainmatter
\pagestyle{main}

\chapter{Background and motivation}
\subfile{chapters/chapter-1}

\chapter{The Herglotz variational problem}\label{ch:herglotz}
\subfile{chapters/chapter-2}

\chapter{Action-dependent Einstein gravity}\label{ch:einstein}
\subfile{chapters/chapter-3}

\chapter{Significance of the equations}\label{ch:significance}
\subfile{chapters/chapter-4}

\chapter{Conclusions}\label{ch:conclusions}
\subfile{chapters/chapter-5}

\backmatter

\pagestyle{plain}
\setquotestyle{english}
\printbibliography

\end{document}
