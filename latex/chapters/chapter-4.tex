\documentclass[../main.tex]{subfiles}

\begin{document}
Let us recap what we did in the previous chapter. We have shown, by computing the
variation of the corresponding action, that the field Equations of an Einstein-Hilbert
Lagrangian with linear dissipation, namely
\begin{equation} 
	L(g_{ab}, g_{ab,\mu}, g_{ab,\mu\nu}, z^\mu) = R(g_{ab}, g_{ab,\mu}, g_{ab,\mu\nu})
	\sqrt{g} - \theta_\mu z^\mu
\end{equation}
are
\begin{equation} 
	R_{ab} - \tfrac{1}{2}Rg_{ab} - K_{ab} + Kg_{ab} = 0
\end{equation}
where \( K_ab \) is the \( (0,2) \) symmetric tensor defined as
\begin{equation}
	K_{ab} = \nabla_{(a}\theta_{b)} + \theta_a\theta_b. 
\end{equation}

These equations are not the ones obtained in \cite{Lazo2017}. For the same Lagrangian, the
equations derived are
\begin{equation} \label{eq:EFE lazo}
	R_{ab} + \tilde{K}_{ab} - \tfrac{1}{2}g_{ab}(R + \tilde{K}) = 0
\end{equation}
where
\begin{equation} \label{eq:}
	\tilde{K}_{ab} = \theta_m {\Gamma^m}_{ab} - \tfrac{1}{2}\left(\theta_a {\Gamma^m}_{mb} +
	\theta_b {\Gamma^m}_{am}\right). 
\end{equation}
We will make to case as to why these equations are not the correct ones. The first issue
is that \( \tilde{K}_{ab} \) is not a tensor. To see this, we will compute \(
\tilde{K}_{ab} \) in two different coordinate systems and show that it does not transform
as a \( (0,2) \)-tensor would. 

The Christoffel symbols of the flat Minkowski vanish if we take cartesian coordinates.
Therefore, for any dissipation form we might consider, \( K_{ab} \) would vanish. Now, if
\( K_{ab} \) actually were a tensor then it would vanish in any other coordinate system.
But we can show this is not the case. If instead of cartesian coordinates we choose
spherical coordinates then the Christoffel symbols don't vanish. Specifically, the
non-vanishing ones are
\begin{align*}
	{\Gamma^r}_{\theta\theta} & = -r & {\Gamma^\theta}_{r\theta} & = \frac{1}{r} &
	{\Gamma^\phi}_{r\phi} & = \frac{1}{r} \\
	{\Gamma^r}_{\phi\phi} & = -r \sin{\theta}^2 & {\Gamma^\theta}_{\phi\phi} & =
	-\sin{\theta}\cos{\theta} & {\Gamma^\phi}_{\theta\phi} & = \frac{1}{\tan{\theta}}. 
\end{align*}
This means that, for example,
\begin{equation*}
	\tilde{K}_{tr} = 0 - \tfrac{1}{2}(\theta_t {\Gamma^m}_{mr} + 0) =
	-\frac{\theta_t}{2r}
\end{equation*}
which is certainly non-zero if \( \theta_t \) does not vanish. This shows that \(
\tilde{K}_{ab} \) is not a tensor, or rather \( \tilde{K}_{ab} \) do not represent the
components of a tensor, since if it were then if it vanishes in some coordinate system it
must do so in any other coordinate system, and we have just exhibited a coordinate system
in which it vanishes and another one in which at least one of its components does not. In
other words, the object derived in \cite{Lazo2017} is not coordinate independent so it
cannot possibly represent meaningful physics. 

On the contrary, the object \( K_{ab} \) that we found in \cref{eq:K} is indeed a tensor.
We could compute explicitly its transformation law and see that it is the one of a \(
(0,2) \) tensor. But a simpler way is writing it out in a coordinate-free manner.
Specifically, \( K_{ab} \) are the components of the object \( K = \theta \otimes \theta +
\nabla \theta \). Indeed, given two vector fields \( X \) and \( Y \)
\begin{align*} 
	K(X,Y) & = (\theta \otimes \theta)(X,Y) + (\nabla_X \theta)Y \\
				 & = \theta_a \theta_b X^a Y^b + X^a \nabla_a \theta_b Y^b \\
				 & = (\theta_a \theta_b + \nabla_{(a}\theta_{b)})X^a Y^b
\end{align*}
where in the last step we used that \( \nabla_{a}\theta_b = \nabla_{(a}\theta_{b)} \)
because \( \theta \) is closed. This shows \( K_{ab} \) are the components of a tensor
since \( \theta \otimes \theta \) and \( \nabla \theta \) are both \( (0,2) \) tensors
(they are bilinear). 

There is another reason that indicates that the equations in \cite{Lazo2017} are not the
correct ones. When we wrote down the Herglotz equations for the harmonic oscillator with
linear dissipation we obtained equations linear in the dissipation coefficient (see
\cref{eq:damped harmonic oscillator}). However, the Lagrangian for this system is first
order, whereas, as we had already discussed, the Einstein-Hilbert Lagrangian is actually
second order. The equations of motion for a second order Lagrangian with linear
dissipation, called the damped Pais-Uhlenbeck oscillator, are derived in \cite{Leon2021a},
and they are in fact not linear in the dissipation coefficient, but rather quadratic. In
our case, the dissipation form plays the role of the dissipation coefficient so by analogy
we would expect the equations to be quadratic in \( \theta \), not linear. And this is
indeed the case for the equations we derived, whereas the equations in \cite{Lazo2017} are
linear in \( \theta \).

So these are reasons for why the equations derived in \cite{Lazo2017} are not the right
ones, but we can actually point at why they derived them in the first place. One of the
simplifying assumptions they made was to take a simplified version of the Ricci curvature.
Specifically, the Ricci curvature consists of four terms. Two of them are contracions of
the Christoffel symbols with themselves, the other two are derivatives of the Christoffel
symbols. It can be shown that if one drops these last two when writing down the
Einstein-Hilbert action, the resulting equations remain unchanged. The justification is
that the terms with derivatives are a divergence, so they leave the action unchanged.
However, this justification fails for action dependent Lagrangians, which is ultimately
what dooms the whole computation. 



\end{document}
