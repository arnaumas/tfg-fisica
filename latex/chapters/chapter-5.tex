\documentclass[../main.tex]{subfiles}

\begin{document}
Let us now summarise the most relevant points of this thesis. 

\section{Further research}
In this thesis we have only explored the derivation of the field equations. Of course, if
we want this theory to produce relevant results we must be able to make testable
predictions. As we have discussed before, the main motivation at the moment to explore and
test modified theories of gravitation is their application to cosmology. Therefore, the
natural next step is to understand the cosmology that follows from them. One of the
driving assumptions of cosmology for the pasat decades has been the hypothesis of
homogeneity and isotropy of space, at least over large scales. There is now mounting
evidence to the contrary, however. The presence of the dissipation form in the equations
we have derived provides a way of breaking the homogeneity and isotropy in the models one
studies. What's more, these equations are not just the addition of some ad hoc terms to
the Einstein field Equations, but instead they come from a variational principle, which is
always preferable in physics. 

\end{document}
