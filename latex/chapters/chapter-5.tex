\documentclass[../main.tex]{subfiles}

\begin{document}
Let us now summarise the most relevant points of this thesis. 

\section{Main contributions}
These are the three principal ideas developed in this thesis
\begin{itemize}
	\item First, we have taken an existing method of deriving the Herglotz equations and
		used it to write down the action of a second-order action-dependent theory. 
	\item With this action we can then calculate its variation directly, which is a way of
		finding the field equations, even for a second order theory. \cref{ch:einstein} is
		dedicated to the detailed computation of this variation. 
	\item We also argue why the equations that appear in existing studies of the
		action-dependent Einstein-Hilbert cannot be correct and exhibit their main problems
\end{itemize}
These are of significance for two reasons. Firstly, these equations are the field
equations of an action-dependent field theory of second order. This is relevant because
the mathematical theory to handle this sort of theories is still in its early stages, so
having concrete examples will be helpful to point in the right direction for the more
general cases. On the other hand, from a more physical point of view, one wishes to
understand what kind of cosmology is implied by this theory and what kind of predictions
can be made from them. Having a solid derivation and understanding that this is the
correct form of the field equations is a necessary starting point. 

Additionally, by attacking this problem from a more geometrical perspective we are able to
clarify the nature of the elements that appear, specifically the role that the action flux
and dissipation form play and the kind of objects they are. 

\section{Further research}
The results of this work open many avenues for further research. For one, we have just
studied linear dissipation, but of course other kinds of dissipation terms could also be
investigated. Furthermore, we have not taken into acount any matter. This is represented
by the matter Lagrangian, which is added to the other terms. In the standard theory the
impact this has on the final equations is easy to see, but in the case of contact geometry
and action-dependent Lagrangians extra care has to be taken because the Herglotz equations
are non-linear. 

Finally there is the study of possible solutions to these equations. Are homogeneous and
isotropic solutions possible? How does the dissipation form influence any anisotropy or
inhomogeneity of the solutions? These are all questions that have yet to be addressed. 

\end{document}
