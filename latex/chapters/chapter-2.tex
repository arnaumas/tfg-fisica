\documentclass[../main.tex]{subfiles}

\begin{document}
We claim that a way of describing non-conservative systems with a variational principle is
by what we will call an action-dependent Lagrangian. In all of the standard examples from
classical mechanics, the Lagrangian is assumed to be a function of the positions and
velocities of the particle. An \emph{action-dependent Lagrangian} is a Lagrangian that is
also allowed to depend on the action of the path. However, recall that the action is
defined as the integral of the Lagrangian function along the path so this seems awfully
circular! Thus, great care has to be taken when formulating the theory of action-dependent
Lagrangians, in order to circumvent issues of circularity. The problem of finding the
stationary paths of the action functional corresponding to an action-dependent Lagrangian
is known as the Herglotz variational problem. In this chapter we will first discuss an
\emph{implicit approach} to this problem and some of its shortcomings. Then we will
frame the Herglotz problem as a constrained optimisation problem and show how this
different approach generalises to field theory. 

\section{The implicit formulation of the Herglotz problem}
Let's briefly describe what we will refer to as the \emph{implicit formulation of the
Herglotz problem}, as presented in \cite{Lazo2018}. We start by writing down a naive
equation for the action corresponding to an action dependent Lagrangian. Namely, given a
path \( q^\mu \colon [a,b] \to M \) we would write something like
\begin{equation*}
	S[q^\mu] = \int_a^b L(q^\mu(t), \dot{q}^\mu(t), S(t)) \ud t
\end{equation*}
where \( S(t) \) is the action of the path until time \( t \). Of course this makes no
sense since we are defining \( S \) on the left-hand side, and it appears on the
right-hand side! However, if, given a path then we define its action as a function of time
then we could write something like
\begin{equation}\label{eq:action integral equation}
	S[q^\mu](t) = \int_a^t L(q^\mu(s), \dot{q}^\mu(s), S[q^\mu](s)) \ud s
\end{equation}
and if we differentiate with respect to time we actually get an ODE for \( S[q^\mu] \)!
Indeed
\begin{equation} \label{eq:action ode}
	\dot{S}[q^\mu](t) = L(q^\mu(t), \dot{q}^\mu(t), S[q^\mu](t)).
\end{equation}
Notice that \cref{eq:action integral equation} actually forces the initial condition \(
S[q^\mu](a) = 0 \). We can even drop this requirement, since all we are interested in is
the difference of values of \( S \):
\begin{equation}\label{eq:action density 1d}
	S[q^\mu](b) - S[q^\mu](a) = \int_a^b L(q^\mu(t), \dot{q}^\mu(t), S[q^\mu](t)) \ud t. 
\end{equation}

What we have here is a functional which, for every path, is defined by an ODE. To find the
stationary paths of this functional we would, in principle, have to solve
\cref{eq:action ode} for any possible path and among all of them find which ones yield
extrema. This is the \emph{Herglotz variational problem}. This approach is in general not
feasible. However, just like the variational problem of classical Lagrangian mechanics can
be turned into a set of ODEs, the Euler-Lagrange equations, so can the Herglotz problem be
turned into a set of ODEs. These are known as the Herglotz equations, which can be written
down as
\begin{equation}\label{eq:herglotz equations mechanics}
	\frac{\partial L}{\partial q^\mu} - \frac{\d}{\d t}\frac{\partial L}{\partial \dot{q}^\mu}
	+\frac{\partial L}{\partial \dot{q}^\mu} \frac{\partial L}{\partial z} = 0
\end{equation}
where the Lagrangian is \( L(q^\mu, \dot{q}^\mu, z) \), \( z \) being the action
dependence. These equations differ from the Euler-Lagrange equations only by one term. And
in fact, if \( L \) is action-independent, thus \( \frac{\partial L}{\partial z} = 0 \),
we recover exactly the Euler-Lagrange equations. 

See \cite{Lazo2018} \S 3.2 of \cite{Leon2021} or for a derivation of the Herglotz
equations following the implicit approach. 

\subsection{Example: the damped harmonic oscillator}
Before we move forward, let's see what kind of equations of motion we get from the
Herglotz equations. The simplest action-dependence we can introduce ---other than no
action-dependence at all, of course--- is a term linear in \( z \). Let's see what happens
when we add this term to the simple harmonic oscillator. Explicitly, consider the
Lagrangian
\begin{equation*}
	L(q, \dot{q}, z) = \tfrac{1}{2}m\dot{q}^2 - \tfrac{1}{2} m\omega^2 q^2 - \gamma z. 
\end{equation*}
A straightforward computation shows that for this Lagrangian, \cref{eq:herglotz equations
mechanics} becomes
\begin{equation*}
	-m\omega^2 q - m\ddot{q} - \gamma m \dot{q} = 0
\end{equation*}
which after some rearranging can be turned into
\begin{equation*}
	\ddot{q} + \gamma\dot{q} + \omega^2 q = 0,
\end{equation*}
which one recognises as the equation of motion of the damped harmonic oscillator. The
Herglotz principle delivers in its promise: we have just derived the equations of motion
of a fundamentally non-conservative system from a variational problem!

\section{The Herglotz problem as a constrained optimisation problem}
We have derived the Herglotz equations, so it would seem we have already solved the theory
of action-dependent Lagrangians. However, general relativity is a field theory, so if we
wish to understand an action-dependent variant of it we have to know the form the Herglotz
equations take for field theory. If we try to apply the implicit method to field theory we
run into a number of problems. For one, it is just not very elegant. The action functional
is defined implicitly through an ordinary differential equation, one for every path.
What's more, this equation is no longer an ODE in field theory, but rather it becomes a
PDE. PDEs are notoriously much more difficult to solve thant ODEs, so we would like a way
to circumvent this issue. 

We can achieve this if we frame the Herglotz problem as a constrained optimisation
problem. Not only that, we also get a much clearer and more elegant formulation. We
describe what this looks like for mehcanics. First, instead of considering paths in
spacetime, \( q^\mu \colon [a,b] \to M \), we enlarge the configuration space with one
extra quantity, which we will call \( z \). At this point \( z \) simply tracks a quantity
that changes along the path, but we will later require that \( z \) actually match the
action of the path at each time. This will be the constraint. 

So, we have paths of the form \( (q^\mu, z) \colon [a,b] \to M \times \R \). We define a
functional on these paths as
\begin{equation}\label{eq:action functional unconstrained}
	S[q^\mu, z] = z(a) - z(b) = \int_a^b \dot{z}(t) \ud t,
\end{equation}
so \( S \) is just the change in \( z \) along the trajectory. This functional as it
stands has no stationary paths, since we can find trajectories with arbitrarily large
changes in \( z \), both positive and negative. So we constrain the possible paths.
Namely, we require that \( z \) actually represent the action. So we try to find the paths
that extremise \( S \) only among those that satisfy the constraint
\begin{equation}\label{eq:constraint mechanics}
	\dot{z}(t) = L(q^\mu(t), \dot{q}^\mu(t), z(t)),
\end{equation}
where \( L \) is the action-dependent Lagrangian that describes the system.  Notice that
this is very similar to \cref{eq:action ode}. 

Say\( (q^\mu, z) \) is a trajectory that satisfies \cref{eq:constraint mechanics}. Then 
\begin{equation*}
	S[q^\mu, z] = z(a) - z(b) = \int_a^b \dot{z}(t) \ud t = \int_a^b L(q^\mu(t),
	\dot{q}^\mu(t), z(t)) \ud t. 
\end{equation*}
So, for paths that satisfy the constraint, the functional \( S \) is indeed the
action functional, understood as the integral of the Lagrangian along the path.

This is all well and good, but how does one actually go about solving a constrained
optimisation problem? It turns out, we can use an infinite dimensional analog of Lagrange
multipliers to turn this into a regular optimisation problem to which we can apply the
tools of the calculus of variations. 

Recall, given some function \( f \colon \R^n \to \R \), \( x \in \R^n \) is an extremum of
\( f \) subject to \( m \) constraints \( g_k \colon \R^n \to \R \) ---i.e. \( g_k(x) = 0
\) --- if and only if there exist numbers \( \lambda_k \in \R \) such that
\( x \) is an extremum of the function
\begin{equation} \label{eq:lagrange multipliers}
	 F = f - \sum_{k = 1}^{m}\lambda_k g_k 
\end{equation}
\emph{without any constraints}. The numbers \( \lambda_k \) are called the Lagrange
multipliers. It can be shown that this result generalises to infinite dimensional spaces
and infinitely many constraints. In our case, the function we are trying to find the
extrema of is the functional \( S \). Our constraints are parameterised by \( t \in [a,b]
\):
\begin{equation*}
	g_t[q^\mu, z] = \dot{z}(t) - L(q^\mu(t), \dot{q}^\mu(t), z(t)).
\end{equation*}
Thus, replacing sums with integrals in \cref{eq:lagrange multipliers}, the extrema of
\cref{eq:action functional unconstrained} subject to \cref{eq:constraint mechanics} will
be those that extremise the following functional: 
\begin{align} \label{eq:lagrange multiplier action}
	\tilde{S}[q^\mu, z] & = S[q^\mu, z] - \int_a^b \lambda_t g_t[q^\mu, z] \ud t \nonumber \\
											& = \int_a^b \dot{z}(t) \ud t - \int_a^b \lambda_t\big[\dot{z}(t) -
											L(q^\mu(t), \dot{q}^\mu(t), z(t))\big] \ud t \nonumber \\
											& = \int_a^b (1 - \lambda_t) \dot{z}(t) + \lambda_t\big(L(q^\mu(t),
											\dot{q}^\mu(t), z(t))\big) \ud t.
\end{align}
A couple of observations. First, we are thinking of the Lagrange multipliers as real
numbers parameterised by \( t \), but we could equivalently think of them as a function of
\( t \) and write \( \lambda(t) \) instead of \( \lambda_t \). We will do this. Secondly,
and more importantly, if we write the integrand of \cref{eq:lagrange multiplier action} as
\begin{equation*}
	\tilde{L}(q^\mu(t), \dot{q}^\mu(t), z(t), \dot{z}(t)) = \big(1 - \lambda(t)\big) \dot{z}(t) +
	\lambda(t)\big(L(q^\mu(t), \dot{q}^\mu(t), z(t))\big)
\end{equation*}
one should be able to recognise \( \tilde{S} \) as something that looks just like a
regular old action functional defined by the integral of a regular old Lagrangian, except
it is now defined on expanded trajectories \( (q^\mu, z) \). So we should be able to use
the Euler-Lagrange equations to write down the equations of motion of its extremal paths!
The equation for \( z \) reads
\begin{equation*}
	0 = \frac{\partial \tilde{L}}{\partial z} - \frac{\d}{\d t} \frac{\partial
	\tilde{L}}{\partial \dot{z}} = \lambda \frac{\partial L}{\partial z} + \dot{\lambda}
\end{equation*}
or equivalently
\begin{equation} \label{eq:euler-lagrange multiplier}
	\dot{\lambda} = -\lambda \frac{\partial L}{\partial z}. 
\end{equation}
The Euler-Lagrange equations for the positions then are
\begin{equation*}
	0 = \frac{\partial \tilde{L}}{\partial q^\mu} - \frac{\d}{\d t} \frac{\partial
	\tilde{L}}{\partial \dot{q}^\mu} = \lambda \frac{\partial L}{\partial q^\mu} -
	\dot{\lambda}\frac{\partial L}{\partial \dot{q}^\mu} - \lambda \frac{\d}{\d
	t}\frac{\partial L}{\partial \dot{q}^\mu}
\end{equation*}
and after substituting in \cref{eq:euler-lagrange multiplier} and dividing through by \(
\lambda \) we find
\begin{equation} \label{eq:euler-lagrange positions}
	0 = \frac{\partial L}{\partial q^\mu} - \frac{\d}{\d t} \frac{\partial L}{\partial
	\dot{q}^\mu} + \frac{\partial L}{\partial z}\frac{\partial L}{\partial \dot{q}^\mu}
\end{equation}
which are exactly the Herglotz equations. 

\section{Action-dependent field theory}
We have seen how to derive the Herglotz equations in a more elegant way using Lagrange
multipliers. More importantly, we have written down a modified Lagrangian which gives rise
to them through standard calculus of variations techniques. This will be very useful as,
in some cases (general relativity being one of them), it is easier to derive the equations
of motion of a system by direct variation of the action, rather than writing down the
Euler-Lagrange equations. We will be able to do exactly this once we have the field
theoretic version of \cref{eq:lagrange multiplier action} in our hands.

\subsection{Classical field theory and Lagrangian densities}\label{sec:lagrangian
densities}
First off, we need to set the stage for Lagrangian field theory. The parameter space is no
longer just time, but rather all of spacetime, \( M \). Fields are the assignment of some
value to each point in spacetime, so we could have scalar fields, vector fields, or, as is
the case in general relativity, tensor fields. Let us first fix some notation. We will
denote by \( \phi \) some field configuration. If \( \phi \) is a scalar field then it
carries no indices. If \( \phi \) is a vector field then it carries one upper index, if it is a
tensor field then it carries multiple indices. The metric carries two lower indices since
it is a \( (0,2) \) tensor field. In almost all that follows we will assume \( \phi \) is
a vector field, but the results we find transfer to tensors of other rank. Of
course \( \phi \) depends on spacetime, which we will sometimes write explicitly as \(
\phi^a(x^\mu) \). As a convention, we will use latin indices for the indices of the field,
and reserve greek indices for spacetime coordinates. 

The Lagrangian of a field theory is a function of the field and of its derivatives.
It is not, however, a function of real values, but rather an \( n \)-dimensional ---where
\( n \) is the dimension of spacetime---, differential form, also known as a top form.
This is because, the action is defined as the integral of the Lagrangian over some region
of spacetime, and the objects we can integrate over spacetime are precisely the top forms.

It turns out, however, that any two top forms differ only by an overall factor, i.e., given two
top forms \( \omega_1 \) and \( \omega_2 \), there exists a scalar function of spacetime
\( M \) such that \( \omega_1 = f\omega_2 \). This means, that if we think of the
Lagrangian as some top form \( \L \), once we pick a distinguished top form there is a
unique scalar function \( L \) such that \( \L = L \omega \). This distinguished top form
will in most cases be the top form induced by the coordinates we are working in, which we
will write as \( \d^nx \). Sometimes we will use Lagrangian density to refer to \( \L \)
and Lagrangian to refer to \( L \). 

The setup in classical field theory is as follows. Given some Lagrangian, which encodes
the system we are studying, we define the action functional on all the possible field
configurations as 
\begin{equation*}
	S[\phi^a] = \int_D \L(\phi^a(x^\mu), \partial_\mu \phi^a(x^\mu)) = \int_D
	L(\phi^a(x^\mu), \partial_\mu \phi^a(x^\mu)) \ud^n x,
\end{equation*}
where \( D \) is some region of spacetime where this integral makes sense. 

Using the calculus of variations one can show that the stationary configurations of this
action functional satisfy the Euler-Lagrange equations of field theory
\begin{equation*}
	\frac{\partial L}{\partial \phi^a} - \partial_\mu \frac{\partial L}{\partial
	(\partial_\mu \phi ^a)} = 0
\end{equation*}
where a summation convention is implied over \( \mu \). 

\subsection{The action flux}
How do we generalise this to an action dependent Lagrangian? The most naive approach would
be to try to replicate the Herglotz equations from mechanics wholesale and write down
\begin{equation*}
	\frac{\partial L}{\partial \phi^a} - \partial_\mu \frac{\partial L}{\partial
	(\partial_\mu \phi ^a)} + \frac{\partial L}{\partial(\partial_\mu \phi^a)}\frac{\partial
L}{\partial z} = 0. 
\end{equation*}
However this will not work because the last term has a pesky free \( \mu \) index. This
seems to suggest that we need to modify the nature of \( z \). We had claimed before that
\( z \) represented the action along the path, but if we look at \cref{eq:action
functional unconstrained} we see this is not quite right. In mechanics the analog of \( D
\) is \( [a,b] \). The difference \( z(a) - z(b) \) can also be written as \(
\int_{\partial [a,b]} z \), since the boundary of \( [a,b] \) is just \( a \) and \( b \).
This seems to indicate that the correct analog of \cref{eq:action functional
unconstrained} for field theory should be
\begin{equation*}
	S[\phi^a, z] = \int_{\partial D} z.
\end{equation*}
What kind of obejct should \( z \) be then? \( \partial D \) has dimension \( n-1 \), so
\( z \) has to be something we can integrate over an \( (n-1) \)-dimensional manifold,
i.e. a differential \( (n-1) \)-form. As it turns out, \( (n-1) \)-forms behave almost
like vector fields. Specifically, by expanding in the basis \( \ud x_\mu \) we have \( z =
z^\mu \ud x_\mu \). Thus, what \( z \) represents is the action flux. We encode this in a
constraint perfectly analogous to \cref{eq:constraint mechanics},
\begin{equation} \label{eq:constraint field theory}
	\ud z(x^\mu) = \L(\phi^a(x^\mu), \partial_\mu \phi^a(x^\mu), z^\mu(x^\mu)).
\end{equation}
\( \ud z \) is the exterior derivative of \( z \), which is a top form, so this equality
makes sense. In coordinates it is easy to show that \( \ud z = \partial_\nu z^\nu \ud^n x
\), so that \cref{eq:constraint field theory} reads in coordinates as
\begin{equation} \label{eq:constraint field theory coordinates}
	\partial_\nu z^\nu (x^\mu) = L(\phi^a(x^\mu), \partial_\mu \phi^a(x^\mu)).
\end{equation}
This expresses the fact that \( \ud z \) has to be the action density. Indeed, for field
configurations that satisfy this constraint then one has, applying Stokes' theorem
\begin{equation*}
	S[\phi^a, z^\mu] = \int_{\partial D} z = \int_D \ud z = \int_D \L(\phi^a, \partial_\mu
	\phi^a, z^\mu). 
\end{equation*}
So we have arrived at the right formulation of the Herglotz variational problem for field
theory.

\subsection{Constrained optimisation in field theory}
So just like before, we will turn this constrained optimisation problem into an
unconstrained one using Lagrange multipliers. The expanded action, in analogy with
\cref{eq:lagrange multiplier action} will be
\begin{equation}\label{eq:expanded action field theory}
	\tilde{S}[\phi^a, z^\mu] = \int_D (1 - \lambda) \ud z + \lambda \L(\phi^a, \partial_\mu
	\phi^a, z^\nu) = \int_D \ud^n x \big[ (1 - \lambda) \partial_\mu z^\mu + \lambda L(\phi^a,
	\partial_\mu \phi^a, z^\nu) \big]
\end{equation}
Note that the Lagrange multiplier \( \lambda \) is a function of spacetime. Let us write
down the integrand of \cref{eq:expanded action field theory} as an expanded Lagrangian:
\begin{equation} \label{eq:expanded lagrangian field theory}
	\tilde{\L}(\phi^a, \partial_\mu \phi^a, z^\nu, \partial_\mu z^\nu) = \tilde{L}(\phi^a,
	\partial_\mu \phi^a, z^\nu, \partial_\mu z^\nu)\ud^n x = \big[ (1 - \lambda)
	\partial_\mu z^\mu + \lambda L(\phi^a, \partial_\mu \phi^a, z^\nu) \big] \ud^n x.
\end{equation}

\subsection{The Herglotz equations for field theory}
Finally, we can derive the Herglotz equations for field theory from the expanded
Lagrangian we have just obtained in \cref{eq:expanded lagrangian field theory}. The
equations for the action flux are
\begin{equation*}
	0 = \frac{\partial \tilde{L}}{\partial z^\nu} - \partial_\mu \frac{\partial
	\tilde{L}}{\partial(\partial_\mu z^\nu)} = \lambda \frac{\partial L}{\partial z^\nu} +
	\partial_\mu(\lambda \delta_\nu^\mu) = \lambda \frac{\partial L}{\partial z^\nu} +
	\partial_\nu \lambda.
\end{equation*}
So, rearranging,
\begin{equation} \label{eq:euler lagrange multiplier field theory}
	\partial_\nu \lambda = - \lambda \frac{\partial L}{\partial z^\nu}. 
\end{equation}

And for the values of the field
\begin{equation*}
	0 = \frac{\partial \tilde{L}}{\partial \phi^a} - \partial_\mu \frac{\partial
	\tilde{L}}{\partial(\partial_\mu \phi^a)} = \lambda \frac{\partial L}{\partial \phi^a} -
	(\partial_\mu \lambda) \frac{\partial L}{\partial (\partial_\mu \phi^a) } - \lambda
	\partial_\mu \frac{\partial L}{\partial (\partial_\mu \phi^a) }
\end{equation*}
and, when plugging in \cref{eq:euler lagrange multiplier field theory} and dividing
through by \( \lambda \) we arrive at the field theoretical Herglotz equations
\begin{equation} \label{eq:herglotz field theory}
	\frac{\partial L}{\partial \phi^a} - \partial_\mu \frac{\partial
	L}{\partial(\partial_\mu \phi^a)} + \frac{\partial L}{\partial z^\mu} \frac{\partial
L}{\partial(\partial_\mu \phi^a)} = 0. 
\end{equation}


% \section{Action dependent Lagrangian}
% The standard formulations of classical mechanics, the Lagrangian and Hamiltonian
% formalisms, can be used to describe and study a great variety of physical systems.
% Nevertheless, all of them exhibit \emph{conservative} behaviour. Even the simplest non
% conservative systems, i.e. dissipative, cannot be described by means of a standard Lagrangian or
% Hamiltonian. One way of describing non conservative systems by means of a variational
% principle is through \emph{action-dependent lagrangians}. This theory was originally
% studied by Herglotz.

% Recall that in ordinary classical mechanics, one models particle trajectories as
% smooth paths in some configuration manifold, \( M \), which generally represents space in
% Newtonian mechanics and spacetime in relativity, i.e. trajectories are of the
% form \( q \colon [t_0, t_1] \to M \). The Lagrangian \( L \) is introduced as a function on the
% velocity phase space, \( TM \), i.e. \( L \colon TM \to \R \), with which the action
% functional is defined as the integral
% \begin{equation*}
% 	S[q] = \int_{t_0}^{t_1} L(q^\mu(t) , \dot{q}^\mu(t)) \ud t. 
% \end{equation*}
% The Principle of Least Action is the statement that physical trajectories correspond to
% extrema of the action functional. The condition of being an extremum of the action can be
% equivalently formulated, by means of the calculus of variations, in terms of the
% well-known Euler-Lagrange equations. 

% In this new framework we expand the configuration space with a new variable, and consider
% paths \( (q,z) \colon [t_0, t_1] \to M \times \R \). The new action functional is 
% \begin{equation}\label{eq:contact action 1}
% 	S[q,z] = z(t_1) - z(t_0).
% \end{equation}
% Physical trajectories will be those that extremise \( S \) \emph{subject to the
% constraint} \( \dot{z}(t) = L(q^\mu(t), \dot{q(t)}, z(t)) \). We see that if a path
% satisfies this constraint then \cref{eq:contact action 1} becomes
% \begin{equation}\label{eq:contact action 2}
% 	S[q,z] = z(t_1) - z(t_0) = \int_{t_0}^{t_1} \dot{z}(t) = \int_{t_0}^{t_1} L(q^\mu(t),
% 	\dot{q}^\mu(t), z(t))
% \end{equation}
% which makes clear why this is an action dependent Lagrangian: \( z \) represents the
% change in action along the path. And if \( L \) is independent of \( z \) then
% \cref{eq:contact action 2} just reduces to the usual action from Lagrangian mechanics. 

% Now, it is not obvious how to solve this variational problem, since it is now a
% constrained optimisation problem. However, using Lagrange multipliers, it can be shown
% that extrema of this action will satisfy the so-called Herglotz equations:
% \begin{equation}\label{eq:herglotz}
%  	\frac{\d}{\d t}\frac{\partial L}{\partial \dot{q}^\mu} - \frac{\partial L}{\partial
%  	q^\mu} - \frac{\partial L}{\partial z} \frac{\partial L}{\partial \dot{q}^\mu} = 0
% \end{equation}
% These differential equations are the action dependent analog to the Euler-Lagrange
% equations, and notice that, as expected, if \( L \) is independent of \( z \) then the
% Herglotz equations reduce to Euler-Lagrange. 

% To see that this is useful to describe physical phenomena, consider the following action
% dependent Lagrangian:
% \begin{equation*}
% 	L(q, \dot{q}, z) = \frac{1}{2}m\dot{q}^2 - \frac{1}{2}m\omega^2 q^2 - \gamma z. 
% \end{equation*}
% The Herglotz equation for this Lagrangian is
% \begin{equation*}
% 	m\ddot{q} + m\omega^2 q + \gamma m \dot{q} = 0  
% \end{equation*}
% which one recognises as the equation of motion of a damped harmonic oscillator.

% The main takeaway is that this is a way of describing non conservative systems by means of
% a variational principle. 

% \section{Action dependent field theory}
% Our aim is to study what general relativity looks like when formulated in this new
% framework. However, general relativity is a field theory, so we need to introduce more
% complex mathematical machinery to understand the generalisation of contact lagrangians to
% field theory. 

% The modern way of formulating field theory is in the language of bundles and
% sections. 

% \section{The action density}
% We have established that the natural space in which the values of a field and its
% derivative is the jet bundle. Therefore, in analogy with mechanics, one would naturally
% introduce the Lagrangian as a functional over this space, \( L \colon J^1 \pi \to \R \).
% However this is not quite right. Recall, the action functional is defined as an integral
% of the lagrangian over spacetime. thus, we want whatever kind of object the Lagrangian is
% to be something we can integrate. This means that \( n \)-forms over \( M \) must somehow
% come into play. The right form of the Lagrangian is actually a function of the form \( \L
% \colon J^1\pi \to \ext^n T^\ast M \). If this is the case, then given a certain field
% configuration \( \phi \in \Gamma(\pi) \), \( \L \circ j^1 \phi \) is actually an element
% of \( \Omega^n(M) \), an \( n \)-form. With this in hand, the action functional for fields
% is defined as 
% \begin{equation*}
% 	S[\phi] = \int_{M} \L \circ j^1\phi. 
% \end{equation*}

% Now we follow the same path we did before to introduce an action dependence in the
% Lagrangian. We must somehow enlarge the domain of the Lagrangian to encode the action
% dependence. This role was played by the \( z \) before. By defintion, the action was given
% by the difference \( z(t_1) - z(t_0) \). This can be interpreted as the integral of \( z
% \) over the boundary of the region of parameter space. So, roughly speaking, we need
% something we can integrate over \( \partial M \). Because the dimension of \( \partial M
% \) is one less than \( M \), we need an \( n-1 \)-form. The expanded domain for the
% Lagrangian is going to be \( J^1 \pi \otimes_M \ext^{n-1} T^\ast M \). A section of this
% bundle is therefore a field configuration \( \phi \) as well as an \( n-1 \)-form \( z \).
% The action functional is defined as
% \begin{equation*}
% 	S[\phi, z] = \int_{\partial M} z.
% \end{equation*}
% We wish to find extrema of this functional subject to the condition that \( z \) be
% actually related to the Lagrangian. This is captured by the requirement \( \ud z = \L \).



% \section{Bundles, sections and jets}
% Field theory studies the dynamics of fields, which are, in the broadest sense, quantities
% that vary from point to point in space (or spacetime). The fundamental example is a scalar
% field, which is the assignment of a real value to each point in space (or spacetime). In
% more precise mathematical terms, if \( M \) denotes spacetime (thought of as a manifold with a
% pseudo-Riemannian metric, as is standard in general relativity), then a scalar field is a
% smooth map \( \phi \colon M \to \R \). However, if we think of a field with vector values,
% for instance, it is not enough to define it as a map \( X \colon M \to \R^4 \), at
% least not if we want our theory to be covariant. Indeed, we think of a vector field as an
% assignment of a vector at each point in spacetime which is tangent to that point. Thus,
% the proper definition of a vector field is as a map \( X \colon M \to TM \), where \( TM
% \) is the tangent bundle of \( M \), together with the condition \( X(p) \in T_pM \),
% which is just the requirement that \( X(p) \) be tangent to \( p \). This condition makes
% \( X \) a \emph{section} of \( TM \). If \( \tau_M \colon TM \to M \) is the map which
% takes a tangent vector to its basepoint, then \( X \) being a section of \( TM \) is
% equivalent to \( \tau_{TM} \circ X = \id_M \). 

% This idea points us in the right direction for a more appropriate definition of fields. We
% introduce the idea of a \emph{fiber bundle}, which is a projection map \( E
% \xrightarrow{\pi} M \), such that for every \( p \in M \), \( \pi^{-1}(p) \) is
% diffeomorphic to some fixed manifold \( F \). \( \pi^{-1}(p) \) is called the fibre over
% \( p \), and \( F \) is the typical fibre of the bundle. What is happening here is that \(
% E \) is the result of attaching a copy of \( F \) at each point of \( M \). If one
% attempts to draw this, taking \( M  \) to be 2 dimensional and \( F \) 1 dimensional, the
% result is something that looks much like a hair brush, where the copies of \( F \) are the
% bristles, or fibres, hence the terminology. \( M \) is known as the \emph{base space}, \( E \)
% receives the name \emph{total space}. 

% For a more concrete picture, let's see what this looks like in coordinates. Consider a
% coordinate system \( x^\mu \) for \( M \), where \( \mu \) runs over the dimension of
% \( M \) ---which will be 4 if \( M \) is spacetime---, and a coordinate system \( u^\alpha
% \) for the typical fibre \( F \), where \( \alpha \) runs over the dimension of \( F \).
% Then points in the total space \( E \) can be labeled by \( (x^\mu, u^\alpha) \). For
% example, in the tangent bundle, these coordinates label the base point of a vector and its
% components with respect to a basis (which is generally taken to be basis induced by the
% coordinates in physics). 

% \subsection{Fields as sections of bundles}




% In classical mechanics, the Lagrangian is defined as a function on the velocity phase
% space of the system, i.e. the tangent bundle of the configuration space \( TM \), \( L
% \colon TM \to \R \). With the Lagrangian in hand, one defines the action functional as
% \begin{equation*}
% 	S[\gamma] = \int_a^b L(\gamma(t), \dot{\gamma}(t)) \ud t. 
% \end{equation*}

% The precise generalisation of the Lagrangian to classical field theory is a bit more
% difficult. For one, whatever the Lagrangian is, it has to be something we can integrate
% over our parameter space, which in general will be a manifold. In the case of general
% relativity, the parameter space is space-time, for example. This  means it has to be
% related to a top form on the parameter space. The precise language to formulate this idea
% is the language of jet bundles. In general, a suitable way to think of fields is as
% sections of some bundle \( E \xrightarrow{\pi} M \), where \( M \) is the parameter space
% and the typical fibre of the bundle is the space in which the field takes values. Then,for
% a first order theory, the Lagrangian \( \L \) is a bundle morphism from the first jet bundle \(
% J^1\pi \) to the bundle \( \ext^n T^\ast M \). Then, for a section \( \phi \) of \( \pi
% \), \( \L \circ j^1 \phi \) is a section of \( \ext^n T^\ast M \), i.e., a top form over
% \( M \). We can then define the action functional as
% \begin{equation*}
% 	S[\phi] = \int_M \L \circ j^1 \phi
% \end{equation*}



% In field theory, the Lagrangian is no longer just a real-valued function, since it is
% something we have to integrate over the parameter space, which is spacetime in the context
% of general relativity. The object we need is going to be a top form. Specifically, if we
% are studying the dynamics of sections of some bundle \( E \xrightarrow{\pi} M \), the
% Lagrangian density \( \L \), or sometimes just the Lagrangian, is a bundle morphism \( \L
% \colon J^1\pi \to \ext^n T^\ast M \), where \( n \) is the dimension of the parameter
% space \( M \). That \( \L \) is a bundle morphism means that for any \( j_p^1 \phi \in
% J^1_p \pi \), \( \L(j_p^1 \phi) \in \wedge^n T^\ast_p M \), or, more succintly, that \(
% \wedge^n \tau_M^\ast \circ \L = \pi_1 \). 

% When we introduce contact geometry into the picture one has to enlarge the source space
% for the Lagrangian, it is now a bundle morphism of the form
% \begin{equation*}
% 	\L \colon J^1\pi \oplus_M \ext^{n-1}T^\ast M \to \ext^{n} T^\ast M. 
% \end{equation*}
% The domain is the Whitney sum of the first jet bundle, where the field and its first
% derivatives live, and of the bundle of \( n-1 \) forms over \( M \), where the action
% density lives. In coordinates, the points in \( J^1\pi \) can be described as \( (x^\mu,
% u^\alpha, u^\alpha_\mu) \). The points of \( \ext^{n-1}T^\ast M \) can be described in
% coordinates by \( z^\mu \). Indeed, \( \ext^{n-1} T^\ast M \) and \( T^\ast M \) both are
% vector bundles of dimension \( n \) over \( M \), so they are isomorphic. Indeed, writing
% \( \d^n x \) for \( \d^1x \wedge \cdots \wedge \d^n x \), we define
% \begin{equation*}
% 	\ast \ud x_\mu = (-1)^\mu \frac{\partial}{\partial x^\mu} \iprod \d^n x
% \end{equation*}
% where no summation is implied over \( \mu \). The \( \ast \ud x_\mu \) are a basis of the
% \( n-1 \) forms. Then we can write for any \( z \in \ext^{n-1} T^\ast M \), \( z = z^\mu
% \ast \d x_\mu \). 

% For general relativity, we are interested in sections of the bundle \( \Sym^2(T^\ast M) \)
% of symmetric 2-forms over \( M \), specifically the subbundle of nondegenerate symmetric
% 2-forms, also known as pseudo-Riemannian metrics. The 

% \section{The Herglotz equations as a constrained version of the Euler-Lagrange equations}
% We know how to find extremal paths of an ordinary action functional, but how do we go
% about doing so in the presence of a constraint? The standard tool to solve constrained
% optimisation problems is the Lagrange multiplier theorem. We know how to do this for
% functions defined on a finite dimensional space. Namely, if \( f \colon \R^n \to \R \),
% then \( x \in \R^n \) is an extremum of \( f \) subject to \( m \) constraints \( g_k(x) = 0
% \) if and only if there exist \( \lambda_k \in \R \) such that \( x \) is an extremum of
% \( f - \sum_{k = 0}^{m} \lambda_k g_k \). Thus, we are able to transform a constrained
% optimisation problem into an unconstrained one. 

% In our case, the functional we are trying to optimise is the action functional, which is
% defined on an infinite dimensional space, namely the space of all possible paths. The
% constraint we are imposing can be written as
% \begin{equation}
% 	\dot{z}(t) = L(\gamma(t), \dot{\gamma}(t), z(t))
% \end{equation}
% for all \( t \in [0,1] \). Another way to think about it is that we have infinitely many
% constraints, parameterised by \( t \). It can be shown that, in analogy with the finite
% dimensional case, extrema of the actiona functional \( S \) subject to this family of
% constraints are extrema of the modified functional
% \begin{equation}\label{eq:unconstrained action}
% 	S_\theta[\gamma, z] = S[\gamma, z] - \int_0^1 \theta(t) \big[\dot{z}(t) - L(\gamma(t),
% 	\dot{\gamma}(t), z(t))\big] \ud t
% \end{equation}
% which, when substituting the action, becomes
% \begin{equation}\label{eq:unconstrained action}
% 	S_\theta[\gamma, z] =  \int_0^1 \dot{z}(t) - \theta(t) \big[\dot{z}(t) - L(\gamma(t),
% 	\dot{\gamma}(t), z(t))\big] \ud t
% \end{equation}

% The key point here is that this is an unconstrained optimisation problem, so we can
% directly apply the Euler-Lagrange equations. Doing so yields exactly the Herglotz
% equations. 

\end{document}
