\documentclass[../main.tex]{subfiles}

\begin{document}
In this chapter we develop the theory of \emph{action-dependent Lagrangians}. The main appeal of
this formalism is that it allows for the description of non-conservative systems in terms of a
variational principle, which is in general not possible with standard Lagrangian
mechanics. The problem of finding the stationary paths of the action given by a Lagrangian
of this sort is known as the Herglotz problem. The main difficulty of this variational
problem is that, as opposed to the standard variational problem of Lagrangian mechanics,
it is an implicit optimisation problem. One way to approach this problem is the use of
Lagrange multipliers. We show how this leads to the equations of motion for this kind of
systems, known as the Herglotz equations, and how it can also be applied to field theory
to derive the field theoretical Herglotz equations. 

\section{The implicit formulation of the Herglotz problem}
Let's briefly describe what we will refer to as the \emph{implicit formulation of the
Herglotz problem}, as presented in \cite{Lazo2018}. First we clarify what we mean by
an action-dependent Lagrangian. The idea is to consider the action as a dynamical quantity
that changes along the path, and then allow the Lagrangian to depend on it. Naively, we
would write the following. Starting with some path \( q^\mu \colon [a,b] \to M \) in some
configuration space \( M \)\footnote{\( M \) could be space in the context of classical
mechanics or spacetime in the context of relativity} we would
write something like
\begin{equation*}
	S[q^\mu] = \int_a^b L(q^\mu(t), \dot{q}^\mu(t), S(t)) \ud t
\end{equation*}
where \( S(t) \) is the action of the path until time \( t \). Of course this makes no
sense since we are defining \( S \) on the left-hand side, and it appears on the
right-hand side! We can turn this expression into something sensible if we add the time
dependence of the action on the left-hand side, so that we write
\begin{equation}\label{eq:action integral equation}
	S[q^\mu](t) = \int_a^t L(q^\mu(\tau), \dot{q}^\mu(\tau), S[q^\mu](\tau)) \ud \tau
\end{equation}
and if we differentiate with respect to time we actually get an ODE for \( S[q^\mu] \)!
Indeed
\begin{equation} \label{eq:action ode}
	\dot{S}[q^\mu](t) = L(q^\mu(t), \dot{q}^\mu(t), S[q^\mu](t)).
\end{equation}
Notice that \cref{eq:action integral equation} actually forces the initial condition \(
S[q^\mu](a) = 0 \). We can even drop this requirement, since all we are interested in is
the difference of values of \( S \):
\begin{equation}\label{eq:action density 1d}
	S[q^\mu](b) - S[q^\mu](a) = \int_a^b L(q^\mu(t), \dot{q}^\mu(t), S[q^\mu](t)) \ud t. 
\end{equation}

What we have here is a functional which, for every path, is defined by an ODE. To find the
stationary paths of this functional we would, in principle, have to solve
\cref{eq:action ode} for any possible path and among all of them find which ones yield
extrema. This is the \emph{Herglotz variational problem}. However, just like the variational problem of classical Lagrangian mechanics can
be turned into a set of ODEs, the Euler-Lagrange equations, so can the Herglotz problem be
turned into a set of ODEs. These are known as the Herglotz equations, which can be written
down as
\begin{equation}\label{eq:herglotz equations mechanics}
	\frac{\partial L}{\partial q^\mu} - \frac{\d}{\d t}\frac{\partial L}{\partial \dot{q}^\mu}
	+\frac{\partial L}{\partial \dot{q}^\mu} \frac{\partial L}{\partial z} = 0
\end{equation}
where the Lagrangian is \( L(q^\mu, \dot{q}^\mu, z) \), \( z \) being the action
dependence. These equations differ from the Euler-Lagrange equations only by one term. And
in fact, if \( L \) is action-independent, thus \( \frac{\partial L}{\partial z} = 0 \),
we recover exactly the Euler-Lagrange equations. 

See \cite{Lazo2018} or \S 3.2 of \cite{Leon2021} for a detailed derivation of the Herglotz
equations following the implicit approach. 

\subsection{Example: the damped harmonic oscillator}
Before we move forward, let's see what the Herglotz equations look like for a particular
system. The simplest action-dependence we can introduce ---other than no
action-dependence at all, of course--- is a term linear in \( z \). We will call this
linear dissipation for reasons that will become clear in a moment. Let's see what happens
when we add this term to the simple harmonic oscillator. Explicitly, consider the
Lagrangian
\begin{equation*}
	L(q, \dot{q}, z) = \tfrac{1}{2}m\dot{q}^2 - \tfrac{1}{2} m\omega^2 q^2 - \gamma z. 
\end{equation*}
A straightforward computation shows that for this Lagrangian, \cref{eq:herglotz equations
mechanics} becomes
\begin{equation*}
	-m\omega^2 q - m\ddot{q} - \gamma m \dot{q} = 0
\end{equation*}
which after some rearranging can be turned into
\begin{equation}\label{eq:damped harmonic oscillator}
	\ddot{q} + \gamma\dot{q} + \omega^2 q = 0,
\end{equation}
which one recognises as the equation of motion of the damped harmonic oscillator. The
Herglotz principle delivers on its promise: we have just derived the equations of motion
of a fundamentally non-conservative system from a variational problem! This also clarifies
why we called the action dependence a dissipation term: the coefficient \( \gamma \) is
related to the damping of the system and governs the rate at which energy is lost after
every cycle. 

\section{The Herglotz problem as constrained optimisation}
We have derived the Herglotz equations, so it would seem we have already solved the theory
of action-dependent Lagrangians. However, general relativity is a field theory, so if we
wish to understand an action-dependent variant of it we have to know the form the Herglotz
equations take for field theory. If we try to apply the implicit method to field theory we
run into a number of problems. For one, it is just not very elegant. The action functional
is defined implicitly through an ordinary differential equation, one for every path.
What's more, this equation is no longer an ODE in field theory, but rather it becomes a
PDE. PDEs are notoriously much more difficult to solve than ODEs, so we would like a way
to circumvent this issue. 

The idea we describe in what follows is being developed by Manuel Lainz, a PhD student at
ICMAT. This is currently in a preprint stage, to which I have been graciously given
access.  It is part of broader ongoing research on the mathematics underpinning the theory
of action-dependent field theory. As it was originally introduced, it was a more elegant
way to derive the Herglotz equations from a variational principle, but actually it also
provides a method of writing down second-order Herglotz equations by taking a direct
variation of the action. This is how we will be able to calculate the field equations of
action-dependent gravity, which is a second-order theory.

The fundamental insight comes from framing the Herglotz problem as a constrained
optimisation problem.  We describe what this looks like for mehcanics. First, instead of
considering paths in spacetime, \( q^\mu \colon [a,b] \to M \), we enlarge the
configuration space with one extra quantity, which we will call \( z \). At this point \(
z \) simply tracks a quantity that changes along the path, but we will later require that
\( z \) actually match the action of the path at each time. This will be the constraint. 

So, we have paths of the form \( (q^\mu, z) \colon [a,b] \to M \times \R \). We define a
functional on these paths as
\begin{equation}\label{eq:action functional unconstrained}
	S[q^\mu, z] = z(a) - z(b) = \int_a^b \dot{z}(t) \ud t,
\end{equation}
so \( S \) is just the change in \( z \) along the trajectory. This functional as it
stands has no stationary paths, since we can find trajectories with arbitrarily large
changes in \( z \), both positive and negative. So we constrain the possible paths.
Namely, we require that \( z \) actually represent the action. We try to find the paths
that extremise \( S \) only among those that satisfy the constraint
\begin{equation}\label{eq:constraint mechanics}
	\dot{z}(t) = L(q^\mu(t), \dot{q}^\mu(t), z(t)),
\end{equation}
where \( L \) is the action-dependent Lagrangian that describes the system.  Notice that
this is very similar to \cref{eq:action ode}. 

Say \( (q^\mu, z) \) is a trajectory that satisfies \cref{eq:constraint mechanics}. Then 
\begin{equation*}
	S[q^\mu, z] = z(b) - z(a) = \int_a^b \dot{z}(t) \ud t = \int_a^b L(q^\mu(t),
	\dot{q}^\mu(t), z(t)) \ud t. 
\end{equation*}
So, for paths that satisfy the constraint, the functional \( S \) is indeed the
action functional, understood as the integral of the Lagrangian along the path.

\subsection{Lagrange multipliers}
This is all well and good, but how does one actually go about solving a constrained
optimisation problem? It turns out, we can use an infinite dimensional analog of Lagrange
multipliers to turn this into a regular optimisation problem to which we can apply the
tools of the calculus of variations. 

Recall, given some function \( f \colon \R^n \to \R \), \( x \in \R^n \) is an extremum of
\( f \) subject to \( m \) constraints \( G_k \colon \R^n \to \R \) ---i.e. \( G_k(x) = 0
\) --- if and only if there exist numbers \( \lambda_k \in \R \) such that
\( x \) is an extremum of the function
\begin{equation} \label{eq:lagrange multipliers}
	 F = f - \sum_{k = 1}^{m}\lambda_k g_k 
\end{equation}
\emph{without any constraints}. The numbers \( \lambda_k \) are called the Lagrange
multipliers. It can be shown that this result generalises to infinite dimensional spaces
and infinitely many constraints. In our case, the function we are trying to find the
extrema of is the functional \( S \). Our constraints are parameterised by \( t \in [a,b]
\):
\begin{equation*}
	G_t[q^\mu, z] = \dot{z}(t) - L(q^\mu(t), \dot{q}^\mu(t), z(t)) = 0.
\end{equation*}
Thus, replacing sums with integrals in \cref{eq:lagrange multipliers}, the extrema of
\cref{eq:action functional unconstrained} subject to \cref{eq:constraint mechanics} will
be those that extremise the following functional: 
\begin{align} \label{eq:lagrange multiplier action}
	\tilde{S}[q^\mu, z, \lambda] & = S[q^\mu, z] - \int_a^b \lambda_t G_t[q^\mu, z] \ud t
	\nonumber \\
															 & = \int_a^b \dot{z}(t) \ud t - \int_a^b \lambda_t\big[\dot{z}(t) -
		L(q^\mu(t), \dot{q}^\mu(t), z(t))\big] \ud t \nonumber \\ & = \int_a^b (1 - \lambda_t)
		\dot{z}(t) + \lambda_t\big(L(q^\mu(t), \dot{q}^\mu(t), z(t))\big) \ud t.
\end{align}
A couple of observations. First, we are thinking of the Lagrange multipliers as real
numbers parameterised by \( t \), but we could equivalently think of them as a function of
\( t \) and write \( \lambda(t) \) instead of \( \lambda_t \). We will do this.
Additionally, we introduced \( \lambda \) as a dynamical variable of the action
functional. When we take the variation of the action with respect to \( \lambda \) we will
actually recover the constraint. 

\subsection{Deriving the Herglotz equations}
If we write the integrand of \cref{eq:lagrange multiplier action} as
\begin{equation*}
	\tilde{L}(q^\mu(t), \dot{q}^\mu(t), z(t), \dot{z}(t)) = \big(1 - \lambda(t)\big) \dot{z}(t) +
	\lambda(t)\big(L(q^\mu(t), \dot{q}^\mu(t), z(t))\big)
\end{equation*}
one should be able to recognise \( \tilde{S} \) as something that looks just like a
regular old action functional defined by the integral of a regular old Lagrangian, except
it is now defined on expanded trajectories \( (q^\mu, z) \). So we should be able to use
the Euler-Lagrange equations to write down the equations of motion of its extremal paths!
The equation for \( z \) reads
\begin{equation*}
	0 = \frac{\partial \tilde{L}}{\partial z} - \frac{\d}{\d t} \frac{\partial
	\tilde{L}}{\partial \dot{z}} = \lambda \frac{\partial L}{\partial z} + \dot{\lambda}
\end{equation*}
or equivalently
\begin{equation} \label{eq:euler-lagrange multiplier}
	\dot{\lambda} = -\lambda \frac{\partial L}{\partial z}. 
\end{equation}
The Euler-Lagrange equations for the positions then are
\begin{equation*}
	0 = \frac{\partial \tilde{L}}{\partial q^\mu} - \frac{\d}{\d t} \frac{\partial
	\tilde{L}}{\partial \dot{q}^\mu} = \lambda \frac{\partial L}{\partial q^\mu} -
	\dot{\lambda}\frac{\partial L}{\partial \dot{q}^\mu} - \lambda \frac{\d}{\d
	t}\frac{\partial L}{\partial \dot{q}^\mu}
\end{equation*}
and after substituting in \cref{eq:euler-lagrange multiplier} and dividing through by \(
\lambda \) we find
\begin{equation} \label{eq:euler-lagrange positions}
	0 = \frac{\partial L}{\partial q^\mu} - \frac{\d}{\d t} \frac{\partial L}{\partial
	\dot{q}^\mu} + \frac{\partial L}{\partial z}\frac{\partial L}{\partial \dot{q}^\mu}
\end{equation}
which are exactly the Herglotz equations. 

Let us for completeness write down the equation that results from taking the variation
with respect to \( \lambda \):
\begin{equation*}
	-\dot{z}(t) + L\left(q^\mu(t), \dot{q}^\mu(t), z(t)\right) = 0,
\end{equation*}
This is exactly the constraint. 

\section{Action-dependent field theory}
We have seen how to derive the Herglotz equations in a more elegant way using Lagrange
multipliers. More importantly, we have written down a modified Lagrangian which gives rise
to them through standard calculus of variations techniques. This will be very useful as,
in some cases (general relativity being one of them), it is easier to derive the equations
of motion of a system by direct variation of the action, rather than writing down the
Euler-Lagrange equations. We will be able to do exactly this once we have the field
theoretic version of \cref{eq:lagrange multiplier action} in our hands.

\subsection{Classical field theory and Lagrangian densities}\label{sec:lagrangian
densities}
First off, we need to set the stage for Lagrangian field theory. The parameter space is no
longer just time, but rather all of spacetime, \( M \). Fields are the assignment of some
value to each point in spacetime, so we could have scalar fields, vector fields, or, as is
the case in general relativity, tensor fields. Let us first fix some notation. We will
denote by \( \phi \) some field configuration. If \( \phi \) is a scalar field then it
carries no indices. If \( \phi \) is a vector field then it carries one upper index, if it is a
tensor field then it carries multiple indices. The metric carries two lower indices since
it is a \( (0,2) \) tensor field. In almost all that follows we will assume \( \phi \) is
a vector field, but the results we find transfer to tensors of other rank. Of
course \( \phi \) depends on spacetime, which we will sometimes write explicitly as \(
\phi^a(x^\mu) \). As a convention, we will use latin indices for the indices of the field,
and reserve greek indices for spacetime coordinates. The Einstein summation convention is
assumed to be in place unless otherwise stated. 

The Lagrangian of a field theory is a function of the field and of its derivatives. If it
only contains first derivatives the theory is called a \emph{first order theory}. General
relativity is actually a second order theory, as we will discuss later. However, most of
the following discussion still applies to general relativity. 

The Lagrangian in field theory does not take values in the real numbers. To define the
action we must integrate the Lagrangian, but as opposed to in mechanics where one
integrates over time, in field theory this integration is performed over a patch of
spacetime. Because spacetime is in general a curved manifold we will need to use the
language of differential forms, which are the objects that can be integrated over
manifolds. In general, a differential \( k \)-form\footnote{A differential \( k \)-form is
a \( k \)-multilinear alternating form that acts on tangent vectors} can be integrated
over a manifold of dimension \( k \). So if spacetime has dimension \( n \), the
Lagrangian has to be a differential \( n \)-form, alsow known as a top form. 

With some care, however, we can still think of the Lagrangian as a function with values
in the real numbers.  It turns out that any two top forms differ only by an overall
factor, i.e., given two top forms \( \omega_1 \) and \( \omega_2 \), there exists a unique
scalar function \( f \colon M \to \R \) such that \( \omega_1 =
f\omega_2 \). So what this means is that, for a given Lagrangian \( \L \), once we pick a
distinguished top form \( \omega \) there is a unique scalar function \( L \) such that \( \L = L
\omega \). This distinguished top form will in most cases be the top form induced by the
coordinates we are working in, which we will write as \( \d^nx \). Sometimes we will use
Lagrangian density to refer to \( \L \) and Lagrangian to refer to \( L \). 

The setup in classical field theory is as follows. Given some Lagrangian, which encodes
the system we are studying, we define the action functional on all the possible field
configurations as 
\begin{equation*}
	S[\phi^a] = \int_D \L(\phi^a(x^\mu), \partial_\mu \phi^a(x^\mu)) = \int_D
	L(\phi^a(x^\mu), \partial_\mu \phi^a(x^\mu)) \ud^n x,
\end{equation*}
where \( D \) is some region of spacetime where this integral makes sense. 

Using the calculus of variations one can show that the stationary configurations of this
action functional satisfy the Euler-Lagrange equations of field theory
\begin{equation*}
	\frac{\partial L}{\partial \phi^a} - \partial_\mu \frac{\partial L}{\partial
	(\partial_\mu \phi ^a)} = 0. 
\end{equation*}

\subsection{The action flux}
How do we generalise this to an action dependent Lagrangian? The most naive approach would
be to try to replicate the Herglotz equations from mechanics wholesale and write down
\begin{equation*}
	\frac{\partial L}{\partial \phi^a} - \partial_\mu \frac{\partial L}{\partial
	(\partial_\mu \phi ^a)} + \frac{\partial L}{\partial(\partial_\mu \phi^a)}\frac{\partial
L}{\partial z} = 0. 
\end{equation*}
However this will not work because the last term has a pesky free \( \mu \) index. This
seems to suggest that we need to modify the nature of \( z \). We had claimed before that
\( z \) represented the action along the path, but if we look at \cref{eq:action
functional unconstrained} we see this is not quite right. In mechanics the analog of \( D
\) is \( [a,b] \). The difference \( z(a) - z(b) \) can also be written as \(
\int_{\partial [a,b]} z \), since the boundary of \( [a,b] \) is just \( a \) and \( b \).
This seems to indicate that the correct analog of \cref{eq:action functional
unconstrained} for field theory should be
\begin{equation*}
	S[\phi^a, z] = \int_{\partial D} z.
\end{equation*}

What kind of obejct should \( z \) be then? \( \partial D \) has dimension \( n-1 \), so
\( z \) has to be something we can integrate over an \( (n-1) \)-dimensional manifold,
i.e. a differential \( (n-1) \)-form. Strictly speaking, then, \( z \) is an object with
\( n-1 \) lower indices. In the case of 4-dimensional spacetime, \( z \) would have 3
antisymmetrised indices:
\begin{equation*} 
	z = z_{[\mu\nu\eta]} \ud x^{\mu} \wedge \d x^\nu \wedge \d x^\eta.
\end{equation*}
As it turns out, \( (n-1) \)-forms can be identified with vector fields. The idea is that
instead of labeling the components of \( z \) by three indices, we label them by the
missing index. Some signs appear because of the antisymmetry, which are encoded by the
Levi-Civita symbol:
\begin{equation*}
	z^\mu = \epsilon^{\mu\nu\eta\alpha} z_{[\nu\eta\alpha]}. 
\end{equation*}
So we have the identity
\begin{equation*} 
	z = z^\mu \ud x_\mu = z_{[\mu\nu\eta]} \ud x^{\mu} \wedge \d x^\nu \wedge \d x^\eta.
\end{equation*}
where
\begin{equation*}
	\ud x^\mu = \epsilon_{\mu\nu\eta\alpha} \ud x^{\nu} \wedge \d x^\eta \wedge \d x^\alpha.
\end{equation*}
But \( z \) is \emph{not} a vector, since its components do not transform as the
components of a vector, but rather as the components of a 3-form. This is very similar to
the distinction between vectors and pseudo-vectors in 3-space. 

Let's have a closer look at what \( z \) represents. We know that \( \int_{\partial D} z
\) has dimensions of action. The volume element of \( \partial D \) in natural units
(setting \( c = 1\)) has dimensions of \( \text{length}^3 \), so the components of \( z \)
have dimensions of \( \text{action}/\text{length}^3 \). This means that \( z \) is the
\emph{action flux}. From Stokes' theorem,
\begin{equation*}
	S[\phi^a, z^\mu] = \int_{\partial D} z = \int_D \ud z,
\end{equation*}
so \( \ud z \) has to be the action density, i.e. the Lagrangian. The constraint we want
to impose is therefore 
\begin{equation} \label{eq:constraint field theory}
	\ud z(x^\mu) = \L(\phi^a(x^\mu), \partial_\mu \phi^a(x^\mu), z^\mu(x^\mu))
\end{equation}
analogous to \cref{eq:constraint mechanics}. 

\( \ud z \) is the exterior derivative of \( z \), which is a top form, so this equality
makes sense. In coordinates it is easy to show that \( \ud z = \partial_\nu z^\nu \ud^n x
\), so that \cref{eq:constraint field theory} reads in coordinates as
\begin{equation} \label{eq:constraint field theory coordinates}
	\partial_\nu z^\nu (x^\mu) = L(\phi^a(x^\mu), \partial_\mu \phi^a(x^\mu)).
\end{equation}
For field configurations that satisfy this constraint then one has, applying Stokes'
theorem
\begin{equation*}
	S[\phi^a, z^\mu] = \int_{\partial D} z = \int_D \ud z = \int_D \L(\phi^a, \partial_\mu
	\phi^a, z^\mu). 
\end{equation*}
So we have arrived at the right formulation of the Herglotz variational problem for field
theory.

\subsection{Constrained optimisation in field theory}
Just like before, we will turn this constrained optimisation problem into an
unconstrained one using Lagrange multipliers. The expanded action, in analogy with
\cref{eq:lagrange multiplier action} will be
\begin{equation}\label{eq:expanded action field theory}
	\tilde{S}[\phi^a, z^\mu] = \int_D (1 - \lambda) \ud z + \lambda \L(\phi^a, \partial_\mu
	\phi^a, z^\nu) = \int_D \ud^n x \big[ (1 - \lambda) \partial_\mu z^\mu + \lambda L(\phi^a,
	\partial_\mu \phi^a, z^\nu) \big]
\end{equation}
Note that the Lagrange multiplier \( \lambda \) is a function of spacetime. Like in
mechanics, the Lagrange multiplier is technically also a dynamical quantity, but variation
with respect to it just gives back the constraint. Let us write down the integrand of
\cref{eq:expanded action field theory} as an expanded Lagrangian:
\begin{equation} \label{eq:expanded lagrangian field theory}
	\tilde{\L}(\phi^a, \partial_\mu \phi^a, z^\nu, \partial_\mu z^\nu) = \tilde{L}(\phi^a,
	\partial_\mu \phi^a, z^\nu, \partial_\mu z^\nu)\ud^n x = \big[ (1 - \lambda)
	\partial_\mu z^\mu + \lambda L(\phi^a, \partial_\mu \phi^a, z^\nu) \big] \ud^n x.
\end{equation}
The Euler-Lagrange equations for this Lagrangian are the Herglotz equations for the
theory. We write them down in the next section. Note that they are actually the Herglotz
equations for a first-order field theory. If the theory is second order then the correct
equations are the second-order Euler-Lagrange equations, which include additional terms.
Alternatively, we could still perform a direct variation of the action to get the field
equations, without worrying about the order of the theory. This is the procedure we will
follow in the next chapter. 

\subsection{The Herglotz equations for field theory}
Finally, we can derive the Herglotz equations for field theory from the expanded
Lagrangian we have just obtained in \cref{eq:expanded lagrangian field theory}. The
equations for the action flux are
\begin{equation*}
	0 = \frac{\partial \tilde{L}}{\partial z^\nu} - \partial_\mu \frac{\partial
	\tilde{L}}{\partial(\partial_\mu z^\nu)} = \lambda \frac{\partial L}{\partial z^\nu} +
	\partial_\mu(\lambda \delta_\nu^\mu) = \lambda \frac{\partial L}{\partial z^\nu} +
	\partial_\nu \lambda.
\end{equation*}
So, rearranging,
\begin{equation} \label{eq:euler lagrange multiplier field theory}
	\partial_\nu \lambda = - \lambda \frac{\partial L}{\partial z^\nu}. 
\end{equation}
This equation actually has implications for the type of action dependence that is allowed
in \( L \). We will see it again later on in the context of relativity as it forces the
dissipation form to be closed. 

And for the values of the field
\begin{equation*}
	0 = \frac{\partial \tilde{L}}{\partial \phi^a} - \partial_\mu \frac{\partial
	\tilde{L}}{\partial(\partial_\mu \phi^a)} = \lambda \frac{\partial L}{\partial \phi^a} -
	(\partial_\mu \lambda) \frac{\partial L}{\partial (\partial_\mu \phi^a) } - \lambda
	\partial_\mu \frac{\partial L}{\partial (\partial_\mu \phi^a) }
\end{equation*}
and, when plugging in \cref{eq:euler lagrange multiplier field theory} and dividing
through by \( \lambda \) we arrive at the field theoretical Herglotz equations
\begin{equation} \label{eq:herglotz field theory}
	\frac{\partial L}{\partial \phi^a} - \partial_\mu \frac{\partial
	L}{\partial(\partial_\mu \phi^a)} + \frac{\partial L}{\partial z^\mu} \frac{\partial
L}{\partial(\partial_\mu \phi^a)} = 0. 
\end{equation}

\end{document}
