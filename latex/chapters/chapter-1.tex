\documentclass[../main.tex]{subfiles}

\begin{document}
In this chapter we provide a brief overview of the state of the art of various modified
theories of gravity geared towards solving problems in cosmology. Specifically, we look at
a proposed \emph{action-dependent} theory of Einstein gravity. On the other hand, we
present the current status of the research into the mathematical formulation of
action-dependent field theories, and how gravitation fits into them. Finally we describe
how we bring these two ideas together in the present thesis. 

\section{Modified theories of gravity}
Einstein's theory of general relativity provides a description of gravity as a geometrical
effect that results from the interplay between the curvature of spacetime and the matter
that inhabits this spacetime. Its predictions have been tested to a very high degree of
accuracy. And yet, there are a number of puzzles in the field of cosmology that don't have
a satisfactory answer. As new technological developments enable more precise observations,
evidence seems to point in the direction of a modified version of Einstein's theory. There
are numerous possible avenues for generalisation, \cite{Olmo2020} lists many of the ones
under current investigation and study. Among them, we will focus on the theory of gravity
that comes from taking what is known as an \emph{action-dependent} version of the
Einstein-Hilbert Lagrangian, which is the Lagrangian from which one derives the field
equations of general relativity. This generalisation operates at the level of the
variational principle, so it is not just the addition of ad-hoc terms to the field
equations. What's more, the framework of action-dependent Lagrangians is not specific to
general relativity, but actually comes from a more general mathematical theory.

\section{Action-dependent Lagrangians and contact geometry}
Lagrangian (and Hamiltonian) mechanics can be formulated in the language of symplectic
geometry. The study of physical systems from this more abstract point of view allows one
to gain insight into how they operate at a geometric level, and provides a very elegant
way of formulating physical theories. On the other hand, all systems that can be described
by symplectic geometry are inherently conservative: the theory centers on symmetries and
conserved quantities. And yet, not all physical systems are conservative, not by any
means. It turns out that symplectic geometry has a sister theory, called contact geometry,
and that it can be used to describe non-conservative (or dissipative systems) from a
geometric point of view. This has been applied to a great number of areas of physics,
perhaps one of the more successful one being thermodynamics, see \cite{Gaset2020a,
Leon2021, Gaset2020} for recent publications in the field. Of interest to us
is the ongoing effort of adapting the theory to second-order field theory, i.e.
fields described by Lagrangians that include first and second derivatives of the field.
General relativity is a second-order field theory, so writing down the equations of a
dissipative version of it would be one of the first times this is done for a second-order
theory. 

\section{The contribution of this thesis}
The main contribution of this work is the derivation of the field equations of Einstein
gravity with an added linear action dependence. This is of interest from the point of view
of cosmology, since these equations are a starting point from which to make predictions,
and also from the point of view of theoretical physics, since these equations are the
equations of a second-order theory. In \cref{ch:herglotz} we will introduce the formalism
of action-dependent Lagrangians and introduce a new version of the formalism based on
constrained optimisation which circumvents a number of problems one encounters when
dealing with field theory. Then, in \cref{ch:einstein} we apply this idea to a modified
Einstein-Hilbert Lagrangian and derive its field equations by direct variation of the
action. This Lagrangian has actually already been studied in a recent publication,
\cite{Lazo2017}, but in \cref{ch:significance} we argue why the equations obtained there
are not the correct ones, and why the right version are those obtained in the present
work. 


\end{document}
