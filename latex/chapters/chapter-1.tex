\documentclass[../main.tex]{subfiles}

\begin{document}

\section{Bundles, sections and jets}
Field theory studies the dynamics of fields, which are, in the broadest sense, quantities
that vary from point to point in space (or spacetime). The fundamental example is a scalar
field, which is the assignment of a real value to each point in space (or spacetime). In
more precise mathematical terms, if \( M \) denotes spacetime (thought of as a manifold with a
pseudo-Riemannian metric, as is standard in general relativity), then a scalar field is a
smooth map \( \phi \colon M \to \R \). However, if we think of a field with vector values,
for instance, it is not enough to define it as a map \( X \colon M \to \R^4 \), at
least not if we want our theory to be covariant. Indeed, we think of a vector field as an
assignment of a vector at each point in spacetime which is tangent to that point. Thus,
the proper definition of a vector field is as a map \( X \colon M \to TM \), where \( TM
\) is the tangent bundle of \( M \), together with the condition \( X(p) \in T_pM \),
which is just the requirement that \( X(p) \) be tangent to \( p \). This condition makes
\( X \) a \emph{section} of \( TM \). If \( \tau_M \colon TM \to M \) is the map which
takes a tangent vector to its basepoint, then \( X \) being a section of \( TM \) is
equivalent to \( \tau_{TM} \circ X = \id_M \). 

This idea points us in the right direction for a more appropriate definition of fields. We
introduce the idea of a \emph{fiber bundle}, which is a projection map \( E
\xrightarrow{\pi} M \), such that for every \( p \in M \), \( \pi^{-1}(p) \) is
diffeomorphic to some fixed manifold \( F \). \( \pi^{-1}(p) \) is called the fibre over
\( p \), and \( F \) is the typical fibre of the bundle. What is happening here is that \(
E \) is the result of attaching a copy of \( F \) at each point of \( M \). If one
attempts to draw this, taking \( M  \) to be 2 dimensional and \( F \) 1 dimensional, the
result is something that looks much like a hair brush, where the copies of \( F \) are the
bristles, or fibres, hence the terminology. \( M \) is known as the \emph{base space}, \( E \)
receives the name \emph{total space}. 

For a more concrete picture, let's see what this looks like in coordinates. Consider a
coordinate system \( x^\mu \) for \( M \), where \( \mu \) runs over the dimension of
\( M \) ---which will be 4 if \( M \) is spacetime---, and a coordinate system \( u^\alpha
\) for the typical fibre \( F \), where \( \alpha \) runs over the dimension of \( F \).
Then points in the total space \( E \) can be labeled by \( (x^\mu, u^\alpha) \). For
example, in the tangent bundle, these coordinates label the base point of a vector and its
components with respect to a basis (which is generally taken to be basis induced by the
coordinates in physics). 

\subsection{Fields as sections of bundles}

\subsection{}



In classical mechanics, the Lagrangian is defined as a function on the velocity phase
space of the system, i.e. the tangent bundle of the configuration space \( TM \), \( L
\colon TM \to \R \). With the Lagrangian in hand, one defines the action functional as
\begin{equation*}
	S[\gamma] = \int_a^b L(\gamma(t), \dot{\gamma}(t)) \ud t. 
\end{equation*}

The precise generalisation of the Lagrangian to classical field theory is a bit more
difficult. For one, whatever the Lagrangian is, it has to be something we can integrate
over our parameter space, which in general will be a manifold. In the case of general
relativity, the parameter space is space-time, for example. This  means it has to be
related to a top form on the parameter space. The precise language to formulate this idea
is the language of jet bundles. In general, a suitable way to think of fields is as
sections of some bundle \( E \xrightarrow{\pi} M \), where \( M \) is the parameter space
and the typical fibre of the bundle is the space in which the field takes values. Then,for
a first order theory, the Lagrangian \( \L \) is a bundle morphism from the first jet bundle \(
J^1\pi \) to the bundle \( \ext^n T^\ast M \). Then, for a section \( \phi \) of \( \pi
\), \( \L \circ j^1 \phi \) is a section of \( \ext^n T^\ast M \), i.e., a top form over
\( M \). We can then define the action functional as
\begin{equation*}
	S[\phi] = \int_M \L \circ j^1 \phi
\end{equation*}



In field theory, the Lagrangian is no longer just a real-valued function, since it is
something we have to integrate over the parameter space, which is spacetime in the context
of general relativity. The object we need is going to be a top form. Specifically, if we
are studying the dynamics of sections of some bundle \( E \xrightarrow{\pi} M \), the
Lagrangian density \( \L \), or sometimes just the Lagrangian, is a bundle morphism \( \L
\colon J^1\pi \to \ext^n T^\ast M \), where \( n \) is the dimension of the parameter
space \( M \). That \( \L \) is a bundle morphism means that for any \( j_p^1 \phi \in
J^1_p \pi \), \( \L(j_p^1 \phi) \in \wedge^n T^\ast_p M \), or, more succintly, that \(
\wedge^n \tau_M^\ast \circ \L = \pi_1 \). 

When we introduce contact geometry into the picture one has to enlarge the source space
for the Lagrangian, it is now a bundle morphism of the form
\begin{equation*}
	\L \colon J^1\pi \oplus_M \ext^{n-1}T^\ast M \to \ext^{n} T^\ast M. 
\end{equation*}
The domain is the Whitney sum of the first jet bundle, where the field and its first
derivatives live, and of the bundle of \( n-1 \) forms over \( M \), where the action
density lives. In coordinates, the points in \( J^1\pi \) can be described as \( (x^\mu,
u^\alpha, u^\alpha_\mu) \). The points of \( \ext^{n-1}T^\ast M \) can be described in
coordinates by \( z^\mu \). Indeed, \( \ext^{n-1} T^\ast M \) and \( T^\ast M \) both are
vector bundles of dimension \( n \) over \( M \), so they are isomorphic. Indeed, writing
\( \d^n x \) for \( \d^1x \wedge \cdots \wedge \d^n x \), we define
\begin{equation*}
	\ast \ud x_\mu = (-1)^\mu \frac{\partial}{\partial x^\mu} \iprod \d^n x
\end{equation*}
where no summation is implied over \( \mu \). The \( \ast \ud x_\mu \) are a basis of the
\( n-1 \) forms. Then we can write for any \( z \in \ext^{n-1} T^\ast M \), \( z = z^\mu
\ast \d x_\mu \). 

For general relativity, we are interested in sections of the bundle \( \Sym^2 T^\ast M \)
of symmetric 2-forms over \( M \), specifically the subbundle of nondegenerate symmetric
2-forms, also known as pseudo-Riemannian metrics. The 

\section{Action dependent Lagrangian}
The standard formulations of classical mechanics, the Lagrangian and Hamiltonian
formalisms, can be used to describe and study a great variety of physical systems.
Nevertheless, all of them exhibit \emph{conservative} behaviour. Even the simplest non
conservative systems, i.e. dissipative, cannot be described by means of a standard Lagrangian or
Hamiltonian. One way of describing non conservative systems by means of a variational
principle is through \emph{action-dependent lagrangians}. This theory was originally
studied by Herglotz.

Recall that in ordinary classical mechanics, one models particle trajectories as
smooth paths in some configuration manifold, \( M \), which generally represents space in
Newtonian mechanics and spacetime in relativity, i.e. trajectories are of the
form \( q \colon [t_0, t_1] \to M \). The Lagrangian \( L \) is introduced as a function on the
velocity phase space, \( TM \), i.e. \( L \colon TM \to \R \), with which the action
functional is defined as the integral
\begin{equation*}
	S[q] = \int_{t_0}^{t_1} L(q^\mu(t) , \dot{q}^\mu(t)) \ud t. 
\end{equation*}
The Principle of Least Action is the statement that physical trajectories correspond to
extrema of the action functional. The condition of being an extremum of the action can be
equivalently formulated, by means of the calculus of variations, in terms of the
well-known Euler-Lagrange equations. 

In this new framework we expand the configuration space with a new variable, and consider
paths \( (q,z) \colon [t_0, t_1] \to M \times \R \). The new action functional is 
\begin{equation}\label{eq:contact action 1}
	S[q,z] = z(t_1) - z(t_0).
\end{equation}
Physical trajectories will be those that extremise \( S \) \emph{subject to the
constraint} \( \dot{z}(t) = L(q^\mu(t), \dot{q(t)}, z(t)) \). We see that if a path
satisfies this constraint then \cref{eq:contact action 1} becomes
\begin{equation}\label{eq:contact action 2}
	S[q,z] = z(t_1) - z(t_0) = \int_{t_0}^{t_1} \dot{z}(t) = \int_{t_0}^{t_1} L(q^\mu(t),
	\dot{q}^\mu(t), z(t))
\end{equation}
which makes clear why this is an action dependent Lagrangian: \( z \) represents the
change in action along the path. And if \( L \) is independent of \( z \) then
\cref{eq:contact action 2} just reduces to the usual action from Lagrangian mechanics. 

Now, it is not obvious how to solve this variational problem, since it is now a
constrained optimisation problem. However, using Lagrange multipliers, it can be shown
that extrema of this action will satisfy the so-called Herglotz equations:
\begin{equation}\label{eq:herglotz}
 	\frac{\d}{\d t}\frac{\partial L}{\partial \dot{q}^\mu} - \frac{\partial L}{\partial
 	q^\mu} - \frac{\partial L}{\partial z} \frac{\partial L}{\partial \dot{q}^\mu} = 0
\end{equation}
These differential equations are the action dependent analog to the Euler-Lagrange
equations, and notice that, as expected, if \( L \) is independent of \( z \) then the
Herglotz equations reduce to Euler-Lagrange. 

To see that this is useful to describe physical phenomena, consider the following action
dependent Lagrangian:
\begin{equation*}
	L(q, \dot{q}, z) = \frac{1}{2}m\dot{q}^2 - \frac{1}{2}m\omega^2 q^2 - \gamma z. 
\end{equation*}
The Herglotz equation for this Lagrangian is
\begin{equation*}
	m\ddot{q} + m\omega^2 q + \gamma m \dot{q} = 0  
\end{equation*}
which one recognises as the equation of motion of a damped harmonic oscillator. 

\end{document}
