\documentclass[../main.tex]{subfiles}

\begin{document}

In classical mechanics, the Lagrangian is defined as a function on the velocity phase
space of the system, i.e. the tangent bundle of the configuration space \( TM \), \( L
\colon TM \to \R \). With the Lagrangian in hand, one defines the action functional as
\begin{equation*}
	S[\gamma] = \int_a^b L(\gamma(t), \dot{\gamma}(t)) \ud t. 
\end{equation*}

The precise generalisation of the Lagrangian to classical field theory is a bit more
difficult. For one, whatever the Lagrangian is, it has to be something we can integrate
over our parameter space, which in general will be a manifold. In the case of general
relativity, the parameter space is space-time, for example. This  means it has to be
related to a top form on the parameter space. The precise language to formulate this idea
is the language of jet bundles. In general, a suitable way to think of fields is as
sections of some bundle \( E \xrightarrow{\pi} M \), where \( M \) is the parameter space
and the typical fibre of the bundle is the space in which the field takes values. Then,for
a first order theory, the Lagrangian \( \L \) is a bundle morphism from the first jet bundle \(
J^1\pi \) to the bundle \( \ext^n T^\ast M \). Then, for a section \( \phi \) of \( \pi
\), \( \L \circ j^1 \phi \) is a section of \( \ext^n T^\ast M \), i.e., a top form over
\( M \). We can then define the action functional as
\begin{equation*}
	S[\phi] = \int_M \L \circ j^1 \phi
\end{equation*}



In field theory, the Lagrangian is no longer just a real-valued function, since it is
something we have to integrate over the parameter space, which is spacetime in the context
of general relativity. The object we need is going to be a top form. Specifically, if we
are studying the dynamics of sections of some bundle \( E \xrightarrow{\pi} M \), the
Lagrangian density \( \L \), or sometimes just the Lagrangian, is a bundle morphism \( \L
\colon J^1\pi \to \ext^n T^\ast M \), where \( n \) is the dimension of the parameter
space \( M \). That \( \L \) is a bundle morphism means that for any \( j_p^1 \phi \in
J^1_p \pi \), \( \L(j_p^1 \phi) \in \wedge^n T^\ast_p M \), or, more succintly, that \(
\wedge^n \tau_M^\ast \circ \L = \pi_1 \). 

\end{document}
