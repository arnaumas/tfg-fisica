\documentclass[../main.tex]{subfiles}

\begin{document}

Einstein's theory of general relativity provides a description of gravity as a geometrical
effect that results from the interplay between the curvature of spacetime and the matter
that inhabits this spacetime. Its predictions have been tested to a very high degree of
accuracy. And yet, there are a number of puzzles in the field of cosmology that don't have
a satisfactory answer. As new technological developments enable more precise observations,
evidence seems to point in the direction of a modified version of Einstein's theory. There
are numerous possible avenues for generalisation, \cite{Olmo2020} lists many of the ones
under current investigation and study. Among them, we will focus on the theory of gravity
that comes from taking what is known as an \emph{action-dependent} version of the
Einstein-Hilbert Lagrangian, the Lagrangian from which the field
equations of general relativity are derived. This has been first developed in
\cite{Lazo2017}. This generalisation operates at the level of the
variational principle, so it is not just the addition of ad-hoc terms to the field
equations. 

Lagrangian (and Hamiltonian) mechanics can be formulated in the language of symplectic
geometry. The study of physical systems from this more abstract point of view allows one
to gain insight into how they operate at a geometric level, and provides a very elegant
way of formulating physical theories. On the other hand, all systems that can be described
by symplectic geometry are inherently conservative: the theory centers on symmetries and
conserved quantities. And yet, not all physical systems are conservative, not by any
means. It turns out that symplectic geometry has a sister theory, called contact geometry,
and that it can be used to describe non-conservative (or dissipative systems) from a
geometric point of view. This has been applied to various areas of physics,
the most notable one being reversible and non-reversible thermodynamics (see
\cite{Mrugala1991}), as well as other areas such as control theory and cosmology
(\cite{Lazo2017}). So far, contact geometry is well understood for mechanics, but not so
much for field theory and even less for higher order field theories. See \cite{Gaset2020a,
Leon2021, Gaset2020} for recent publications related to ongoing efforts. General
relativity is a second-order field theory, so writing down the equations of a dissipative
version of it would be one of the first times this is done for a second-order theory. 

The main contribution of this work is the derivation of the field equations of Einstein
gravity with an added linear action dependence. This is of interest from the point of view
of cosmology, since these equations are a starting point from which to make predictions,
and also from the point of view of mathematical physics, since these equations are the
equations of a second-order theory.

The manuscript is organised as follows: in \cref{ch:herglotz} we will introduce the formalism
of action-dependent Lagrangians and show how it is a problem of constrained optimisation.
This is a very recent breakthrough which circumvents many of the problems one encounters
when dealing with action-depenednet field theories. Then, in \cref{ch:einstein} we apply
this idea to a modified Einstein-Hilbert Lagrangian and derive its field equations by
direct variation of the action. This Lagrangian has actually already been studied in a
recent publication, \cite{Lazo2017}, but in \cref{ch:significance} we argue why the
equations obtained there are not the correct ones, and why the right version are those
obtained in the present work. 

\end{document}
