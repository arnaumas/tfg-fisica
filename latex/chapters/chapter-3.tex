\documentclass[../main.tex]{subfiles}

\begin{document}
\section{The Einstein-Hilbert Lagrangian}
As is well-known, the Einstein field equations can be derived by means of a variational
principle. The Lagrangian that gives rise to these equations is the Einstein-Hilbert
Lagrangian. Let's write it down in the language of \cref{sec:lagrangian densities}. The
Einstein-Hilbert action acts on metrics. And given a metric \( g \), one can construct a
top form, which we will write \( \omega_g \). Choosing coordinates one has \( \omega_g =
\sqrt{g} \ud^4 x \), where \( \sqrt{g} \) is the square root of the determinant of the
expression of \( g \) in the coordinates that induce \( \d^4 x \). This form is not the
Einstein-Hilbert Lagrangian. The other element is the scalar curvature \( R = g^{ab}R_{ab}
\), which is a Lorentz invariant scalar that encodes the curvature associated to \( g \).
The Einstein-Hilbert Lagrangian is
\begin{equation}\label{eq:EH lagrangian}
	\L_{\text{E-H}}(g_{ab}, \partial_\mu g_{ab}, \partial_\mu \partial_\nu g_{ab}) =
	R\omega_g = R \sqrt{g}\ud^4 x.
\end{equation}
From now on we will write \( g_{ab,\mu} \) and \( g_{ab,\mu\nu} \) instead of \(
\partial_\mu g_{ab} \) and \( \partial_\mu\partial_\nu g_{ab} \). 

The Einstein-Hilbert action is therefore
\begin{equation} \label{eq:EH action}
	S_{\text{E-H}}[g_{ab}] = \int_D \L_\text{E-H}(g_{ab}, g_{ab,\mu}, g_{ab,\mu\nu}) = \int_D R
	\sqrt{g}\ud^4x.
\end{equation}
A variation of this action leads one to the Einstein field equations, which in natural
units are
\begin{equation} \label{eq:EFE vacuum}
	R_{ab} - \tfrac{1}{2}g_{ab}R = 0. 
\end{equation}
More precisely, these are the Einstein field equations in a vacuum, since one can add
various matter terms to the Einstein-Hilbert Lagrangian which leads to the Einstein
equations in the presence of matter,
\begin{equation} \label{eq:EFE matter}
	R_{ab} - \tfrac{1}{2}g_{ab}R = T_{ab}. 
\end{equation}
The object \( T_{ab} \) is the energy-momentum tensor and collects all of the terms coming
from the presence of matter. See \S4 of \cite{Carroll1997} for a derivation of the
Einstein field equations. 

\section{An action dependent Einstein-Hilbert Lagrangian}
What kind of action dependence can we incorporate into the Einstein-Hilbert Lagrangian?
The simplest one is perhaps the following
\begin{equation} \label{eq:action dependent einstein-hilbert}
	\L_{\text{E-H}}(g_{ab}, g_{ab,\mu}, g_{ab,\mu\nu}, z^\mu) = R\omega_g - \theta \wedge z. 
\end{equation}
Recall, \( z \) is the action flux which is an \( (n-1) \)-differential form (so a 3-form
in general relativity), so, if \( \theta \) is a given covector field (a 1-form), the wedge \(
\theta \wedge z \) is a 4-form and \cref{eq:action dependent einstein-hilbert} makes
perfect sense. 

In coordinates we have
\begin{equation*}
	\theta \wedge z = (\theta_\mu \ud x^\mu) \wedge (z^\nu \ud x_\nu) = \theta_\mu z^\nu \ud
	x^\mu \wedge \ud x_\nu = \theta_\mu z^\nu \delta^\mu_\nu \ud^4 x = \theta_\mu z^\mu
	\ud^4x
\end{equation*}
so
\begin{equation}\label{eq:action dependent EH coordinates 1}
	\L_{\text{E-H}}(g_{ab}, g_{ab,\mu}, g_{ab,\mu\nu}, z^\mu) = (R\sqrt{g} - \theta_\mu
	z^\mu) \ud^4 x. 
\end{equation}
This Lagrangian does not quite match the one proposed in eq. (9) of \cite{Lazo2017}. It
can be shown that they are in fact one and the same. If instead of choosing the basis \(
\d x^\mu \) for the 3- forms we expand with respect to the basis \( \omega_{g,\mu} = \sqrt{g}
\ud x_\mu \) we have
\begin{equation*}
	z = \zeta^\mu \omega_{g,\mu} = \zeta^\mu \sqrt{g} \ud x_\mu
\end{equation*}
which implies \( z^\mu = \sqrt{g} \zeta^\mu \). So, with these new coordinates for the
action flux, \cref{eq:action dependent EH coordinates 1} looks like
\begin{equation}\label{eq:action dependent EH coordinates 2}
	\L_{\text{E-H}}(g_{ab}, g_{ab,\mu}, g_{ab,\mu\nu}, \zeta^\mu) = (R\sqrt{g} - \theta_\mu
	\zeta^\mu \sqrt{g}) \ud^4 x = (R - \theta_\mu \zeta^\mu)\omega_g
\end{equation}
which is the same Lagrangian proposed in \cite{Lazo2017}. 

We now write down the constraint in \cref{eq:constraint field theory} for this Lagrangian.
In the original coordinates for the action flux we have
\begin{equation*}
	\ud z = \partial_\mu z^\mu \ud^4 x
\end{equation*}
so, in coordinates
\begin{equation} \label{eq:constraint coordinates 1}
	\partial_\mu z^\mu = R\sqrt{g} - \theta_\mu z^\mu. 
\end{equation}
If instead we choose the other coordinates for the action flux, we see
\begin{equation*}
	\ud z = \partial_\mu (\sqrt{g} \zeta^\mu) \ud^4 x = \nabla_\mu \zeta^\mu \sqrt{g}\ud^4x
	= \nabla_\mu \zeta^\mu \omega_g
\end{equation*}
where \( \nabla \) is the covariant derivative induced by \( g \). In these coordinates
the constraint takes the form
\begin{equation} \label{eq:constraint coordinates 2}
	\nabla_\mu \zeta^\mu = R - \theta_\mu \zeta^\mu
\end{equation}
which is the same form that appears in \cite{Lazo2017}. 

\section{The expanded action}
So we have seen what the Herglotz problem looks like for an Einstein-Hilbert Lagrangian
with a linear action dependence. We will apply the method of Lagrange multipliers, as
described in previous chapter, to derive a modified version of Einstein's equations. The
expanded Lagrangian is
\begin{equation*}
	\tilde{\L}_\text{E-H}(g_{ab}, g_{ab,\mu}, g_{ab,\mu\nu}, z^\nu, \partial_\mu z^\nu) =
	\big[(1-\lambda) \partial_\mu z^\mu + \lambda(R\sqrt{g} - \theta_\mu z^\mu)\big] \ud^4
	x.
\end{equation*}

Let's therefore compute the variation of the action given by this Lagrangian
\begin{align}
	\delta \tilde{S}[g_{ab}, z^\nu] & = \int_D \big[(1-\lambda) \delta \partial_\mu z^\mu +
	\lambda(\delta(R\sqrt{g}) - \theta_\mu \delta z^\mu)\big] \ud^4 x \notag \\
																	& = \int_D (1 - \lambda) \partial_\mu \delta z^\mu -
																	\lambda \theta_\mu \delta z^\mu \ud^4x + \int_D \lambda
																	\delta(R \sqrt{g}) \ud^4 x \notag \\
																	& = \int_{D} \partial_\mu \big((1 -
																	\lambda)\delta z^\mu\big) \ud^4 x + \int_D (\partial_\mu \lambda
																	- \lambda\theta_\mu) \delta z^\mu \ud^4 x + \int_D \lambda
																	\delta(R \sqrt{g}) \ud^4 x \label{eq:variation expanded
																	action}. 
\end{align}
The first integral is a boundary term coming from an integration by parts. Since we are
assuming the variations vanish at the boundary of \( D \) so must this boundary term also
vanish. From the second integral we can read off, using the fundamental theorem of the
calculus of variations, that
\begin{equation} \label{eq:action flux variation}
	\partial_\mu \lambda = \lambda\theta_\mu.
\end{equation}
Coordinate free this can also be written as \( \ud \lambda = \lambda \theta \). This has
an interesting implication for \( \theta \) since
\begin{equation*}
	\ud(\lambda \theta) = \ud \lambda \wedge \theta + \lambda \ud \theta = \lambda \theta
	\wedge \theta + \lambda \ud \theta = \lambda \ud \theta
\end{equation*}
and 
\begin{equation*}
	\lambda \ud \theta = \ud(\lambda \theta) = \ud^2 \lambda = 0
\end{equation*}
so we conclude \( \ud \theta = 0 \), i.e. \( \theta \) cannot be any 1-form, it must be a
a \emph{closed form}. This means that in coordinates \( \partial_\mu \theta_\nu =
\partial_\nu \theta_\mu \). 

We retake the calculation from \cref{eq:variation expanded action}. Since the integrals
involving \( z \) and \( g \) decouple, we can just consider the last term. We will follow the
derivation in \cite{Carroll1997} for as long as we can. Since \( R\sqrt{g} =
g^{ab}R_{ab}\sqrt{g} \), from the product rule its variation results in three terms:
\begin{equation}\label{eq:variation of scalar curvature}
	\int_D \lambda \delta(R \sqrt{g}) \ud^4 x = \int_D \lambda \delta g^{ab} R_{ab} \sqrt{g}
	\ud^4 x + \int_D \lambda g^{ab} \delta R_{ab} \sqrt{g} \ud^4 x + \int_{D} \lambda R
	\delta\sqrt{g} \ud^4 x
\end{equation}
The first term is already in the form required to apply the fundamental theorem of the
calculus of variations. For the third one uses the standard result
\begin{equation*}
	\delta \sqrt{g} = -\tfrac{1}{2}\sqrt{g} g_{ab} \delta g^{ab}.
\end{equation*}
The first and third terms of \cref{eq:variation of scalar curvature} can be combined into
\begin{equation}\label{eq:variation vacuum}
	\int_D \lambda (R_{ab} - \tfrac{1}{2} Rg_{ab}) \delta g^{ab} \sqrt{g} \ud^4 x.
\end{equation}
In the standard derivation of Einstein's equations, one shows that the middle integral of
\cref{eq:variation of scalar curvature} actually vanishes, so that \cref{eq:variation
vacuum} must vanish for any variation \( \delta g_{ab} \), or equivalently for any
variation of the inverse metric \( \delta g^{ab} \). Therefore the integrand itself must
vanish, which gives Einstein's equations. In the presence of \( \lambda \), however, the
middle integral doesn't vanish and actually contributes additional terms to the
equations.

The variation of the Ricci curvature can be shown to be
\begin{equation}\label{eq:variation ricci curvature}
	g^{ab} \delta R_{ab} = g^{ab}(\nabla_m \delta{\Gamma^m}_{ab} - \nabla_a \delta
	{\Gamma^m}_{mb}) = \nabla_n(g^{ab} {\delta\Gamma^n}_{ab} - g^{nb} \delta
	{\Gamma^m}_{mb})
\end{equation}
so
\begin{equation*}
	\int_D \lambda g^{ab}\delta R_{ab} \sqrt{g} \ud^4 x = \int_D \lambda \nabla_n(g^{ab}
	{\delta\Gamma^n}_{ab} - g^{nb} \delta {\Gamma^m}_{mb}) \sqrt{g} \ud^4x
\end{equation*}
and if \( \lambda \) weren't there this integral would vanish because of the divergence
theorem and the fact that the variations vanish on the boundary of \( D \). But \( \lambda
\) is there, so we must work on this integral some more:
\begin{align*}
	& \int_D \lambda g^{ab}\delta R_{ab} \sqrt{g} \ud^4 x = \\
	& \quad = \int_D \lambda \nabla_n(g^{ab} {\delta\Gamma^n}_{ab} - g^{nb} \delta
	{\Gamma^m}_{mb}) \sqrt{g} \ud^4x \\
	& \quad = \int_D \lambda \partial_n \left( \sqrt{g}(g^{ab} {\delta\Gamma^n}_{ab} - g^{nb} \delta
	{\Gamma^m}_{mb})\right) \ud^4 x \\
	& \quad = \int_D \partial_n \left( \lambda\sqrt{g}(g^{ab} {\delta\Gamma^n}_{ab} - g^{nb}
	\delta {\Gamma^m}_{mb})\right) \ud^4 x - \int_D (\partial_n \lambda) (g^{ab}
	{\delta\Gamma^n}_{ab} - g^{nb} \delta {\Gamma^m}_{mb}) \sqrt{g} \ud^4 x. 
\end{align*}
The first integral vanishes because it is the integral of a divergence and the variations
vanish on the boundary of \( D \). So we are left with just the second term, which we
split into two integrals. The variation of the Christoffel symbols can be shown to be
\begin{equation} \label{eq:variation christoffel symbols}
	\delta {\Gamma^a}_{bc} = \tfrac{1}{2} g^{am}(\nabla_c \delta g_{bm} + \nabla_b \delta
	g_{mc} - \nabla_m \delta g_{bc})
\end{equation}
Using this and \cref{eq:action flux variation} we compute for the first integral
\begin{equation}\label{eq:first three terms}
	-\int_D (\partial_n \lambda) g^{ab} \delta{\Gamma^n}_{ab} \sqrt{g} \ud^4 x = -
	\tfrac{1}{2} \int_D \lambda\theta_n g^{ab} g^{nk}(\nabla_b \delta g_{ak} + \nabla_a
	\delta g_{kb} - \nabla_k \delta g_{ab}) \sqrt{g} \ud^4 x. 
\end{equation}
The presence of \( g^{ab} \) means the indices \( a \) and \( b \) are symmetrised, so
\begin{equation*}
	g^{ab} \nabla_b \delta g_{ak} = g^{ab} \nabla_a \delta g_{kb}. 
\end{equation*}
This means \cref{eq:first three terms} simplifies to
\begin{align}
	&	-\int_D (\partial_n \lambda) g^{ab} \delta{\Gamma^n}_{ab} \sqrt{g} \ud^4 x = \notag \\
	& \quad = - \int_D \lambda \theta_n g^{ab}g^{nk} \nabla_b \delta g_{ak} \sqrt{g} \ud^4 x
	+ \tfrac{1}{2} \int_D \lambda \theta_n g^{ab}g^{nk} \nabla_k \delta g_{ab} \sqrt{g}
	\ud^4 x \notag \\
	& \quad = - \int_D \lambda \theta_n \nabla_b (g^{ab}g^{nk} \delta g_{ak}) \sqrt{g} \ud^4 x
	+ \tfrac{1}{2} \int_D \lambda \theta_n \nabla_k (g^{ab}g^{nk}\delta g_{ab}) \sqrt{g}
	\ud^4 x. \label{eq:two integrals}
\end{align}
Let's try to perform an integration by parts for the first integral. We have to be a bit
careful. Introducing the shorthand \( X^{bn} = g^{ab}g^{nk}\delta g_{ak} \), we compute
\begin{equation*}
	\nabla_c(\lambda\theta_n X^{bn}) = \nabla_c(\lambda \theta_n)X^{bn} + \lambda \theta_n
	\nabla_c X^{bn}
\end{equation*}
so
\begin{align*}
	-\int_D \lambda \theta_n \nabla_b (g^{ab}g^{nk} \delta g_{ak})\sqrt{g}\ud^4x 
	& = - \int_D \lambda \theta_n \nabla_bX^{bn} \sqrt{g} \ud^4 x \\
	& = - \int_D \nabla_b(\lambda \theta_n X^{bn})\sqrt{g}\ud^4x + \int_{D}
	\nabla_b(\lambda\theta_n)X^{bn}\sqrt{g}\ud^4x. 
\end{align*}
The first integral is the integral of a divergence, so it vanishes. We are left with the
second which we can expand into
\begin{align*}
	\int_D \nabla_b(\lambda \theta_n) X^{bn}\sqrt{g}\ud^4 x 
	& = \int_D (\partial_b \lambda \theta_n + \lambda \nabla_b\theta_n)(g^{ab}g^{nk}\delta
	g_{ak})\sqrt{g} \ud^4 x \\
	& = \int_D \lambda(\theta_b\theta_n +
	\nabla_b\theta_n)(g^{ab}g^{nk}\delta g_{ak})\sqrt{g}\ud^4 x
\end{align*}
As a last step, we use the identity
\begin{equation*}
	\delta g^{ab} = - g^{am}g^{bn} \delta g_{mn}
\end{equation*}
to write our integral as a variation with respect to the inverse metric.
\begin{align*}
	\int_D \lambda(\theta_b\theta_n + \nabla_b\theta_n)(g^{ab}g^{nk}\delta
	g_{ak})\sqrt{g}\ud^4 x 
	& = -\int_D \lambda(\theta_b\theta_n + \nabla_b \theta_n)\delta g^{bn} \sqrt{g} \ud^4 x. 
\end{align*}

Without going through the details again, the other integral in \cref{eq:two integrals} can
be brought to the form
\begin{align*} \label{eq:}
	\tfrac{1}{2} \int_D \lambda \theta_n \nabla_k (g^{ab}g^{nk}\delta g_{ab}) \sqrt{g} \ud^4
	x 
	& = - \tfrac{1}{2} \int_D \nabla_k(\lambda \theta_n) g^{ab}g^{nk}\delta g_{ab} \sqrt{g}
	\ud^4 x \\
	& = \tfrac{1}{2} \int_D \lambda(\theta_k\theta_n + \nabla_k\theta_n) g^{ab}g^{nk}
	g_{ma}g_{lb}\delta g^{ml} \sqrt{g}\ud^4 x \\
	& = \tfrac{1}{2}\int_D \lambda g^{nk}(\theta_k\theta_n + \nabla_k \theta_n)g_{ml} \delta
	g^{ml} \sqrt{g} \ud^4 x
\end{align*}

There is still another integral we need to evaluate, namely
\begin{align}
	\int_D (\partial_n \lambda) g^{nb}\delta{\Gamma^m}_{mb} \sqrt{g} \ud^4 x 
	& = \tfrac{1}{2} \int_D \lambda \theta_n g^{nb} g^{mk}(\nabla_b \delta g_{mk} + \nabla_m
	\delta g_{kb} - \nabla_k \delta g_{mb}) \sqrt{g} \ud^4 x. 
\end{align}
Because \( m \) and \( k \) are symmetrised, the second and third terms cancel, leaving
us with
\begin{align}
	\tfrac{1}{2}\int_D \lambda \theta_n g^{nb}g^{mk}\nabla_b \delta g_{mk} \sqrt{g} \ud^4 x
	& = - \tfrac{1}{2} \int_D \nabla_b(\lambda \theta_n) g^{nb}g^{mk} \delta g_{mk} \sqrt{g}
	\ud^4 x \\
	& = \tfrac{1}{2} \int_D \lambda(\theta_b\theta_n + \nabla_b
	\theta_n)g^{nb}g^{mk}g_{am}g_{lk}\delta g^{al} \sqrt{g} \ud^4 x \\
	& = \tfrac{1}{2} \int_D \lambda g^{nb}(\theta_b \theta_n + \nabla_b \theta_n)
	g_{al} \delta g^{al} \sqrt{g} \ud^4 x. 
\end{align}

We have calculated all the integrals we need. Before we put them all together, let us make
the following observation:
\begin{equation*}
	\nabla_a \theta_b = \partial_a \theta_b - {\Gamma^m}_{ab} \theta_m = \partial_b \theta_a
	- {\Gamma^m}_{ba} \theta_m = \nabla_b \theta_a
\end{equation*}
which uses the fact that \( \theta \) must be closed. Therefore we can define the
follwoing symmetric (0,2) tensor
\begin{equation} \label{eq:K}
	K_{ab} = \theta_a\theta_b + \nabla_{(a}\theta_{b)}. 
\end{equation}
So, after liberal relabeling of indices, we find that \cref{eq:variation expanded action}
becomes
\begin{equation} \label{eq:final variation of action}
	\delta\tilde{S}[g_{ab}, z^\mu] = \int_D(\partial_\mu \lambda - \lambda \theta_\mu)
	\delta z^\mu \ud^4 x + \int_D \lambda (R_{ab} - \tfrac{1}{2} Rg_{ab} - K_{ab} +
	Kg_{ab})\delta g^{ab} \sqrt{g} \ud^4 x.
\end{equation}
Applying the fundamental theorem of the calculus of variations, the action will be
stationary if and only if the integrands of both terms vanish. From the first integral we
get \cref{eq:action flux variation}, which we have already used. And from the second one
we get the modified Einstein field equations
\begin{equation} \label{eq:modified EFE}
	R_{ab} - \tfrac{1}{2}Rg_{ab} - K_{ab} + Kg_{ab} = 0
\end{equation}
with \( K_{ab} \) defined as in \cref{eq:K} and \( K = g^{mn}K_{mn} \) its trace. 

Note that \( K \) is indeed a tensor. One could see this by showing that it transforms
like one, or alternatively by observing that \( K_{ab} \) are the components of \( \theta
\otimes \theta + \nabla \theta \), which is certainly a tensor. 


% Let's try and write down a Herglotz-like variational principle for general relativity.
% The fundamental equations of general relativity are the Einstein field equations. In a
% vacuum and without considering the cosmological constant, they take the form
% \begin{equation} \label{eq:einstein equations}
% 	R_{ab} - \frac{1}{2}g_{ab}R = 0.
% \end{equation}
% These are a set of partial differential equations which govern the metric of spacetime, \(
% g \). \( R_{ab} \) is the Ricci curvature tensor, which is a function of the components of
% the metric \( g_{ab} \), and \( R = g^{ab}R_{ab} \) is its trace, the scalar curvature. There are various ways
% one can arrive at these equations, Einstein's original derivation was essentially
% heuristic. However, they can also be derived from a variational principle, as Hilbert
% showed. The corresponding action is therefore known as the Einstein-Hilbert action and is
% defined as
% \begin{equation} \label{eq:}
% 	S_{\text{EH}}[g] = \int_{D} R(g_{ab}, \partial_{\mu}g_{ab}, \partial_{\mu\nu}g_{ab})
% 	\sqrt{g} \ud^4 x,
% \end{equation}
% where \( R \) is the scalar curvature, just as before, we just have been explicit about
% what it depends on and \( \sqrt{g} \) is the square root of the determinant of \( g \). In
% a way, this is the simplest action one can write down, because the simplest covariant
% scalar that contains some information about the curvature of spacetime is \( R \). The
% factor of \( \sqrt{g} \) is needed to turn the integrand into a density we can integrate.

% Let's try to understand this in the language we have developed in the previous chapter. We
% will have to work backwards, because we start from an action instead of a Lagrangian.
% First, we need to identify the appropriate space where all the possible metrics we can
% equip spacetime with live. Recall, a spacetime metric (or Lorentzian metric) is a \( (0,2)
% \) symmetric tensor field which is nowhere degenerate, and additionally it has signature
% \( (1,3) \)\footnote{or \( (3,1) \) depending on the convention}. This means \(
% S_{\text{EH}} \) takes values over \( \Gamma(\Lor(M)) \), the space of sections of the
% bundle of Lorentzian metrics \( \Lor(M) \subseteq \Sym^2(M) \). 

% Recall that the Lagrangian is no longer just a function of real values, but a density. We
% can write down the Einstein-Hilbert Lagrangian in coordinate free manner as
% \begin{equation} \label{eq:einstein hilbert lagrangian}
% 	\L_{\text{EH}} = R \omega_g
% \end{equation}
% where \( \omega_g \) is the volume form induced by the metric. From this we see \( \L
% \colon J^2 \Lor(M) \to \ext^4 T^\ast M \). 

% In analogy with mechanics, an action dependent lagrangian density will have as its domain
% an enlarged version of the usual classical field theory domain. Specifically, it will look
% something like
% \begin{equation} \label{eq:}
% 	\L \colon J^2 \Lor(M) \oplus_M \ext^{3} T^\ast M \to \ext^4 T^\ast M 
% \end{equation}
% This means we have at our disposal a Lorentzian metric and a 3-form, and we have to
% construct out of them a 4-form. We know how to do that with the metric. The simplest
% action-dependence we can introduce is, given some 1-form over spacetime \( \lambda \), we
% consider the Lagrangian density
% \begin{equation}
% 	\L = R \omega_g - \lambda \wedge z. 
% \end{equation}
% Let's see what this looks like in coordinates. If we expand \( z \) the basis we have
% described for \( \ext^3 T^\ast M \) then
% \begin{equation*}
% 	z = z^\mu \ud x_\mu
% \end{equation*}
% and if \( \lambda = \lambda_\nu \ud x^\nu \) then
% \begin{equation*}
% 	\lambda \wedge z = (\lambda_\nu \ud x^\nu) \wedge (z^\mu \ud x_\mu) = \lambda_\nu z^\mu
% 	\ud x^\nu \wedge \d x_\mu = \lambda_\nu z^\mu \delta_\mu^\nu \ud^4 x = \lambda_\mu z^\mu
% 	\ud^4 x
% \end{equation*}
% and \( R \omega_g = R \sqrt{g} \ud^4 x \) so that
% \begin{equation*}
% 	\L = (R \sqrt{g} - \lambda_\mu z^\mu) \ud^4 x. 
% \end{equation*}

% The constraint is \( \ud z = \L \). It is easy to show that \( \d z = \partial_\mu z^\mu
% \ud^4 x \), so that the coordinate expression of the constraint is
% \begin{equation*}
% 	\partial_\mu z^\mu = R\sqrt{g} - \lambda_\mu z^\mu. 
% \end{equation*}



\end{document}
