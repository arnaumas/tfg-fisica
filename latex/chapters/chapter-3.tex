\documentclass[../main.tex]{subfiles}

\begin{document}
In this chapter we apply the language and tools developed in the previous chapter to the
specific case of Einstein gravity. We first describe the Lagrangian from which the Einstein
field equations come from and then introduce an action-dependent version of this Lagrangian and
derive its field equations. 

\section{The Einstein-Hilbert Lagrangian}
As is well-known, the Einstein field equations can be obtained from a variational
principle. The Lagrangian that gives rise to these equations is the Einstein-Hilbert
Lagrangian. Let's write it down in the language of \cref{sec:lagrangian densities}. The
Einstein-Hilbert action acts on metrics. Given a metric \( g \), one can construct a
top form, which we will write \( \omega_g \). Choosing coordinates one has \( \omega_g =
\sqrt{g} \ud^4 x \), where \( \sqrt{g} \) is the square root of the determinant of the
expression of \( g \) in the coordinates that induce \( \d^4 x \). This is not yet the
Einstein-Hilbert Lagrangian. The other element is the scalar curvature \( R = g^{ab}R_{ab}
\), which is a Lorentz invariant scalar that encodes the curvature associated to \( g \).
In coordinates, the Ricci tensor \( R_{ab} \) takes the form
\begin{equation} \label{eq:ricci tensor}
	R_{ab} = \partial_m{\Gamma^m}_{ab} - \partial_a {\Gamma^m}_{mb} + {\Gamma^m}_{mn}
	{\Gamma^n}_{ab} - {\Gamma^m}_{an}{\Gamma^n}_{mb}
\end{equation}
and since the Christoffel symbols contain first derivatives of the metric, \( R_{ab} \) and
hence \( R \) contain second derivatives of the metric. 

The Einstein-Hilbert Lagrangian is
\begin{equation}\label{eq:EH lagrangian}
	\L_{\text{E-H}}(g_{ab}, \partial_\mu g_{ab}, \partial_\mu \partial_\nu g_{ab}) =
	R\omega_g = R \sqrt{g}\ud^4 x.
\end{equation}
This is a second order Lagrangian. From now on we will write \( g_{ab,\mu} \) and \(
g_{ab,\mu\nu} \) instead of \( \partial_\mu g_{ab} \) and \( \partial_\mu\partial_\nu
g_{ab} \). 

The Einstein-Hilbert action is therefore
\begin{equation} \label{eq:EH action}
	S_{\text{E-H}}[g_{ab}] = \int_D \L_\text{E-H}(g_{ab}, g_{ab,\mu}, g_{ab,\mu\nu}) = \int_D R
	\sqrt{g}\ud^4x.
\end{equation}
A variation of this action leads one to the Einstein field equations, which in natural
units are
\begin{equation} \label{eq:EFE vacuum}
	R_{ab} - \tfrac{1}{2}g_{ab}R = 0. 
\end{equation}
More precisely, these are the Einstein field equations in a vacuum, since one can add
various matter terms to the Einstein-Hilbert Lagrangian which leads to the Einstein
equations in the presence of matter,
\begin{equation} \label{eq:EFE matter}
	R_{ab} - \tfrac{1}{2}g_{ab}R = T_{ab}. 
\end{equation}
The object \( T_{ab} \) is the energy-momentum tensor and collects all of the terms coming
from the presence of matter. See \S4 of \cite{Carroll1997} for a derivation of the
Einstein field equations. 

\section{An action dependent Einstein-Hilbert Lagrangian}
What kind of action dependence can we incorporate into the Einstein-Hilbert Lagrangian?
The simplest one is a linear dissipation term:
\begin{equation} \label{eq:action dependent einstein-hilbert}
	\L_{\text{E-H}}(g_{ab}, g_{ab,\mu}, g_{ab,\mu\nu}, z^\mu) = R\omega_g - \theta \wedge z. 
\end{equation}
Recall, \( z \) is the action flux which is a 3-form. Then if \( \theta \) is a 1-form
over spacetime, also known as a covector, the wedge \( \theta \wedge z \) is a 4-form, so
that \cref{eq:action dependent einstein-hilbert} makes sense. We call this a linear
dissipation term because it is linear in \( z \). It is also linear in \( \theta \) which
we will call the dissipation form. If we write \( \theta \) out in some coordinate system
then \( \theta = \theta_\mu \ud x^\mu \), and its components transform covariantly. In
natural units, \( \theta \) must have dimensions of \( \text{length}^{-1} \), since then
\( \theta \wedge z \) has dimensions of \( \text{action}/(\text{length})^4 \), i.e.
dimensions of action density. 

In coordinates 
\begin{equation*}
	\theta \wedge z = (\theta_\mu \ud x^\mu) \wedge (z^\nu \ud x_\nu) = \theta_\mu z^\nu \ud
	x^\mu \wedge \ud x_\nu = \theta_\mu z^\nu \delta^\mu_\nu \ud^4 x = \theta_\mu z^\mu
	\ud^4x
\end{equation*}
so \cref{eq:action dependent einstein-hilbert} becomes 
\begin{equation}\label{eq:action dependent EH coordinates 1}
	\L_{\text{E-H}}(g_{ab}, g_{ab,\mu}, g_{ab,\mu\nu}, z^\mu) = (R\sqrt{g} - \theta_\mu
	z^\mu) \ud^4 x. 
\end{equation}
This Lagrangian does not quite match the one proposed in eq. (9) of \cite{Lazo2017}. It
can be shown that in fact the Lagrangian we propose and the one they propose are actually
the same, just expressed in different coordinates. 

Consider a modified basis for the 3-forms, \( \omega_{g,\mu} = \sqrt{g}
\ud x_\mu \). If \( \zeta^\mu \) are the components of the action flux with respect to
this new basis, then
\begin{equation*}
	z = \zeta^\mu \omega_{g,\mu} = \zeta^\mu \sqrt{g} \ud x_\mu
\end{equation*}
which implies \( z^\mu = \sqrt{g} \zeta^\mu \). So, writing the components of the action
flux in the basis \( \omega_{g,\mu} \), \cref{eq:action dependent EH coordinates 1} looks like
\begin{equation}\label{eq:action dependent EH coordinates 2}
	\L_{\text{E-H}}(g_{ab}, g_{ab,\mu}, g_{ab,\mu\nu}, \zeta^\mu) = (R\sqrt{g} - \theta_\mu
	\zeta^\mu \sqrt{g}) \ud^4 x = (R - \theta_\mu \zeta^\mu)\omega_g.
\end{equation}
This is the Lagrangian proposed in equation (9) of \cite{Lazo2017}. 

We now write down the constraint in \cref{eq:constraint field theory} for this Lagrangian.
In the original coordinates for the action flux we have
\begin{equation*}
	\ud z = \partial_\mu z^\mu \ud^4 x
\end{equation*}
so, in coordinates
\begin{equation} \label{eq:constraint coordinates 1}
	\partial_\mu z^\mu = R\sqrt{g} - \theta_\mu z^\mu. 
\end{equation}
If instead we choose the other basis for the action flux, we see
\begin{equation*}
	\ud z = \partial_\mu (\sqrt{g} \zeta^\mu) \ud^4 x = \nabla_\mu \zeta^\mu \sqrt{g}\ud^4x
	= \nabla_\mu \zeta^\mu \omega_g
\end{equation*}
where \( \nabla \) is the covariant derivative induced by \( g \). We have made use of a
useful identity about the divergence:
\begin{equation} \label{eq:divergence identity}
	\nabla_\mu X^\mu = \frac{1}{\sqrt{g}} \partial_\mu (\sqrt{g}X^\mu).
\end{equation}

In the new coordinates the constraint looks like 
\begin{equation} \label{eq:constraint coordinates 2}
	\nabla_\mu \zeta^\mu = R - \theta_\mu \zeta^\mu
\end{equation}
which is the same form that appears in equation (8) of \cite{Lazo2017}. 

\section{Derivation of the field equations}
So we have seen what the Herglotz problem looks like for an Einstein-Hilbert Lagrangian
with a linear action dependence. We will apply the method of Lagrange multipliers, as
described in previous chapter, to derive a modified version of Einstein's equations. The
expanded Lagrangian is
\begin{equation*}
	\tilde{\L}_\text{E-H}(g_{ab}, g_{ab,\mu}, g_{ab,\mu\nu}, z^\nu, \partial_\mu z^\nu) =
	\big[(1-\lambda) \partial_\mu z^\mu + \lambda(R\sqrt{g} - \theta_\mu z^\mu)\big] \ud^4
	x.
\end{equation*}

\subsection{Variation of the action flux}
Let's compute the variation of the action given by this Lagrangian
\begin{align}
	\delta \tilde{S}[g_{ab}, z^\nu] & = \int_D \big[(1-\lambda) \delta \partial_\mu z^\mu +
	\lambda(\delta(R\sqrt{g}) - \theta_\mu \delta z^\mu)\big] \ud^4 x \notag \\
																	& = \int_D (1 - \lambda) \partial_\mu \delta z^\mu -
																	\lambda \theta_\mu \delta z^\mu \ud^4x + \int_D \lambda
																	\delta(R \sqrt{g}) \ud^4 x \notag \\
																	& = \int_{D} \partial_\mu \big((1 -
																	\lambda)\delta z^\mu\big) \ud^4 x + \int_D (\partial_\mu \lambda
																	- \lambda\theta_\mu) \delta z^\mu \ud^4 x + \int_D \lambda
																	\delta(R \sqrt{g}) \ud^4 x \label{eq:variation expanded
																	action}. 
\end{align}
The first integral is a boundary term coming from an integration by parts. It vanishes
because we assume the variations vanish at the boundary of \( D \). From the second
integral we can read off, using the fundamental theorem of the calculus of variations,
that
\begin{equation} \label{eq:action flux variation}
	\partial_\mu \lambda = \lambda\theta_\mu.
\end{equation}
Coordinate free this can also be written as \( \ud \lambda = \lambda \theta \). This has
an interesting implication for \( \theta \) since
\begin{equation*}
	\ud(\lambda \theta) = \ud \lambda \wedge \theta + \lambda \ud \theta = \lambda \theta
	\wedge \theta + \lambda \ud \theta = \lambda \ud \theta
\end{equation*}
and 
\begin{equation*}
	\lambda \ud \theta = \ud(\lambda \theta) = \ud^2 \lambda = 0
\end{equation*}
so we conclude \( \ud \theta = 0 \), i.e. \( \theta \) cannot be any 1-form, it must be a
a \emph{closed form}. This means that in coordinates \( \partial_\mu \theta_\nu =
\partial_\nu \theta_\mu \). 

\subsection{Variation of the metric}
We retake the calculation from \cref{eq:variation expanded action}. Since the integrals
involving \( z \) and \( g \) decouple, we can just consider the last term. We will follow the
derivation in \cite{Carroll1997} for as long as we can. Since \( R\sqrt{g} =
g^{ab}R_{ab}\sqrt{g} \), from the product rule its variation results in three terms:
\begin{equation}\label{eq:variation of scalar curvature}
	\int_D \lambda \delta(R \sqrt{g}) \ud^4 x = \int_D \lambda \delta g^{ab} R_{ab} \sqrt{g}
	\ud^4 x + \int_D \lambda g^{ab} \delta R_{ab} \sqrt{g} \ud^4 x + \int_{D} \lambda R
	\delta\sqrt{g} \ud^4 x
\end{equation}
The first term is already in the form required to apply the fundamental theorem of the
calculus of variations. For the third one uses the standard result
\begin{equation*}
	\delta \sqrt{g} = -\tfrac{1}{2}\sqrt{g} g_{ab} \delta g^{ab}.
\end{equation*}
The first and third terms of \cref{eq:variation of scalar curvature} can be combined into
\begin{equation}\label{eq:variation vacuum}
	\int_D \lambda (R_{ab} - \tfrac{1}{2} Rg_{ab}) \delta g^{ab} \sqrt{g} \ud^4 x.
\end{equation}
In the standard derivation of Einstein's equations, one shows that the middle integral of
\cref{eq:variation of scalar curvature} actually vanishes, so that if \cref{eq:variation
vacuum} is to vanish for any variation \( \delta g_{ab} \), or equivalently for any
variation of the inverse metric \( \delta g^{ab} \) the integrand itself must
vanish. This gives Einstein's equations. In the presence of \( \lambda \), however, the
middle integral doesn't vanish and actually contributes additional terms to the
equations.

Let's work out the variation of the middle integral of \cref{eq:variation of scalar
curvature}. We will do this step by step. The variation of the Ricci curvature can be shown to be
\begin{equation}\label{eq:variation ricci curvature}
	g^{ab} \delta R_{ab} = g^{ab}(\nabla_m \delta{\Gamma^m}_{ab} - \nabla_a \delta
	{\Gamma^m}_{mb}) = \nabla_n(g^{ab} {\delta\Gamma^n}_{ab} - g^{nb} \delta
	{\Gamma^m}_{mb})
\end{equation}
so
\begin{equation*}
	\int_D \lambda g^{ab}\delta R_{ab} \sqrt{g} \ud^4 x = \int_D \lambda \nabla_n(g^{ab}
	{\delta\Gamma^n}_{ab} - g^{nb} \delta {\Gamma^m}_{mb}) \sqrt{g} \ud^4x
\end{equation*}
and if \( \lambda \) weren't there this integral would vanish because of the divergence
theorem and the fact that the variations vanish on the boundary of \( D \). In the
presence of \( \lambda \) the standard trick is to perform integration by parts:
\begin{align*}
	& \int_D \lambda g^{ab}\delta R_{ab} \sqrt{g} \ud^4 x = \\
	& \quad = \int_D \lambda \nabla_n(g^{ab} {\delta\Gamma^n}_{ab} - g^{nb} \delta
	{\Gamma^m}_{mb}) \sqrt{g} \ud^4x \\
	& \quad = \int_D \nabla_n \left(\lambda (g^{ab} {\delta\Gamma^n}_{ab} - g^{nb}
	\delta {\Gamma^m}_{mb})\right) \sqrt{g}\ud^4 x - \int_D (\nabla_n \lambda) (g^{ab}
	{\delta\Gamma^n}_{ab} - g^{nb} \delta {\Gamma^m}_{mb}) \sqrt{g} \ud^4 x. 
\end{align*}
The first integral vanishes because it is the integral of a divergence and the variations
vanish on the boundary of \( D \). The second integral is where the additional terms will
come from. We split it into two terms. 

The variation of the Christoffel symbols can be shown to be
\begin{equation} \label{eq:variation christoffel symbols}
	\delta {\Gamma^a}_{bc} = \tfrac{1}{2} g^{am}(\nabla_c \delta g_{bm} + \nabla_b \delta
	g_{mc} - \nabla_m \delta g_{bc}).
\end{equation}
Using this and \cref{eq:action flux variation} (since \( \nabla_n \lambda = \partial_n
\lambda \)) we compute for the first integral
\begin{equation}\label{eq:first three terms}
	-\int_D (\nabla_n \lambda) g^{ab} \delta{\Gamma^n}_{ab} \sqrt{g} \ud^4 x = -
	\tfrac{1}{2} \int_D \lambda\theta_n g^{ab} g^{nk}(\nabla_b \delta g_{ak} + \nabla_a
	\delta g_{kb} - \nabla_k \delta g_{ab}) \sqrt{g} \ud^4 x. 
\end{equation}
The presence of \( g^{ab} \) means the indices \( a \) and \( b \) are symmetrised, so
\begin{equation*}
	g^{ab} \nabla_b \delta g_{ak} = g^{ab} \nabla_a \delta g_{kb}. 
\end{equation*}
This means \cref{eq:first three terms} simplifies to
\begin{align}
	&	-\int_D (\nabla_n \lambda) g^{ab} \delta{\Gamma^n}_{ab} \sqrt{g} \ud^4 x = \notag \\
	& \quad = - \int_D \lambda \theta_n g^{ab}g^{nk} \nabla_b \delta g_{ak} \sqrt{g} \ud^4 x
	+ \tfrac{1}{2} \int_D \lambda \theta_n g^{ab}g^{nk} \nabla_k \delta g_{ab} \sqrt{g}
	\ud^4 x \notag \\
	& \quad = - \int_D \lambda \theta_n \nabla_b (g^{ab}g^{nk} \delta g_{ak}) \sqrt{g} \ud^4 x
	+ \tfrac{1}{2} \int_D \lambda \theta_n \nabla_k (g^{ab}g^{nk}\delta g_{ab}) \sqrt{g}
	\ud^4 x. \label{eq:two integrals}
\end{align}
Let's perform an integration by parts for the first integral. We have to be a bit
careful. Introducing the shorthand \( X^{bn} = g^{ab}g^{nk}\delta g_{ak} \), we compute
\begin{equation*}
	\nabla_c(\lambda\theta_n X^{bn}) = \nabla_c(\lambda \theta_n)X^{bn} + \lambda \theta_n
	\nabla_c X^{bn}
\end{equation*}
so
\begin{align*}
	-\int_D \lambda \theta_n \nabla_b (g^{ab}g^{nk} \delta g_{ak})\sqrt{g}\ud^4x 
	& = - \int_D \lambda \theta_n \nabla_bX^{bn} \sqrt{g} \ud^4 x \\
	& = - \int_D \nabla_b(\lambda \theta_n X^{bn})\sqrt{g}\ud^4x + \int_{D}
	\nabla_b(\lambda\theta_n)X^{bn}\sqrt{g}\ud^4x. 
\end{align*}
The first integral is the integral of a divergence, so it vanishes. We are left with the
second which we can expand into
\begin{align*}
	\int_D \nabla_b(\lambda \theta_n) X^{bn}\sqrt{g}\ud^4 x 
	& = \int_D (\theta_n \partial_b \lambda + \lambda \nabla_b\theta_n)(g^{ab}g^{nk}\delta
	g_{ak})\sqrt{g} \ud^4 x \\
	& = \int_D \lambda(\theta_b\theta_n +
	\nabla_b\theta_n)(g^{ab}g^{nk}\delta g_{ak})\sqrt{g}\ud^4 x
\end{align*}
As a last step, we use the identity
\begin{equation*}
	\delta g^{ab} = - g^{am}g^{bn} \delta g_{mn}
\end{equation*}
to write our integral as a variation with respect to the inverse metric.
\begin{align*}
	\int_D \lambda(\theta_b\theta_n + \nabla_b\theta_n)(g^{ab}g^{nk}\delta
	g_{ak})\sqrt{g}\ud^4 x 
	& = -\int_D \lambda(\theta_b\theta_n + \nabla_b \theta_n)\delta g^{bn} \sqrt{g} \ud^4 x. 
\end{align*}

Without going through the details again, the other integral in \cref{eq:two integrals} can
be brought to the form
\begin{align*} 
	\tfrac{1}{2} \int_D \lambda \theta_n \nabla_k (g^{ab}g^{nk}\delta g_{ab}) \sqrt{g} \ud^4
	x 
	& = - \tfrac{1}{2} \int_D \nabla_k(\lambda \theta_n) g^{ab}g^{nk}\delta g_{ab} \sqrt{g}
	\ud^4 x \\
	& = \tfrac{1}{2} \int_D \lambda(\theta_k\theta_n + \nabla_k\theta_n) g^{ab}g^{nk}
	g_{ma}g_{lb}\delta g^{ml} \sqrt{g}\ud^4 x \\
	& = \tfrac{1}{2}\int_D \lambda g^{nk}(\theta_k\theta_n + \nabla_k \theta_n)g_{ml} \delta
	g^{ml} \sqrt{g} \ud^4 x
\end{align*}

There is still another integral we need to evaluate, the second term in the variation of
\( R_{ab} \), namely
\begin{align}
	\int_D (\partial_n \lambda) g^{nb}\delta{\Gamma^m}_{mb} \sqrt{g} \ud^4 x 
	& = \tfrac{1}{2} \int_D \lambda \theta_n g^{nb} g^{mk}(\nabla_b \delta g_{mk} + \nabla_m
	\delta g_{kb} - \nabla_k \delta g_{mb}) \sqrt{g} \ud^4 x. 
\end{align}
Because \( m \) and \( k \) are symmetrised, the second and third terms cancel, leaving
us with
\begin{align}
	\tfrac{1}{2}\int_D \lambda \theta_n g^{nb}g^{mk}\nabla_b \delta g_{mk} \sqrt{g} \ud^4 x
	& = - \tfrac{1}{2} \int_D \nabla_b(\lambda \theta_n) g^{nb}g^{mk} \delta g_{mk} \sqrt{g}
	\ud^4 x \\
	& = \tfrac{1}{2} \int_D \lambda(\theta_b\theta_n + \nabla_b
	\theta_n)g^{nb}g^{mk}g_{am}g_{lk}\delta g^{al} \sqrt{g} \ud^4 x \\
	& = \tfrac{1}{2} \int_D \lambda g^{nb}(\theta_b \theta_n + \nabla_b \theta_n)
	g_{al} \delta g^{al} \sqrt{g} \ud^4 x. 
\end{align}

We have calculated all the integrals we need. Before we put them all together, let us make
the following observation:
\begin{equation*}
	\nabla_a \theta_b = \partial_a \theta_b - {\Gamma^m}_{ab} \theta_m = \partial_b \theta_a
	- {\Gamma^m}_{ba} \theta_m = \nabla_b \theta_a
\end{equation*}
which uses the fact that \( \theta \) must be closed. Therefore we can define the
follwoing symmetric (0,2) tensor
\begin{equation} \label{eq:K}
	K_{ab} = \theta_a\theta_b + \nabla_{(a}\theta_{b)}. 
\end{equation}
So, after liberal relabeling of indices, we find that \cref{eq:variation expanded action}
becomes
\begin{equation} \label{eq:final variation of action}
	\delta\tilde{S}[g_{ab}, z^\mu] = \int_D(\partial_\mu \lambda - \lambda \theta_\mu)
	\delta z^\mu \ud^4 x + \int_D \lambda (R_{ab} - \tfrac{1}{2} Rg_{ab} - K_{ab} +
	Kg_{ab})\delta g^{ab} \sqrt{g} \ud^4 x
\end{equation}
with \( K_{ab} \) defined as in \cref{eq:K} and \( K = g^{mn}K_{mn} \) its trace. 

Applying the fundamental theorem of the calculus of variations, the action will be
stationary if and only if the integrands of both terms vanish. From the first integral we
get \cref{eq:action flux variation}, which we have already used. And from the second one
we get the modified Einstein field equations
\begin{equation} \label{eq:modified EFE}
	R_{ab} - \tfrac{1}{2}Rg_{ab} - K_{ab} + Kg_{ab} = 0.
\end{equation}

\end{document}
