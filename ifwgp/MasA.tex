
%%%%%%%%%%%%%%%%%%%%%%%%%%%%%%%%%%%%%%%%%%%%%%%%%%%%%%%%%%%%%%%%%%%%%%%%%%Abstract Template for short oral contributions and posters%%%%%%
% XXVIII INTERNATIONAL FALL WORKSHOP ON GEOMETRY AND PHYSICS%%
% September 2--6, 2019, Madrid (Spain)
%%%%%%%%%%%%%%%%%%%%%%%%%%%%%%%%%%%%%%%%%%%%%%%%%%%%%%%%%%%%%%%%%%%%%%%%%%%%%%%%%%%%%%%%%%%%%%%%%%%%%%%%%%%%%%%%%%%%%%%%%%%%%%%%%%%%%%%%%%%%%%%%

%%%%% INSTRUCTIONS (PLEASE, READ!!) %%%%%

% The conference abstract will be prepared in
% LaTeX2e.  Please do not send AMS-TeX, plain TeX,
% or any other dialect of TeX.
% Do not use any additional packages, define any macros,
% or make any alterations to lines below that finish
% with the characters %*
%
% To aid processing of the abstracts, please include your
% surname followed by the initial of your
% first name in the name of the file you submit as your
% abstract. For example, Mike Adams would submit
%
% AdamsM.tex
%


\documentclass[11pt]{article}
\usepackage{amsfonts}
\usepackage{amssymb}
\usepackage{amsmath}
\usepackage{amsthm}

%%%%%%%%%%%%%%%%%%%%%%%%%%%%%%%%%%%%%%%%%%%%%%%%%
%             SIZE             %
%%%%%%%%%%%%%%%%%%%%%%%%%%%%%%%%%%%%%%%%%%%%%%%%%


\parskip=0.5ex
\oddsidemargin= 0.35cm
\evensidemargin= 0.35cm

\parindent=1.5em
\textheight=23.0cm
\textwidth=15.5cm


%%%%%%%%%%%%%%%%%%%%%%%%%%%%%%%%%%%%%%%%%%%%%%%%%
%%%%%%%%%%%%%%%%%%%%%%%%%%%%%%%%%%%%%%%%%%%%%%%%%

\begin{document}

\thispagestyle{empty}

\

\begin{center}
 {\Large{\bf {Herglotz's Variational Principle for an action dependent Einstein-Hilbert Lagrangian}} }

\bigskip
\bigskip


{\sc \underline{Arnau Mas}$^a$ and Jordi Gaset$^b$}
% Please underline the author that presents the contribution

\bigskip

{$^a$ Universitat Aut\`onoma de Barcelona\\ ~~E-mail: arnau.mas@outlook.com\\[10pt]
$^b$ Universitat Aut\`onoma de Barcelona\\ ~~E-mail: jordi.gaset@uab.cat
}

\end{center}


\section*{Abstract}
We describe a suitable differential geometric setting in which to formulate an action dependent version of the Einstein-Hilbert Lagrangian. We derive the Herglotz equations following variational methods. The result are a modified version of Einstein's equations. This is an interesting application of Herglotz's variational principle to a singular, second-order field theory. 



\begin{thebibliography}{5}
% Maximum three references.
% Please use the reference format given here.

\bibitem{ref1}
J. Gaset et al.  {\em A \( k \)-contact Lagrangian formulation for
nonconservative field theories}. arXiv:2002.10458. 2020.

\bibitem{ref2} M. Lazo et al. {\em An action principle for action-dependent Lagrangians}.
Journal of Mathematical Physics. Vol. 59, No. 3 (2018).

\end{thebibliography}


\end{document}
