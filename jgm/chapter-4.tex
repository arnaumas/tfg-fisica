\documentclass[../main.tex]{subfiles}

\begin{document}
In this section we discuss the equations we have obtained and how they compare to those
appearing in existing publications. We also make the case that the version we have derived
is a more adequate version. 

\subsection{The dissipation tensor}
Let us recap what we did in the previous section. We have shown, by computing the
variation of the corresponding action, that the field equations of an Einstein-Hilbert
Lagrangian with linear dissipation, namely
\begin{equation} 
	L(g_{ab}, \partial_\mu g_{ab}, \partial_\mu \partial_\nu g_{ab}, z^\mu) = R(g_{ab}, \partial_\mu g_{ab}, \partial_\mu \partial_\nu g_{ab})
	\sqrt{g} - \theta_\mu z^\mu,
\end{equation}
are
\begin{equation}\label{eq:EFE with dissipation}
   R_{ab} - \tfrac{1}{2}Rg_{ab} - K_{ab} + Kg_{ab} = 0,
\end{equation}
where \( K_{ab} \) are the components of the \( (0,2) \) symmetric tensor
\begin{equation}
	\mathbf{K} = \nabla \theta + \theta \otimes \theta.
\end{equation}
We will call \( K \) the dissipation tensor. Since the first two terms of \cref{eq:EFE with dissipation} have zero divergence (they are the components of the Einstein tensor), it must be the case that, on-shell, the divergence of the second two terms also vanishes. This imposes a constraint on the space of solutions to \cref{eq:modified EFE}, which depends on the dissipation form $\theta$. Namely,
\begin{equation}
\nabla_a\left(g^{ab}K_{bc}-n\delta^a_cK\right)=0
\end{equation}

This means, that if we add a matter term to the Lagrangian, the resulting  tensor in the field equations may not be conserved.

This means, that if we were to couple a matter term to \cref{eq:action dependent einstein-hilbert}, its energy-momentum tensor need not in general have zero divergence. Specifically, what must have zero divergence will be a combination of the energy-momentum tensors of the matter fields and terms containing the dissipation 1-form. Nevertheless, further investigation is required to determine the precise way in which the dissipation tensor governs the non-conservation of other quantities.

\subsection{Non-covariance of existing equations}
These equations are not the ones obtained in \cite{Lazo2017}. For the same Lagrangian, the
equations derived are
\begin{equation} \label{eq:EFE lazo}
	R_{ab} + \tilde{K}_{ab} - \tfrac{1}{2}g_{ab}(R + \tilde{K}) = 0,
\end{equation}
where $\tilde{K}=g^{ab}\tilde{K}_{ab}$, and $\tilde{K}_{ab}$ is 
\begin{equation} \label{eq:}
	\tilde{K}_{ab} = \theta_m {\Gamma^m}_{ab} - \tfrac{1}{2}\left(\theta_a {\Gamma^m}_{mb} +
	\theta_b {\Gamma^m}_{am}\right). 
\end{equation}
These cannot possibly represent the components of a tensor. Very explicitly,  for the flat Minkowski metric, their
expression in Cartesian coordinates is 0. If they represented the components of a tensor then they would also vanish for any other choice of coordinates for the flat metric. Nevertheless, in spherical coordinates one computes
\begin{align*}
	{\Gamma^r}_{\theta\theta} & = -r & {\Gamma^\theta}_{r\theta} & = \frac{1}{r} &
	{\Gamma^\phi}_{r\phi} & = \frac{1}{r} \\
	{\Gamma^r}_{\phi\phi} & = -r \sin{\theta}^2 & {\Gamma^\theta}_{\phi\phi} & =
	-\sin{\theta}\cos{\theta} & {\Gamma^\phi}_{\theta\phi} & = \frac{1}{\tan{\theta}}. 
\end{align*}
Which means, for example,
\begin{equation*}
	\tilde{K}_{tr} = 0 - \tfrac{1}{2}(\theta_t {\Gamma^m}_{mr} + 0) =
	-\frac{\theta_t}{2r},
\end{equation*}
which is certainly non-zero if \( \theta_t \) does not vanish. Hence the object derived in \cite{Lazo2017} is not coordinate independent so it
cannot possibly represent meaningful physics. 

There is another fact that points to the equations in \cite{Lazo2017} not being what one would expect as the Herglotz equations coming from a second-order action-dependent Lagrangian. The Herglotz equations for the harmonic oscillator with
linear dissipation lead to equations linear in the dissipation coefficient (\cite{Gaset2020b}). However, the Lagrangian for this system is first
order, whereas, as we had already discussed, the Einstein-Hilbert Lagrangian is actually
second order. There is a second order Lagrangian with linear
dissipation, called the damped Pais-Uhlenbeck oscillator, whose equations of motion are derived in \cite{Leon2021a}. These are in fact not linear in the dissipation coefficient, but rather quadratic. In
our case, the dissipation form plays the role of the dissipation coefficient, and indeed, \(\mathbf{K}\) is quadratic in it. The equations in \cite{Lazo2017} instead lack a quadratic term.

One can pinpoint the exact reason for the problems with \cref{eq:EFE lazo}. One of the
simplifying assumptions made in their derivation was to only consider certain terms of the Ricci curvature.
Specifically, the Ricci curvature consists of four terms. Two of them are contractions of
the Christoffel symbols with themselves, the other two are derivatives of the Christoffel
symbols. In the classical case, without dissipation, one can show that the second two terms are actually a divergence, so they do not contribute to the variation of the Einstein-Hilbert action and the resulting equations remain unchanged (see \cite{Gaset2018,Maria2015}). For an action-dependent theory, however, adding a divergence to the unexpanded Lagrangian does not lead, in general, to the same equations \cite{inverse,Lazo2021}. 


\end{document}
