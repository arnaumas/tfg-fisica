\documentclass[../main.tex]{subfiles}

\begin{document}

In this chapter we apply the language and tools developed in the previous chapter to the
specific case of Einstein gravity. We first describe the Lagrangian from which the Einstein
field equations come and then introduce an action-dependent version of it and
derive its field equations. 

\section{The Einstein-Hilbert Lagrangian}
As is well-known, the Einstein field equations can be obtained from a variational
principle. The classical Lagrangian that gives rise to these equations is the Einstein-Hilbert
Lagrangian. We formulate it in the language of \cref{sec:lagrangian densities}. The field of interest in relativity is the metric on the given spacetime \(M\) ---which we take to be of dimension 4 and orientable from now on---. Hence the bundle of interest to us is a subbundle of the second symmetric power of the cotangent bundle of \( M\), 
\begin{equation}
    S^2 T^\ast M \to M.
\end{equation}
Specifically it is the subbundle determined by the condition of non-degeneracy. We denote it by \( G(M) \to M \).  

As advertised, the theory of general relativity is a second-order theory, which means that the Einstein-Hilbert Lagrangian must be a bundle map from \( J^2 G(M)\) to \(\bigwedge^n T^\ast M \). Specifically, given a metric \( g \in \Gamma(G(M)) \),
\begin{equation} \label{eq:EH lagrangian}
     \L_\text{E-H} \circ j^2 g = R(g) \omega_g.
\end{equation}
Here we use \( \omega_ g \) to denote the volume form determined by \(g\), which in a choice of coordinates  \(x^\mu\) becomes
\begin{equation}
    \sqrt{g} \d^4 x,
\end{equation}
where \( \sqrt{g} \) is the square root of the absolute value of the determinant of the expression of the metric in the coordinates \(x^\mu\). 
The other factor, \(R(g)\), is the scalar curvature of \(g\), which is defined as the trace of the Ricci tensor:
\begin{equation}
    R(g) = \tr(g^{-1} \Ric(g)).
\end{equation}
The Ricci tensor is of type \((0,2) \), so by contracting with \(g^{-1}\), the metric induced on \(T^\ast M\), we obtain a \((1,1)\) tensor, whose trace is well-defined. The coordinate expression of the components of the Ricci tensor, \(R_{ab}\), is
\begin{equation} \label{eq:ricci tensor}
	R_{ab} = \partial_m{\Gamma^m}_{ab} - \partial_a {\Gamma^m}_{mb} + {\Gamma^m}_{mn}
	{\Gamma^n}_{ab} - {\Gamma^m}_{an}{\Gamma^n}_{mb},
\end{equation}
where \( \Gamma_{ab}^c \) are the Christoffel symbols of the Levi-Civita connection determined by \(g\). These contain first derivatives of the metric, so \( R_{ab} \) and
hence \( R \) contain second derivatives of the metric, and the Einstein-Hilbert Lagrangian is indeed of second order.

The Einstein-Hilbert action is therefore
\begin{equation} \label{eq:EH action}
	S_{\text{E-H}}(g) = \int_D \L_\text{E-H} \circ j^2g = \int_D R
	\sqrt{g}\ud^4x,
\end{equation}
for some domain \(D\) on which the integral is finite. 
A variation of this action leads one to the Einstein field equations 
\begin{equation} \label{eq:EFE vacuum}
	R_{ab} - \tfrac{1}{2}g_{ab}R = 0. 
\end{equation}
More precisely, these are the Einstein field equations in a vacuum, since one can add
various matter terms to the Einstein-Hilbert Lagrangian which lead to the Einstein
equations in the presence of matter,
\begin{equation} \label{eq:EFE matter}
	R_{ab} - \tfrac{1}{2}g_{ab}R = T_{ab}. 
\end{equation}
The object \( T_{ab} \) is the energy-momentum tensor and collects all of the terms coming
from the presence of matter. See \S4 of \cite{Carroll1997} for a detailed derivation.

\section{An action dependent Einstein-Hilbert Lagrangian}
What kind of action dependence can we incorporate into the Einstein-Hilbert Lagrangian?
The simplest one is a linear dissipation term:
\begin{equation} \label{eq:action dependent einstein-hilbert}
	\L_{\text{E-H}} \circ (j^2 g,z) = R\omega_g - \theta \wedge z. 
\end{equation}
Now \( \L_{\text{E-H}} \) is defined on the expanded bundle, \( G(M) \oplus \bigwedge^3T^\ast M \to M \). Hence \( \theta \) must be a 1-form on \(M\), which we will refer to as the \emph{dissipation form}.

The coordinate expression of this dissipation term is
\begin{equation*}
	\theta \wedge z = (\theta_\mu \ud x^\mu) \wedge \left(z^\nu \frac{\partial}{\partial x^\nu}\lrcorner \ud^4 x\right) = \theta_\mu z^\nu \ud
	x^\mu \wedge \left(\frac{\partial}{\partial x^\nu}\lrcorner \ud^4 x\right) = \theta_\mu z^\mu \ud^4x
\end{equation*}
where once again we use the coordinates defined in \cref{eq:coordinates flux}. Then \cref{eq:action dependent einstein-hilbert} becomes 
\begin{equation}\label{eq:action dependent EH coordinates 1}
	\L_{\text{E-H}} \circ (j^2g, z) = (R\sqrt{g} - \theta_\mu
	z^\mu) \ud^4 x. 
\end{equation}
This Lagrangian does not exactly match the one proposed in Equation (9) of \cite{Lazo2017}. The discrepancy is down to a different choice of coordinates. Indeed, in the previous computation we used the isomorphism between \(\Omega^3(M)\) and \(\Gamma(TM)\) induced by contracting with \(\d^4x\). Instead we may contract with the volume form induced by the metric, \(\omega_g\). Let \(\zeta^\mu\) be the components of \(z\) in this new choice of coordinates, i.e.
\begin{equation*}
	z = \zeta^\mu \frac{\partial}{\partial x^\mu} \lrcorner \omega_{g} = \zeta^\mu \sqrt{g} \frac{\partial}{\partial x^\mu} \lrcorner \ud^4x,
\end{equation*}
which implies \( z^\mu = \sqrt{g} \zeta^\mu \). In these new coordinates \cref{eq:action dependent EH coordinates 1} looks like
\begin{equation}\label{eq:action dependent EH coordinates 2}
	\L_{\text{E-H}} \circ (j^2g, z) = (R\sqrt{g} - \theta_\mu
	\zeta^\mu \sqrt{g}) \ud^4 x = (R - \theta_\mu \zeta^\mu)\omega_g.
\end{equation}
This is the Lagrangian proposed in Equation (9) of \cite{Lazo2017}. 

We now write down the constraint in \cref{eq:constraint field theory} for this Lagrangian.
In the original coordinates for the action flux we have
\begin{equation*}
	\ud z = \partial_\mu z^\mu \ud^4 x,
\end{equation*}
so
\begin{equation} \label{eq:constraint coordinates 1}
	\partial_\mu z^\mu = R\sqrt{g} - \theta_\mu z^\mu. 
\end{equation}
In the other set of coordinates, induced by contracting with \(\omega_g\), one sees
\begin{equation*}
	\ud z = \partial_\mu (\sqrt{g} \zeta^\mu) \ud^4 x = \nabla_\mu \zeta^\mu \sqrt{g}\ud^4x
	= \nabla_\mu \zeta^\mu \omega_g,
\end{equation*}
where \( \nabla \) is the covariant derivative induced by \( g \). We have made use of a
useful identity about the divergence:
\begin{equation} \label{eq:divergence identity}
	\nabla_\mu X^\mu = \frac{1}{\sqrt{g}} \partial_\mu (\sqrt{g}X^\mu).
\end{equation}
This is the statement that the divergence induced by the volume form of a metric coincides with the trace of the covariant derivative. 

In the new coordinates the constraint becomes
\begin{equation} \label{eq:constraint coordinates 2}
	\nabla_\mu \zeta^\mu = R - \theta_\mu \zeta^\mu,
\end{equation}
which is the same form that appears in Equation (8) of \cite{Lazo2017}. 

\section{Derivation of the field equations}
We now apply the method of Lagrange multipliers, as
described in previous chapter, to derive a modified version of Einstein's equations. The
expanded Lagrangian is
\begin{equation*}
\tilde{\L}_\text{E-H} \circ (j^2 g, j^1 z) = \big[(1-\lambda) \partial_\mu z^\mu + \lambda(R\sqrt{g} - \theta_\mu z^\mu)\big] \ud^4
	x.
\end{equation*}
We will compute the variation of the corresponding expanded action, \(\Tilde{S}(g,z) = \int_D \Tilde{L}_\text{E-H} \circ (j^2g,j^1z)\), with respect to the two dynamical degrees of freedom, \(z\) and \(g\).

\subsection{Variation of the action flux}\label{sec:variationonaction}
The variation with respect to the action flux is
\begin{align}
	\delta \tilde{S}(g, z) & = \int_D \big[(1-\lambda) \delta \partial_\mu z^\mu +
\lambda(\delta(R\sqrt{g}) - \theta_\mu \delta z^\mu)\big] \ud^4 x \notag \\						& = \int_D (1 - \lambda) \partial_\mu \delta z^\mu -
			\lambda \theta_\mu \delta z^\mu \ud^4x + \int_D \lambda
			\delta(R \sqrt{g}) \ud^4 x \notag \\
& = \int_{D} \partial_\mu \big((1 -
		\lambda)\delta z^\mu\big) \ud^4 x + \int_D (\partial_\mu \lambda
		- \lambda\theta_\mu) \delta z^\mu \ud^4 x + \int_D \lambda
		\delta(R \sqrt{g}) \ud^4 x \label{eq:variation expanded
										action}. 
\end{align}
The first integral is a boundary term coming from an integration by parts. It vanishes
if we assume the variations vanish at the boundary of \( D \). If the action is stationary then its variation must vanish for any variation of the fields. This means that the second term of \cref{eq:variation expanded action} must vanish, since in particular we may choose not to vary the metric. Hence, the quantity inside the brackets must vanish, since it vanishes when integrated against any variation. Therefore
\begin{equation} \label{eq:action flux variation}
	\partial_\mu \lambda = \lambda\theta_\mu.
\end{equation}
In other words, \( \ud \lambda = \lambda \theta \). As we had advertised before, this forces the dissipation form \(\theta\) to be closed, as
\begin{equation*}
	\ud(\lambda \theta) = \ud \lambda \wedge \theta + \lambda \ud \theta = \lambda \theta
	\wedge \theta + \lambda \ud \theta = \lambda \ud \theta,
\end{equation*}
hence
\begin{equation*}
	\lambda \ud \theta = \ud(\lambda \theta) = \ud^2 \lambda = 0,
\end{equation*}
and we conclude \( \ud \theta = 0 \) provided \(\lambda\) does not vanish. 

\subsection{Variation of the metric}
We retake the calculation from \cref{eq:variation expanded action}. We may now only consider the last integral, as we can vary \(g\) and \(z\) independently. We will follow the
derivation in \cite{Carroll1997} for as long as we can. In particular, we take the spacetime \(M\) to be closed, and hence avoid consideration of Gibbons-Hawking-York type boundary terms. Since \( R\sqrt{g} =
g^{ab}R_{ab}\sqrt{g} \), from the product rule its variation results in three terms:
\begin{equation}\label{eq:variation of scalar curvature}
	\int_D \lambda \delta(R \sqrt{g}) \ud^4 x = \int_D \lambda \delta g^{ab} R_{ab} \sqrt{g}
	\ud^4 x + \int_D \lambda g^{ab} \delta R_{ab} \sqrt{g} \ud^4 x + \int_{D} \lambda R
	\delta\sqrt{g} \ud^4 x
\end{equation}
The first term is already in the form required to apply the fundamental theorem of the
calculus of variations. For the third one uses the standard result
\begin{equation*}
	\delta \sqrt{g} = -\tfrac{1}{2}\sqrt{g} g_{ab} \delta g^{ab}.
\end{equation*}
The first and third terms of \cref{eq:variation of scalar curvature} can be combined into
\begin{equation}\label{eq:variation vacuum}
	\int_D \lambda (R_{ab} - \tfrac{1}{2} Rg_{ab}) \delta g^{ab} \sqrt{g} \ud^4 x.
\end{equation}
In the standard derivation of Einstein's equations, one shows that the middle integral of
\cref{eq:variation of scalar curvature} actually vanishes, so that if \cref{eq:variation
vacuum} is to vanish for any variation \( \delta g_{ab} \), or equivalently for any
variation of the inverse metric \( \delta g^{ab} \), the integrand of \cref{eq:variation vacuum} itself must
vanish. This gives Einstein's equations. In the presence of \( \lambda \), however, the
middle integral does not vanish and contributes additional terms to the
equations.

We compute the variation of the middle integral in \cref{eq:variation of scalar curvature}.  The variation of the Ricci curvature can be shown to be
\begin{equation}\label{eq:variation ricci curvature}
	g^{ab} \delta R_{ab} = g^{ab}(\nabla_m \delta{\Gamma^m}_{ab} - \nabla_a \delta
	{\Gamma^m}_{mb}) = \nabla_n(g^{ab} {\delta\Gamma^n}_{ab} - g^{nb} \delta
	{\Gamma^m}_{mb})\,,
\end{equation}
so
\begin{equation*}
	\int_D \lambda g^{ab}\delta R_{ab} \sqrt{g} \ud^4 x = \int_D \lambda \nabla_n(g^{ab}
	{\delta\Gamma^n}_{ab} - g^{nb} \delta {\Gamma^m}_{mb}) \sqrt{g} \ud^4x,
\end{equation*}
and if \( \lambda \) weren't there this integral would vanish because of the divergence
theorem and the fact that the variations vanish on the boundary of \( D \). In the
presence of \( \lambda \) we perform an integration by parts:
\begin{align*}
	& \int_D \lambda g^{ab}\delta R_{ab} \sqrt{g} \ud^4 x = \\
	& \quad = \int_D \lambda \nabla_n(g^{ab} {\delta\Gamma^n}_{ab} - g^{nb} \delta
	{\Gamma^m}_{mb}) \sqrt{g} \ud^4x \\
	& \quad = \int_D \nabla_n \left(\lambda (g^{ab} {\delta\Gamma^n}_{ab} - g^{nb}
	\delta {\Gamma^m}_{mb})\right) \sqrt{g}\ud^4 x - \int_D (\nabla_n \lambda) (g^{ab}
	{\delta\Gamma^n}_{ab} - g^{nb} \delta {\Gamma^m}_{mb}) \sqrt{g} \ud^4 x. 
\end{align*}
The first integral vanishes because it is the integral of a divergence and the variations
vanish on the boundary of \( D \). The second integral is where the additional terms will
come from. We split it into two terms. 

The variation of the Christoffel symbols can be shown to be
\begin{equation} \label{eq:variation christoffel symbols}
	\delta {\Gamma^a}_{bc} = \tfrac{1}{2} g^{am}(\nabla_c \delta g_{bm} + \nabla_b \delta
	g_{mc} - \nabla_m \delta g_{bc}).
\end{equation}
Using this and \cref{eq:action flux variation} (since \( \nabla_n \lambda = \partial_n
\lambda \)) we compute for the first integral
\begin{equation}\label{eq:first three terms}
	-\int_D (\nabla_n \lambda) g^{ab} \delta{\Gamma^n}_{ab} \sqrt{g} \ud^4 x = -
	\tfrac{1}{2} \int_D \lambda\theta_n g^{ab} g^{nk}(\nabla_b \delta g_{ak} + \nabla_a
	\delta g_{kb} - \nabla_k \delta g_{ab}) \sqrt{g} \ud^4 x. 
\end{equation}
The presence of \( g^{ab} \) means the indices \( a \) and \( b \) are symmetrised, so
\begin{equation*}
	g^{ab} \nabla_b \delta g_{ak} = g^{ab} \nabla_a \delta g_{kb}. 
\end{equation*}
This means \cref{eq:first three terms} simplifies to
\begin{align}
	&	-\int_D (\nabla_n \lambda) g^{ab} \delta{\Gamma^n}_{ab} \sqrt{g} \ud^4 x = \notag \\
	& \quad = - \int_D \lambda \theta_n g^{ab}g^{nk} \nabla_b \delta g_{ak} \sqrt{g} \ud^4 x
	+ \tfrac{1}{2} \int_D \lambda \theta_n g^{ab}g^{nk} \nabla_k \delta g_{ab} \sqrt{g}
	\ud^4 x \notag \\
	& \quad = - \int_D \lambda \theta_n \nabla_b (g^{ab}g^{nk} \delta g_{ak}) \sqrt{g} \ud^4 x
	+ \tfrac{1}{2} \int_D \lambda \theta_n \nabla_k (g^{ab}g^{nk}\delta g_{ab}) \sqrt{g}
	\ud^4 x. \label{eq:two integrals}
\end{align}
Let's perform an integration by parts for the first integral. Introducing the shorthand \( X^{bn} = g^{ab}g^{nk}\delta g_{ak} \), we compute
\begin{equation*}
	\nabla_c(\lambda\theta_n X^{bn}) = \nabla_c(\lambda \theta_n)X^{bn} + \lambda \theta_n
	\nabla_c X^{bn},
\end{equation*}
so
\begin{align*}
	-\int_D \lambda \theta_n \nabla_b (g^{ab}g^{nk} \delta g_{ak})\sqrt{g}\ud^4x 
	& = - \int_D \lambda \theta_n \nabla_bX^{bn} \sqrt{g} \ud^4 x \\
	& = - \int_D \nabla_b(\lambda \theta_n X^{bn})\sqrt{g}\ud^4x + \int_{D}
	\nabla_b(\lambda\theta_n)X^{bn}\sqrt{g}\ud^4x. 
\end{align*}
The first integral is the integral of a divergence, so it vanishes. We are left with the
second which we can expand into
\begin{align*}
	\int_D \nabla_b(\lambda \theta_n) X^{bn}\sqrt{g}\ud^4 x 
	& = \int_D (\theta_n \partial_b \lambda + \lambda \nabla_b\theta_n)(g^{ab}g^{nk}\delta
	g_{ak})\sqrt{g} \ud^4 x \\
	& = \int_D \lambda(\theta_b\theta_n +
	\nabla_b\theta_n)(g^{ab}g^{nk}\delta g_{ak})\sqrt{g}\ud^4 x.
\end{align*}
As a last step, we use the identity
\begin{equation*}
	\delta g^{ab} = - g^{am}g^{bn} \delta g_{mn}
\end{equation*}
to write our integral as a variation with respect to the inverse metric.
\begin{align*}
	\int_D \lambda(\theta_b\theta_n + \nabla_b\theta_n)(g^{ab}g^{nk}\delta
	g_{ak})\sqrt{g}\ud^4 x 
	& = -\int_D \lambda(\theta_b\theta_n + \nabla_b \theta_n)\delta g^{bn} \sqrt{g} \ud^4 x. 
\end{align*}

Without going through the details again, the other integral in \cref{eq:two integrals} can
be brought to the form
\begin{align*} 
	\tfrac{1}{2} \int_D \lambda \theta_n \nabla_k (g^{ab}g^{nk}\delta g_{ab}) \sqrt{g} \ud^4
	x 
	& = - \tfrac{1}{2} \int_D \nabla_k(\lambda \theta_n) g^{ab}g^{nk}\delta g_{ab} \sqrt{g}
	\ud^4 x \\
	& = \tfrac{1}{2} \int_D \lambda(\theta_k\theta_n + \nabla_k\theta_n) g^{ab}g^{nk}
	g_{ma}g_{lb}\delta g^{ml} \sqrt{g}\ud^4 x \\
	& = \tfrac{1}{2}\int_D \lambda g^{nk}(\theta_k\theta_n + \nabla_k \theta_n)g_{ml} \delta
	g^{ml} \sqrt{g} \ud^4 x.
\end{align*}

There is still another integral we need to evaluate, the second term in the variation of
\( R_{ab} \), namely
\begin{align}
	\int_D (\partial_n \lambda) g^{nb}\delta{\Gamma^m}_{mb} \sqrt{g} \ud^4 x 
	& = \tfrac{1}{2} \int_D \lambda \theta_n g^{nb} g^{mk}(\nabla_b \delta g_{mk} + \nabla_m
	\delta g_{kb} - \nabla_k \delta g_{mb}) \sqrt{g} \ud^4 x. 
\end{align}
Because \( m \) and \( k \) are symmetrised, the second and third terms cancel, leaving
us with
\begin{align}
	\tfrac{1}{2}\int_D \lambda \theta_n g^{nb}g^{mk}\nabla_b \delta g_{mk} \sqrt{g} \ud^4 x
	& = - \tfrac{1}{2} \int_D \nabla_b(\lambda \theta_n) g^{nb}g^{mk} \delta g_{mk} \sqrt{g}
	\ud^4 x \\
	& = \tfrac{1}{2} \int_D \lambda(\theta_b\theta_n + \nabla_b
	\theta_n)g^{nb}g^{mk}g_{am}g_{lk}\delta g^{al} \sqrt{g} \ud^4 x \\
	& = \tfrac{1}{2} \int_D \lambda g^{nb}(\theta_b \theta_n + \nabla_b \theta_n)
	g_{al} \delta g^{al} \sqrt{g} \ud^4 x. 
\end{align}

We have calculated all the integrals we need. Before we put them all together, let us make
the following observation:
\begin{equation*}
	\nabla_a \theta_b = \partial_a \theta_b - {\Gamma^m}_{ab} \theta_m = \partial_b \theta_a
	- {\Gamma^m}_{ba} \theta_m = \nabla_b \theta_a,
\end{equation*}
which uses the fact that \( \theta \) must be closed. We may therefore define the following (0,2) symmetric tensor
\begin{equation} \label{eq:K}
\mathbf{K} = \theta \otimes \theta + \nabla \theta,
\end{equation}
whose components are
\begin{equation}
	K_{ab} = \theta_a\theta_b + \frac12\left(\nabla_{a}\theta_{b}+\nabla_{b}\theta_{a}\right) = \theta_a\theta_b + \nabla_{(a}\theta_{b)}=\theta_a\theta_b + \nabla_{a}\theta_{b},
\end{equation}
where parentheses surrounding indices indicate symmetrisation. 
All three expressions are equal because $\nabla_a \theta_b= \nabla_b \theta_a$. Nevertheless, we will use the second one to make the symmetry of the indices explicit. So, after liberal relabeling of indices, we find that \cref{eq:variation expanded action}
becomes
\begin{equation} \label{eq:final variation of action}
	\delta\tilde{S}[g_{ab}, z^\mu] = \int_D(\partial_\mu \lambda - \lambda \theta_\mu)
	\delta z^\mu \ud^4 x + \int_D \lambda (R_{ab} - \tfrac{1}{2} Rg_{ab} - K_{ab} +
	Kg_{ab})\delta g^{ab} \sqrt{g} \ud^4 x,
\end{equation}
with \( K_{ab} \) defined as in \cref{eq:K} and \( K = g^{mn}K_{mn} \) its trace. 

Applying the fundamental theorem of the calculus of variations, the action will be
stationary if and only if the integrands of both terms vanish. From the first integral we
get \cref{eq:action flux variation}, which we have already used. And from the second one
we get the modified Einstein field equations
\begin{equation} \label{eq:modified EFE}
	R_{ab} - \tfrac{1}{2}Rg_{ab} - K_{ab} + Kg_{ab} = 0.
\end{equation}

These equations coincide with the ones derived in \cite{Lazo2021}.


\end{document}
