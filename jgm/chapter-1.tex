\documentclass[../main.tex]{subfiles}

\begin{document}

Most of the physical systems of interest in physics admit a description in terms of a
variational principle: physical solutions are the extrema of some functional defined on
the state of all possible evolutions of the system, called the action. If the action is
defined in terms of the integral of a Lagrangian then its extrema are precisely the
solutions to the Euler-Lagrange equations of the Lagrangian. For mechanical systems, i.e.
systems in which solutions are functions only of time, the phase space can be equipped
with a symplectic structure. From this point of view, time evolution is just the flow
generated by the Hamiltonian of the system, and one has at one's disposal all of the tools
known from symplectic mechanics: Poisson bracket, Noether's theorem, etc. The geometric
framework can be generalized to field theories using, for instance, the multisymplectic
formalism.

This description, nevertheless, excludes a large class of systems, namely dissipative
systems. In some cases the phase space for such systems is naturally equipped with a
contact structure, which can in many ways be seen as the odd dimensional analogue of a
symplectic structure (see \cite{grabowska_geometric_2022} for a more general discussion).
What in the symplectic world were conservation laws, now become dissipation laws
\cite{Gaset2020b}. Contact structures have made appearances in various fields in recent
years: reversible and non-reversible thermodynamics
\cite{Bravetti2019,Mrugala1991,Simoes2020}, quantum mechanics \cite{ciaglia_contact_2018},
statistical mechanics \cite{goto_contact_2016}, cosmology \cite{Lazo2017,Sloan} or
electromagnetism \cite{GasetMarin}. The contact framework is well understood for
mechanical systems, see
\cite{Gaset2020b,geiges_introduction_2008,Lainz2019,Leon2021,Leon2019,Leon2021a,cocontact}.
The field theory analog, multicontact geometry is under current development, see
\cite{Gaset2020, Gaset2020a,Georgieva, de_leon_multicontact_2022} for recent efforts. One
of the main challenges is a successful understanding of singular and higher-order
theories.

In this work we study one such theory, namely a dissipative version of Einstein gravity.
To circumvent the difficulties coming from the as of now not fully understood contact
formalism, we make use of the fact that contact systems, when seen from the Lagrangian
point of view, can also be formulated in terms of a variational principle, the Herglotz
variational principle. The Lagrangians that fit in this framework are called
\emph{action-dependent}. We apply variational calculus to derive the analog of the
Euler-Lagrange equations, the Herglotz equations, for this system. This is of relevance
since examples of singular or higher-order dissipative field theories are scarce in the
literature. 

The result obtained is also relevant to the study of modifications of Einstein's theory of
gravity, that would explain some observations of cosmological phenomena that do not fit
within the current picture, as well as open avenues towards the successful quantisation of
gravity. A survey of theories of this kind is in \cite{Olmo2020} and in
\cite{baker_strong_2017}. In \cite{Lazo2017,Lazo2021}, the same Lagrangian we introduce
was studied. We frame it within the broader context of the Herglotz variational principle
and dissipative theories. In this sense this work is complementary to \cite{Lazo2017}, in
terms of clarifying the geometric nature of the objects at play and writing down a set of
equations that is Lorentz invariant, as opposed to the ones originally derived. This issue
is also remedied in \cite{Lazo2021}.

The work is organised as follows: in \cref{ch:herglotz} we introduce the Herglotz
variational principle and show how it can be equivalently formulated as a constrained
optimisation problem. This allows one to use calculus of variations to derive the correct
Herglotz equations of motion, especially relevant for field theories. In
\cref{ch:einstein} we apply this language to a dissipative version of the Einstein-Hilbert
Lagrangian to derive its field equations. In \cref{ch:significance} we discuss how these
equations differ from the ones originally derived in \cite{Lazo2017} and why they are a
Lorentz invariant generalisation of them.

\end{document}
