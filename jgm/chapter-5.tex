\documentclass[../main.tex]{subfiles}

\begin{document}
The are three main ideas presented in this article.

Firstly, we show how the Herglotz problem can be turned from a constrained optimisation problem to an unconstrained one by promoting the action dependence to a dynamic degree of freedom and using Lagrange multipliers to implement the non-holonomic constraint.

Secondly, we describe how the Einstein-Hilbert Lagrangian can be modified with an action dependence in a coordinate-independent manner. This allows one to derive a correct, Lorentz-invariant set of field equations that remedy the issues present in previous derivations. 

Finally, the computation performed constitutes an important example for the ongoing development of contact geometry and its applications, since it is a singular second-order field theory. Having a concrete example at hand will aid to understand these systems.

There are various avenues for future follow-up work. One can consider more general dissipation terms to add to the Einstein-Hilbert Lagrangian, and study their phenomenology. It will also be interesting to consider the boundary effects in the case of manifolds with boundary (the appropriate Gibbons-Hawking-York term). 

General relativity has several equivalent formulations \cite{gaset2019,Vey_2015}. It would be interesting to add dissipation to these formalisms and study their properties and relations.

Finally, the current tools in contact geometry fall short of completely describing this kind of Lagrangians. A more general geometric structure, akin to multisymplectic geometry, needs to be developed for more general action-dependent Lagrangians in order to describe relevant theories. In this line, the multicontact structure recently presented in \cite{de_leon_multicontact_2022} could be the adequate geometric framework for action-dependent gravity.

\end{document}
