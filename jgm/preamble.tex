% -------------------
% PACKAGES
% Basic font setup
\usepackage[utf8]{inputenc}
\usepackage[T1]{fontenc}
\usepackage{lmodern}

% Figures
\usepackage{graphicx}
\usepackage[font={footnotesize, sf}, labelfont=bf]{caption} 

% Maths tools
\usepackage{amsmath, amssymb, mathtools}
\usepackage{amsthm, thmtools}
\usepackage{tikz-cd}
\usepackage{siunitx}

\sisetup{detect-family = true, group-minimum-digits = 4}

% TOC Format
% \usepackage[titles]{tocloft}
% \renewcommand{\cftchappagefont}{\sffamily}   
% \renewcommand{\cftsecpagefont}{\sffamily}   
% \renewcommand{\cftsubsecpagefont}{\sffamily}   

% References
\usepackage{csquotes}

% Utilites
\usepackage{enumerate}

% -------------------
% CUSTOMISATION
% Geometry setup
\usepackage{geometry}
\usepackage{setspace}
\geometry{
	a4paper,
	right = 2.5cm,
	left = 2.5cm,
	bottom = 3cm,
	top = 3cm
}
\renewcommand{\baselinestretch}{1.3}

% -------------------
% Reference setup
\usepackage{hyperref}
\usepackage[english, capitalise, noabbrev]{cleveref}
\hypersetup{
	colorlinks,
	linkcolor = {red!50!blue},
	citecolor = {red!50!blue},
	urlcolor = {red!50!blue},
	linktoc = page
}

% -------------------
% BIBLIOGRAPHY
\usepackage[style=alphabetic, citestyle=alphabetic]{biblatex}
\usepackage[english]{babel}
\addbibresource{refs.bib}

% -------------------
% Theorem environments
\newcommand{\qedtriangle}{\ensuremath{\triangle}}
\newcommand{\qedtriangledown}{\ensuremath{\bigtriangledown}}
\allowdisplaybreaks

\declaretheoremstyle[spaceabove=6pt, spacebelow=6pt, headfont=\bfseries,
notefont=\normalfont, notebraces={(}{)}, qed=\qedtriangle]{definition}
\declaretheoremstyle[spaceabove=6pt, spacebelow=6pt, headfont=\bfseries,
notefont=\normalfont, notebraces={(}{)}, qed=\qedtriangledown]{example}

\declaretheorem[name=Theorem, refname={theorem,theorems}, Refname={Theorem,Theorem},
numberwithin = chapter ]{theorem}

\declaretheorem[name=Proposition, refname={proposition,propositions},
Refname={Proposition,Propositions}, numberlike=theorem]{proposition}

\declaretheorem[name=Lemma, refname={lemma,lemmas},
Refname={Lemma,Lemmas}, numberlike=theorem]{lemma}

\declaretheorem[name=Definition, style=definition, refname={definition,definitions},
Refname={Definitio,Definitions}, numberlike=theorem]{definition}

\declaretheorem[name=Example, style=example, refname={example,examples},
Refname={Example,Examples}, numberlike=theorem]{example}

\declaretheorem[name=Remark, style=remark, refname={remark,remarks},
Refname={Remark,Remarks}, numberlike=theorem]{remark}

% -------------------
% LAYOUT
\usepackage[bf,sf,small,pagestyles]{titlesec}
\usepackage{titling}

% Pagestyle defintions
\newpagestyle{main}[\sffamily \footnotesize]{
	\sethead*{\ifthesection{{\bfseries \thesection} \sectiontitle}{}}{}{{\bfseries Chapter \thechapter.} \chaptertitle}
	\headrule
	\footrule
	\setfoot*{}{}{\thepage}
}
\renewpagestyle{plain}[\sffamily \footnotesize]{
	\footrule
	\setfoot*{}{}{\thepage}
}
\renewpagestyle{empty}{}

% Format of chapter titles
\titleformat{\chapter}[block]{\sffamily \bfseries \Huge}{\filleft \large Chapter \Huge \thechapter\\}{0pt}{\Huge \titlerule[1pt] \vspace{1ex} \filleft}

% -------------------
% SUBFILES
\usepackage{subfiles}
