\documentclass[12pt]{article}
% -------------------
% PACKAGES
% Basic font setup
\usepackage[english]{babel}
\usepackage[utf8]{inputenc}
\usepackage[T1]{fontenc}
\usepackage{lmodern}

% Figures
\usepackage{graphicx}
\usepackage[font={footnotesize, sf}, labelfont=bf]{caption} 
\graphicspath{{figs/}{../figs/}}

% Maths tools
\usepackage{amsmath, amssymb, mathtools}
\usepackage{amsthm, thmtools}
\usepackage{tikz-cd}
\usepackage{siunitx}

\sisetup{detect-family = true}

% TOC Format
% \usepackage[titles]{tocloft}
% \renewcommand{\cftchappagefont}{\sffamily}   
% \renewcommand{\cftsecpagefont}{\sffamily}   
% \renewcommand{\cftsubsecpagefont}{\sffamily}   

% References
\usepackage{csquotes}

% Utilites
\usepackage{enumerate}

% -------------------
% CUSTOMISATION
% Geometry setup
\usepackage{geometry}
\usepackage{setspace}
\geometry{
	a4paper,
	right = 2.5cm,
	left = 2.5cm,
	bottom = 3cm,
	top = 3cm
}
\renewcommand{\baselinestretch}{1.3}

% -------------------
% Reference setup
\usepackage{hyperref}
\usepackage[english, capitalise, noabbrev]{cleveref}
\hypersetup{
	colorlinks,
	linkcolor = {red!50!blue},
	citecolor = {red!50!blue},
	urlcolor = {red!50!blue},
	linktoc = page
}

% -------------------
% BIBLIOGRAPHY
\usepackage[style=alphabetic, citestyle=alphabetic]{biblatex}
\addbibresource{refs.bib}

% -------------------
% Theorem environments
\newcommand{\qedtriangle}{\ensuremath{\triangle}}
\newcommand{\qedtriangledown}{\ensuremath{\bigtriangledown}}
\allowdisplaybreaks

\declaretheoremstyle[spaceabove=6pt, spacebelow=6pt, headfont=\bfseries,
notefont=\normalfont, notebraces={(}{)}, qed=\qedtriangle]{definition}
\declaretheoremstyle[spaceabove=6pt, spacebelow=6pt, headfont=\bfseries,
notefont=\normalfont, notebraces={(}{)}, qed=\qedtriangledown]{example}

\declaretheorem[name=Theorem, refname={theorem,theorems}, Refname={Theorem,Theorem},
numberwithin = chapter ]{theorem}

\declaretheorem[name=Proposition, refname={proposition,propositions},
Refname={Proposition,Propositions}, numberlike=theorem]{proposition}

\declaretheorem[name=Lemma, refname={lemma,lemmas},
Refname={Lemma,Lemmas}, numberlike=theorem]{lemma}

\declaretheorem[name=Definition, style=definition, refname={definition,definitions},
Refname={Definitio,Definitions}, numberlike=theorem]{definition}

\declaretheorem[name=Example, style=example, refname={example,examples},
Refname={Example,Examples}, numberlike=theorem]{example}

\declaretheorem[name=Remark, style=remark, refname={remark,remarks},
Refname={Remark,Remarks}, numberlike=theorem]{remark}

% -------------------
% LAYOUT
\usepackage[bf,sf,small,pagestyles]{titlesec}
\usepackage{titling}

% Pagestyle defintions
\newpagestyle{main}[\sffamily \footnotesize]{
	\sethead*{\ifthesection{{\bfseries \thesection} \sectiontitle}{}}{}{{\bfseries Chapter \thechapter.} \chaptertitle}
	\headrule
	\footrule
	\setfoot*{}{}{\thepage}
}
\renewpagestyle{plain}[\sffamily \footnotesize]{
	\footrule
	\setfoot*{}{}{\thepage}
}
\renewpagestyle{empty}{}

% Format of chapter titles
\titleformat{\chapter}[block]{\sffamily \bfseries \Huge}{\filleft \large Chapter \Huge \thechapter\\}{0pt}{\Huge \titlerule[1pt] \vspace{1ex} \filleft}

% -------------------
% SUBFILES
\usepackage{subfiles}

\newcommand{\set}[1]{\left\{ #1 \right\}}
\newcommand{\abs}[1]{\left\lvert #1 \right\rvert}

\newcommand{\N}{\mathbb{N}}
\newcommand{\Z}{\mathbb{Z}}
\newcommand{\R}{\mathbb{R}}
\newcommand{\F}{\mathbb{F}}
\renewcommand{\L}{\mathcal{L}}

\DeclareMathOperator{\Sym}{Sym}
\DeclareMathOperator{\Lor}{Lor}
\newcommand{\ext}{\mathsf{\Lambda}}
\newcommand{\ud}{\,\mathrm{d}}
\newcommand{\id}{\,\mathrm{id}}
\renewcommand{\d}{\mathrm{d}}

\newcommand{\iprod}{\mathbin{\lrcorner}}

\newcommand{\Top}{\mathsf{Top}}
\newcommand{\Grp}{\mathsf{Grp}}
\newcommand{\Ab}{\mathsf{Ab}}
\newcommand{\Vect}{\mathsf{Vect}}
\newcommand{\Ring}{\mathsf{Ring}}
\newcommand{\Simp}{\mathsf{Simp}}
\newcommand{\Ch}{\mathsf{Ch}}


\makeatletter
\newcommand*{\defeq}{\mathrel{\rlap{%
    \raisebox{0.3ex}{$\m@th\cdot$}}%
  \raisebox{-0.3ex}{$\m@th\cdot$}}%
	=
}
\makeatother

\DeclareMathOperator{\im}{im}

% Category theory
\newcommand{\longto}{\longrightarrow}
\newcommand{\into}{\hookrightarrow}

\newcommand{\MKNN}{M\( k \)NN }
\newcommand{\hombound}[2]{\bar{\partial}_{#1}^{#2}}
\newcommand{\hbound}[1]{\bar{\partial}^{#1}}

% Cycles and boundaries
\newcommand{\cyc}[1]{\ker{\partial}_{#1}}
\newcommand{\bound}[1]{\im{\partial}_{#1}}

\newcommand{\pcyc}[2]{\ker{\partial}_{#1}^{#2}}
\newcommand{\pbound}[2]{\im{\partial}_{#1}^{#2}}


\newpagestyle{pagina}{
\headrule
\sethead*{\sffamily {\bfseries TC1.} Problem sheet 1}{}{\theauthor}
\footrule
\setfoot*{}{}{\sffamily \thepage}
}
\renewpagestyle{plain}{
\footrule
\setfoot*{}{}{\sffamily \thepage}
}
\pagestyle{plain}

\begin{document}
Consider a contact Lagrangian of the form
\begin{equation}
	L(\phi, \partial_\mu \phi, z^\mu). 
\end{equation}
Given a scalar field \( \phi \colon M \to \R \), consider the vector field \( z^\mu(\phi) \) which solves the PDE
\begin{equation}
	\partial_\mu z^\mu(\phi) = L(\phi, \partial_\mu \phi, z^\mu(\phi)).
\end{equation}
We define the action functional as
\begin{equation}
	A(\phi) = \int_M \partial_\mu z^\mu(\phi) (x^\nu) \ud^4 x =
	\int_{\partial M} z^\mu(\phi)(x^\nu) \ud n_\mu.
\end{equation}
We wish to compute the variation of \( A \) to determine which is a necessary and
sufficient condition \( \phi \) must satisfy in order to be an extremum of \( A \) (which
will turn out to be the Herglotz equations). 

Consider a variation about a field \( \phi \), i.e. a parameterised family of fields \(
\phi_\epsilon \) such that \( \phi_0 = \phi \) and \( \phi_\epsilon \vert_{\partial M} =
\phi\vert_{\partial M} \), for \( \epsilon \in (-\delta, \delta) \). Then \( \phi \) is by
definition an extremum of \( A \) if and only if
\begin{equation}
	\deriv{\epsilon} A(\phi_\epsilon) = 0
\end{equation}
for any variation about \( \phi \). And
\begin{equation}
	\deriv{\epsilon} A(\phi_\epsilon) = \int_{\partial M} \deriv{\epsilon}
	z^\mu(\phi_\epsilon)(x^\nu) \ud n_\mu.
\end{equation}
We write \( \zeta^\mu(\phi_0) \) for \( \deriv{\epsilon}z^\mu(\phi_\epsilon) \) and \(
\eta \) for \( \deriv{\epsilon}\phi_\epsilon \). From the
definition of \( z \), we have, introducing the shorthand \( \Phi_\epsilon(x^\nu) =
(\phi_\epsilon(x^\nu), \partial_\mu \phi_\epsilon(x^\nu), z^\mu(\phi_\epsilon)(x^\nu)) \),
\begin{align}
	\partial_\mu \zeta^\mu(\phi_0)
	& = \deriv{\epsilon} \partial_\mu z^\mu(\phi_\epsilon) \\
	& = \deriv{\epsilon} L(\Phi_\epsilon) \\
	& = \frac{\partial L}{\partial \phi}(\Phi_0) \deriv{\epsilon} \phi_\epsilon
	+ \frac{\partial L}{\partial(\partial_\mu \phi)}(\Phi_0) \deriv{\epsilon} \partial_\mu
	\phi_\epsilon 
	+ \frac{\partial L}{\partial z^\mu}(\Phi_0) \zeta^\mu(\phi_0) \\
	& = \frac{\partial L}{\partial \phi}(\Phi_0) \eta
	+ \frac{\partial L}{\partial(\partial_\mu \phi)}(\Phi_0) \partial_\mu \eta
	+ \frac{\partial L}{\partial z^\mu}(\Phi_0) \zeta^\mu(\phi_0)
\end{align}
This gives a PDE for \( \zeta^\mu(\phi_0) \). If \( \frac{\partial L}{\partial
z^\mu} (\Phi_0) \) is the gradient of some function, i.e. there is a function \( f
\) such that \( \partial_\mu f = \frac{\partial L}{\partial z^\mu} (\Phi_0) \),
in which case we say the dependence of \( L \) on \( z \) is \emph{exact}, then we can
solve this equation for \( \zeta^\mu(\phi_0) \). Indeed, following the 1-dimensional case,
if we try the ansatz
\begin{equation}
	\zeta^\mu(\phi_0)(x^\nu) = M^\mu(x^\nu) e^{f(x^\nu)}
\end{equation}
then (writing \( \zeta^\mu \) for \( \zeta^\mu(\phi_0) \))
\begin{equation}
	\partial_\mu \zeta^\mu(x^\nu) = \partial_\mu f(x^\nu) M^\mu(x^\nu) e^{f(x^\nu)} +
	\partial_\mu M^\mu(x^\nu) e^{f(x^\nu)}
\end{equation}
so it must be
\begin{equation}
	\partial_\mu M^\mu = e^{-f}\left(\frac{\partial L}{\partial \phi}(\Phi_0) \deriv{\epsilon}
	\phi_\epsilon + \frac{\partial L}{\partial(\partial_\mu \phi)}(\Phi_0) \deriv{\epsilon}
	\partial_\mu \phi_\epsilon\right). 
\end{equation}

Now
\begin{equation}
	\deriv{\epsilon} A(\phi_\epsilon) = \int_{\partial M} \zeta^\mu \ud n_\mu =
	\int_{\partial M} e^{f}M^\mu \ud n_\mu.
\end{equation}
The field \( \phi \) will be an extremum if and only if this integral vanishes for all
variations about \( \phi \). 

PAS MISTERIÓS: equival a veure que s'anu\l.la la integral sense l'exponencial. 

We rewrite the integral using integration by parts:
\begin{align}
	\int_{\partial M} M^\mu \ud n_\mu 
	& = \int_M \partial_\mu M^\mu \\
	& = \int_M e^{-f}\left(\frac{\partial L}{\partial \phi}(\Phi_0) \eta + \frac{\partial
		L}{\partial(\partial_\mu \phi)}(\Phi_0) \partial_\mu \eta \right) \\
	& = \int_M e^{-f} \frac{\partial L}{\partial \phi}(\Phi_0) \eta
	+ \int_{\partial M} e^{-f} \frac{\partial L}{\partial(\partial_\mu \phi)} \eta
	- \int_M \partial_\mu\left(e^{-f} \frac{\partial L}{\partial(\partial_\mu
	\phi)}\right)\eta
\end{align}
The second integral vanishes because \( \eta \) vanishes on \( \partial M \), since \(
\phi_\epsilon \vert_{\partial M} = \phi\vert_{\partial M} \) for all \( \epsilon \). Using
the product rule for the third integral and recalling that \( \partial_\mu f =
\frac{\partial L}{\partial z^\mu}(\Phi_0) \) we arrive at
\begin{equation}
	\int_{\partial M} M^\mu \ud n_\mu = \int_M \eta e^{-f} \left(\frac{\partial
			L}{\partial \phi}(\Phi_0) - \partial_\mu\left(\frac{\partial L}{\partial(\partial_\mu
	\phi)}(\Phi_0)\right) + \frac{\partial L}{\partial z^\mu}
\frac{\partial L}{\partial(\partial_\mu \phi)} \right).
\end{equation}
So, \( \phi \) will be an extremum of the action if and only if this last integral
vanishes for any variation. And, if it is to vanish for any variation, it must vanish for
any \( \eta \), thus the whole integrand must vanish. Since \( e^{-f} \) is nonzero, this
is equivalent to the Herglotz equation,
\begin{equation}
	\frac{\partial L}{\partial \phi}(\Phi_0) - \partial_\mu\left(\frac{\partial
	L}{\partial(\partial_\mu \phi)}(\Phi_0)\right) + \frac{\partial L}{\partial z^\mu}
	\frac{\partial L}{\partial(\partial_\mu \phi)} = 0
\end{equation}



\end{document}
